% Редакционная информация
\section*{Редакционная информация}

\begin{minipage}{\textwidth}
\textbf{Главный редактор:} \
В.И. Моисеев, профессор и заведующий кафедрой философии, биомедицинской этики и гуманитарных наук Московского государственного медико-стоматологического университета имени А.Е. Евдокимова, д. филос. н., профессор

\vspace{0.5cm}

\textbf{Технический редактор:} \\
Е.Г. Луговская, доцент кафедры русской филологии Приднестровского государственного университета им. Т.Г. Шевченко, к. филол. н., доцент

\vspace{0.5cm}

\textbf{Редакционная коллегия:} \\
\begin{itemize}
    \item В.И. Моисеев, профессор, доктор философских наук
    \item Е.Г. Луговская, доцент, кандидат филологических наук
    \item Н.А. Подзолкова, доцент, кандидат философских наук
    \item Ю.Д. Бухаров, член Союза журналистов России и Российского философского общества
\end{itemize}

\end{minipage}

\vspace{1cm}

% Библиографическое описание
\begin{minipage}{\textwidth}
\textbf{Интегральная философия} [Электронный ресурс]: научный журнал; главный редактор В.И. Моисеев, техн. ред. Е.Г. Луговская. – онлайн издание, 2025. – 354 с.

\vspace{0.5cm}

\textit{Настоящий выпуск журнала «Интегральная философия» представляет первые научные результаты работы Института интегральной науки (ИИН), созданного в январе 2025 года как общественного объединения, посвященного систематической интеграции знания в рамках философии неовсеединства.}

\end{minipage}
