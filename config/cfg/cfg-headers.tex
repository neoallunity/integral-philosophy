% ============================================
% cfg-headers.tex — колонтитулы
% ============================================

\usepackage{fancyhdr}
\usepackage{tikzpagenodes}
\usepackage{xcolor}

\setlength{\headheight}{18.5pt}
% Два режима: строгий (без текстуры) и "жар" (с текстурой)
\fancypagestyle{journalPlain}{%
  \togglefalse{journaltexture}
  \fancyhf{}
}
\fancypagestyle{journalHot}{%
  \toggletrue{journaltexture}
  \fancyhf{}
}

% Титул: без общих оверлеев (они там свои)
\fancypagestyle{journalTitle}{%
  \togglefalse{journaltexture}
  \fancyhf{}
}

\pagestyle{journalHot}

% Смелее: угловые акценты + верхняя линия + «плашка» внизу (overlay)
\AddToHook{shipout/foreground}{%
  \begin{tikzpicture}[remember picture,overlay]

    % Верхняя линия
    \draw[JournalAccent!65,line width=0.6pt]
      ([xshift=\JournalMarginX,yshift=\JournalHeaderY]current page.north west) --
      ([xshift=-\JournalMarginX,yshift=\JournalHeaderY]current page.north east);

    % Второй слой (градиент-эффект за счет 2 линий)
    \draw[JournalAccent2!45,line width=1.2pt,opacity=0.25]
      ([xshift=14mm,yshift=-13.3mm]current page.north west) --
      ([xshift=-14mm,yshift=-13.3mm]current page.north east);

    % Угловые "скобы" сверху
    \draw[JournalAccent!65,line width=0.6pt]
      ([xshift=\JournalMarginX,yshift=-10mm]current page.north west) -- ++(10mm,0) -- ++(0,-6mm);
    \draw[JournalAccent!65,line width=0.6pt]
      ([xshift=-\JournalMarginX,yshift=-10mm]current page.north east) -- ++(-10mm,0) -- ++(0,-6mm);

    % Нижняя цветная плашка
    \fill[JournalAccent!9]
      ([xshift=0mm,yshift=0mm]current page.south west) rectangle
      ([xshift=0mm,yshift=\JournalFooterH]current page.south east);

    % «Жар»: диагональные штрихи по плашке
    \JournalIfTexture{%
      \foreach \x in {0,\JournalHatchStep,...,22} {
        \draw[JournalAccent2!25,line width=0.35pt,opacity=0.55]
          ([xshift=\x cm,yshift=0mm]current page.south west) -- ++(\JournalHatchDx,\JournalFooterH);
      }%
    }

    % Тонкая линия над футером
    \draw[JournalAccent!55,line width=0.5pt]
      ([xshift=\JournalMarginX,yshift=\JournalFooterH]current page.south west) --
      ([xshift=-\JournalMarginX,yshift=\JournalFooterH]current page.south east);

  \end{tikzpicture}%
}

% Наполнение колонтитулов
\fancyhead[LE,RO]{\textcolor{JournalAccent!85}{\thepage}}
\fancyhead[RE]{\textit{\nouppercase{\leftmark}}}
\fancyhead[LO]{\textit{\nouppercase{\rightmark}}}

% Микро-акцент рядом с номером страницы
\fancyhead[LE]{\textcolor{JournalAccent2!65}{\large\raisebox{0.2ex}{\textbullet}}\;\textcolor{JournalAccent!85}{\thepage}}
\fancyhead[RO]{\textcolor{JournalAccent!85}{\thepage}\;\textcolor{JournalAccent2!65}{\large\raisebox{0.2ex}{\textbullet}}}

% Низ: слева URL, по центру название, справа "Выпуск · Год"
\fancyfoot[L]{\footnotesize\textcolor{JournalAccent!80}{\journalurl}}
\fancyfoot[C]{\footnotesize\textcolor{JournalAccent!80}{\textit{Интегральная философия}}}
\fancyfoot[R]{\footnotesize\textcolor{JournalAccent!80}{\journalissue\;•\;\journalyear}}

\renewcommand{\headrulewidth}{0pt}
\renewcommand{\footrulewidth}{0pt}
\renewcommand{\chaptermark}[1]{\markboth{#1}{}}
\renewcommand{\sectionmark}[1]{\markright{\thesection\ #1}}
