% ============================================
% cfg-articles.tex — единый шаблон статьи и служебных разделов
% ============================================

\RequirePackage{xparse}

% Статья: автор как глава, заголовок как ненумерованный раздел, запись в TOC и колонтитулы.
% Использование:
%   \JournalArticle{Фамилия И.О.}{Название статьи}
\NewDocumentCommand{\JournalArticle}{m m}{%
  \chapter{#1}
  \section*{#2}
  \addcontentsline{toc}{section}{#2}%
  \markright{#2}%
}

% Русская аннотация + ключевые слова
%   \RUAbstract{Текст...}{Ключевые слова: ...}
\NewDocumentCommand{\RUAbstract}{m m}{%
  \noindent #1

  \vspace{0.3cm}
  \noindent\textit{#2}
  \vspace{0.7cm}
}

% Английская аннотация + keywords
%   \ENAbstract{Text...}{Keywords: ...}
\NewDocumentCommand{\ENAbstract}{m m}{%
  \begin{EN}
  \noindent #1

  \vspace{0.3cm}
  \noindent\textit{#2}
  \end{EN}
  \vspace{0.7cm}
}

% ============================================
% Управление списком статей
% ============================================
\newif\ifprintabstract

\ExplSyntaxOn % Start expl3 syntax mode

% --- Internal variables ---
\clist_new:N \g_journal_articles_clist

% --- Internal functions ---
% Appends an article identifier to the global list
\cs_new_protected:Nn \journal_add_article:n
  {
    \clist_put_right:Nn \g_journal_articles_clist {#1}
  }

% Maps through the article list and inputs the main file for each
\cs_new_protected:Nn \journal_print_articles:
  {
    \clist_map_inline:Nn \g_journal_articles_clist { \ifprintabstract
    \begin{english}
    % Lugowska H. - English abstract
    \subsection{Lugowska H. PROJECTIVE-MODAL ONTOLOGY OF V. SOLOVYOV'S IMAGE IN D.S. MEREZHKOVSKY'S CRITICAL ESSAY "THE SILENT PROPHET": AN APPLICATION OF FORMAL APPARATUS TO LITERARY-CRITICAL DISCOURSE}
    \label{subsec:lugowska-en

    {\itshape }

    This article proposes an application of the projective-modal ontology (PMO) framework—a formal philosophical apparatus developed by V.I. Moiseev for examining the structural relationship between essence (modus), manifestation (mode), and constraining conditions (models)—to the analysis of existential portrayal as a critical method in D.S. Merezhkovsky's literary-critical discourse. Specifically, the study reconstructs Merezhkovsky's characterisation of Vladimir Solovyov in the critical essay "The Silent Prophet" (1908) through the systematic application of PMO's formal apparatus, demonstrating how existential portrayal functions as a disciplined procedure of projective variation of a subject's fundamental nature across a multiplicity of ontological, temporal, and epistemological models.

    The projective-modal ontology, whilst originally developed for philosophical-logical inquiry, provides a metalanguage capable of formalising the conceptual structures underlying literary criticism. The PMO framework operates through a seven-place predicate establishing relationships between: a modus (the source or generator of being—in this case, Solovyov's fundamental gnostic-contemplative nature); modes (particular manifestations or aspects of that source); models (limiting conditions under which modes do or do not emerge); projectors and surjectors (operations of restriction and expansion); and contexts of determination.

    Critically, the concept of proper models—the set of conditions upon which a modus is capable of generating modes—proves essential: a modus cannot generate authentic modes on all conceivable conditions. For instance, a revolutionary pragmatist model lies entirely outside Solovyov's ontological capacity, rendering any attempt to project his essence onto such a model structurally impossible rather than merely improbable.

    The study identifies Solovyov's base modus as immanently gnostic and contemplative, rather than a neutral subject oscillating between contemplation and action. This fundamental characterisation renders the revolutionary pragmatist model improper—incapable of generating authentic modes. Through systematic textual analysis, the investigation reveals three proper temporal models upon which Solovyov's authentic modes emerge: (1) the past, generating the mode of "restorer" and secret Slavophile; (2) the present, generating the mode of "conservator" and supporter of collapsing structures; and (3) an eschatological future, generating the mode of "eschatologist" and herald of apocalyptic catastrophe. Conversely, attempts to project Solovyov onto models of revolutionary action produce not legitimate modes but social masks—in particular, the artificial persona of the philosophical systematist composed of "ten volumes" of rigorous argumentation, a pseudo-mode bearing all markers of inauthenticity through its excessive smoothness and polish.

    The article formalises Merezhkovsky's critical method as an asymmetrical conflict between two second-order modes: (1) the hidden prophet—ontologically true yet epistemologically inaccessible, incapable of articulation within public discourse; and (2) the public philosopher—discursively expressed yet ontologically false, a mere mask concealing the authentic prophetic truth. This generates the fundamental ontological aporia structuring Solovyov's tragedy: the authentic prophetic mode remains structurally inexpressible within the model of public philosophical discourse, whilst anything expressed through such discourse fails to capture the prophetic truth. The concept of "quiet eddy" (omut) is reinterpreted not as an actual synthesis of opposites but as an eschatological limit—a fusion achievable only in finite time, perpetually deferred beyond historical reality. Solovyov, according to Merezhkovsky's interpretation, exists before the attainment of such synthesis, dying before unity is achieved.

    The tragedy of the "silent prophet" is formalised as an ontological aporia arising from three structural impossibilities: (1) a fundamental mismatch between the modus required by historical epoch (pragmatist activist) and the actual modus of the subject (gnostic contemplative); (2) blockage of the projector operation, rendering revolutionary action a structural impossibility rather than a moral choice; and (3) fission of authentic and inauthentic modes, whereby the only possible public expression necessarily betrays the underlying truth. The muteness of prophecy emerges as a limit function: as one approaches absolute prophetic truth, the capacity for expression approaches zero. Absolute authenticity correlates with absolute inexpressibility.

    The investigation demonstrates the heuristic potential of PMO for formalising existential portrayal as a critical method. Through coordination of PMO structures with linguistic-rhetorical analysis, the study reveals that Merezhkovsky's critical procedure is isomorphic to ancient dialectical method—a systematic procedure of projective variation capable of precise formal representation. The formal apparatus provides a metalanguage unifying macro-logical structures (associative-semantic fields and conceptual networks identified through philological analysis) with micro-logical structures (specific rhetorical devices and textual markers). This coordination permits rigorous diagnosis of the ontological nature of Solovyov's tragedy as portrayed by Merezhkovsky, transforming intuitive literary criticism into a formally specifiable procedure.

    The study concludes that PMO-based formalisation yields five principal findings: (1) identification of the base modus as immanently gnostic, rendering pragmatic revolution an improper model; (2) reconstruction of temporal architecture through three proper models generating authentic modes; (3) reconceptualisation of the "quiet eddy" as an eschatological limit-transition; (4) formalisation of prophetic muteness as an ontological aporia; and (5) diagnosis of three types of ontological aporias structuring Solovyov's tragic characterisation. The article thus establishes PMO as a productive framework for literary-critical analysis, extending its application beyond its original philosophical domain to encompass the formal structures underlying existential portrayal in twentieth-century critical discourse.

    \paragraph{Keywords:} {\itshape Projective-modal ontology, V.S. Solovyov, D.S. Merezhkovsky, existential portrayal, mode-modus-model, gnosticism-pragmatism, temporality, tragedy of muteness, formalisation of critical method, ontological aporia, silent prophet, twentieth-century Russian literary criticism}
    \end{english}
\else
    \begin{russian}
    % Луговская Е.Г. - русская статья

    \subsection{\texorpdfstring{\textbf{Луговская Е.Г., Е.К. Грудина. ПРОЕКТИВНО-МОДАЛЬНАЯ ОНТОЛОГИЯ ОБРАЗА В.С. СОЛОВЬЕВА В КРИТИЧЕСКОЙ СТАТЬЕ МЕРЕЖКОВСКОГО Д.С. «НЕМОЙ ПРОРОК»: ОПЫТ ПРИМЕНЕНИЯ ФОРМАЛЬНОГО АППАРАТА ПМО К ЛИТЕРАТУРНО-КРИТИЧЕСКОМУ ДИСКУРСУ}}{Луговская Е.Г., Е.К. Грудина. ПРОЕКТИВНО-МОДАЛЬНАЯ ОНТОЛОГИЯ ОБРАЗА В.С. СОЛОВЬЕВА В КРИТИЧЕСКОЙ СТАТЬЕ МЕРЕЖКОВСКОГО Д.С. «НЕМОЙ ПРОРОК»: ОПЫТ ПРИМЕНЕНИЯ ФОРМАЛЬНОГО АППАРАТА ПМО К ЛИТЕРАТУРНО-КРИТИЧЕСКОМУ ДИСКУРСУ}}
    \label{subsec:lugowska-ru

    \textit{В статье предложена реконструкция образа В.С. Соловьева в критической статье Д.С. Мережковского «Немой пророк» на основе аппарата Проективно-модальной онтологии (ПМО) В.И. Моисеева. Продемонстрировано, что критический метод Мережковского может быть формализован как систематическое варьирование модуса личности философа на множестве темпоральных, онтологических и эпистемологических моделей. Установлено, что базовый модус Соловьева идентифицируется Мережковским как имманентно гностический (созерцательный), что делает революционный прагматизм невозможной моделью образования моды деятеля. Трагедия «немого пророка» интерпретируется как онтологическая апория, возникающая из структурной невозможности одновременной выражаемости и аутентичности пророческой истины. Розовый башмачок реконструируется как материализованная модель прошлого, функционирующая одновременно как условие образования моды реставратора и как точка амбивалентного перехода между созерцанием и деятельностью. Метафора «омута» переинтерпретируется как эсхатологический закон неизбежного синтеза противоположностей, недостижимый в историческом времени. Исследование демонстрирует эвристический потенциал ПМО как метаязыка для описания концептуальных структур, выявленных лингвориторическим анализом.}

    \textbf{Проективно-модальная онтология (ПМО), В.С. Соловьев, Д.С. Мережковский, экзистенциальное портретирование, модус-мода-модель, формализация критического метода}

    \textbf{Введение: проблема формализации философского портретирования}
    Владимир Сергеевич Соловьев (1853–1900) функционирует в метадискурсе русской культуры как фигура парадоксальная: образ «великого учителя» символистов одновременно маркируется как «двойственный», «нераскрытый», «невидимый» (Блок, 1906; Розанов, 1900; Трубецкой, 1913). Критическая рецепция Д.С. Мережковского в статье «Немой пророк» (1908) представляет попытку экзистенциального портретирования, которое не сводится ни к биографическому описанию, ни к философской экспликации, но предстает как развертывание личности сквозь противоречивые модусы бытия.
    Предшествующий лингвориторический анализ (Луговская, Грудина, 2024) выявил, что Мережковский конструирует образ Соловьева через операционализацию бинарной оппозиции \textit{«философ-созерцатель» / «человек-деятель»}	extit{\textbf{,}} актуализированной в общественном дискурсе начала XX века в контексте идеологических дискуссий о революции и реформации. Анализ выявил также систему риторических приемов (оксюмороны, катахрезы, апоретические вопросы, символические детали), порождающих семантическое напряжение и вовлекающих читателя в активный процесс конструирования смысла образа.
    Однако остается нерешенной проблема формализации критического метода как систематической процедуры, позволяющей локализовать точки экзистенциальной трагедии личности в ее онтологической структуре.
    Настоящее исследование направлено на применение аппарата Проективно-модальной онтологии (ПМО) В.И. Моисеева (Моисеев, 2002, 2004) к анализу образа Соловьева у Мережковского\textbf{. }ПМО рассматривается здесь не как универсальный метод анализа художественного текста, но как метаязык описания для концептуальных структур, выявленных лингвориторическим анализом\textbf{.} Допускается метатеоретическое расширение: введение понятия множества собственных моделей для описания условий, при которых модус (личность) способен образовывать свои моды (аспекты существования).
    \textbf{Цель исследования:} формализовать концептуальную структуру образа В.С. Соловьева посредством предикатного синтаксиса ПМО, выявив онтологическую архитектонику трагедии «немого пророка» и верифицировав результаты против установленных лингвориторическим анализом фактов.
    Теоретические основания: Проективно-модальная онтология В.И. Моисеева
    2.1. Базовый аппарат ПМО
    Проективно-модальная онтология В.И. Моисеева разработана на основе онтологии С. Лесьневского с использованием категориального синтаксиса (Моисеев, 2002). Центральное место в ПМО занимает \textbf{семиместный предикат Mod}, имеющий следующий категориальный тип:
    \begin{center}
    \includegraphics[width=1\linewidth]{articles/lugowska/images/image1.png}
\end{center}
    где \includegraphics[width=1\linewidth]{articles/lugowska/images/image2.png}– тип предложений, \includegraphics[width=1\linewidth]{articles/lugowska/images/image3.png}– тип имен, \includegraphics[width=1\linewidth]{articles/lugowska/images/image4.png}– произвольный категориальный тип, \includegraphics[width=1\linewidth]{articles/lugowska/images/image5.png}– тип двуместных функторов. 
    \textbf{Определение: семиместный предикат Mod (a,b,c,f,d,h,$\alpha$)}
    Предикат Mod выражает отношение между семью элементами:
    \begin{adjustbox}{center}\begin{tabular}{ |l|l|l|l|l| }
      \hline
      {\centering \textbf{Позиция} \par} & {\centering \textbf{Переменная} \par} & {\centering \textbf{Название} \par} & {\centering \textbf{Интерпретация} \par} &  \\
      \hline
      {\centering 1 \par} & \begin{center}
    \includegraphics[width=1\linewidth]{articles/lugowska/images/image6.png}
\end{center} & {\centering \textbf{мода} \par} & {\centering Аспект, проявление, образуется из модуса \par} & \\
      \hline
      {\centering 2 \par} & \begin{center}
    \includegraphics[width=1\linewidth]{articles/lugowska/images/image7.png}
\end{center} & {\centering \textbf{модус} \par} & {\centering Источник, генератор бытия, основание \par} & \\
      \hline
      {\centering 3 \par} & \begin{center}
    \includegraphics[width=1\linewidth]{articles/lugowska/images/image8.png}
\end{center} & {\centering \textbf{модель} \par} & {\centering Ограничивающее условие, обстоятельство \par} & \\
      \hline
      {\centering 4 \par} & \begin{center}
    \includegraphics[width=1\linewidth]{articles/lugowska/images/image9.png}
\end{center} & {\centering \textbf{проектор} \par} & {\centering Операция ограничения модуса до моды \par} & \\
      \hline
      {\centering 5 \par} & \begin{center}
    \includegraphics[width=1\linewidth]{articles/lugowska/images/image10.png}
\end{center} & {\centering \textbf{модуль} \par} & {\centering Начало расширения моды до модуса \par} & \\
      \hline
      {\centering 6 \par} & \begin{center}
    \includegraphics[width=1\linewidth]{articles/lugowska/images/image11.png}
\end{center} & {\centering \textbf{сюръектор} \par} & {\centering Операция расширения моды до модуса \par} & \\
      \hline
      {\centering 7 \par} & \begin{center}
    \includegraphics[width=1\linewidth]{articles/lugowska/images/image12.png}
\end{center} & {\centering \textbf{спецификатор} \par} & {\centering Контекст определения \par} & \\
      \hline
    \end{tabular}\end{adjustbox}\

    \textbf{Неформальное прочтение:}
    \begin{center}
    \includegraphics[width=1\linewidth]{articles/lugowska/images/image13.png}
\end{center}




    \textbf{Визуальная схема отношений (Моисеев, 2002: 217):}
                  f (проектор)
        модус b  ───────────►  мода a
          ▲                       ▼
          │                       │
          └─────────────────────────
               h (сюръектор)
           
    модель c            модуль d
    (ограничение)      (расширение)

    На основе семиместного предиката Mod определяются сокращенные формы, позволяющие исключать кванторы по отдельным компонентам [Моисеев, 2002].

    \textbf{Гипотеза применения ПМО к литературно-критическому дискурсу}
    В настоящей работе авторами предлагается интерпретационное расширение ПМО в части применения ПМО к образу личности, представленной в литературно-критическом дискурсе.

    Базовая гипотеза нашего рассуждения состоит в том, что литературно-критический образ реальной личности может быть реконструирован как модус, реализующийся в различных модах (аспектах существования) на множестве моделей (темпоральных, онтологических, эпистемологических условиях).
    Вводим понятие множества собственных моделей \includegraphics[width=1\linewidth]{articles/lugowska/images/image14.png}как совокупности условий \includegraphics[width=1\linewidth]{articles/lugowska/images/image15.png}, на которых модус \includegraphics[width=1\linewidth]{articles/lugowska/images/image16.png}способен образовывать свои моды. Формально: 
    \begin{center}
    \includegraphics[width=1\linewidth]{articles/lugowska/images/image17.png}
\end{center}
    То есть модель \includegraphics[width=1\linewidth]{articles/lugowska/images/image15.png} входит в множество собственных моделей модуса \includegraphics[width=1\linewidth]{articles/lugowska/images/image16.png}, если существует проектор \includegraphics[width=1\linewidth]{articles/lugowska/images/image9.png}, образующий из \includegraphics[width=1\linewidth]{articles/lugowska/images/image16.png}в условии \includegraphics[width=1\linewidth]{articles/lugowska/images/image15.png}некоторую моду \includegraphics[width=1\linewidth]{articles/lugowska/images/image6.png}. 
    Обратно, если модель \includegraphics[width=1\linewidth]{articles/lugowska/images/image15.png}не входит в множество собственных моделей модуса:
    \begin{center}
    \includegraphics[width=1\linewidth]{articles/lugowska/images/image18.png}
\end{center}
    то есть не существует проектора\textbf{,} который смог бы произвести моду из модуса \includegraphics[width=1\linewidth]{articles/lugowska/images/image16.png}в модели \includegraphics[width=1\linewidth]{articles/lugowska/images/image15.png}. Это – онтологическая невозможность\textbf{,} а не выбор или негативное суждение.

    ПМО используется здесь не как самостоятельный метод анализа литературно-критического текста Д. С. Мережковского\textbf{, }но как\textbf{ }метаязык для описания и формализации концептуальных структур, выявленных лингвориторическим анализом.

    Предлагаем следующую схему использования конструкций ПМО для указанных целей. 

    \begin{adjustbox}{center}\begin{tabular}{ |l|l|l|l| }
      \hline
      {\centering \textbf{Уровень анализа} \par} & {\centering \textbf{Инструментарий} \par} & {\centering \textbf{Результат} \par} &  \\
      \hline
      \textbf{Уровень 1: Текстовый анализ} & Лингвориторический анализ АСП, риторические приемы & Выявление концептов «философ-созерцатель» / «человек-деятель», символических деталей, семантического напряжения \ 
      \hline
      \textbf{Уровень 2: Концептуальная структура} & ПМО-синтаксис (Mod-предикаты) & Формализация концептов как модусов и мод, определение моделей и условий \ 
      \hline
      \textbf{Уровень 3: Онтологическая архитектоника} & Анализ множеств собственных моделей, валентные определения & Выявление точек трагедии, онтологических апорий, невозможных мод \ 
      \hline
      \textbf{Уровень 4: Верификация} & Сопоставление результатов ПМО и Л-анализа & Подтверждение или коррекция интерпретаций \ 
      \hline
    \end{tabular}\end{adjustbox}\

    \textbf{ПМО-анализ образа Соловьева: реконструкция концептуальной структуры}
    Представим базовый модус: \textit{В. С. Соловьев как гностик-созерцатель}
    Д. С. Мережковский эксплицитно определяет В. С. Соловьева не как нейтральный субъект, способный быть либо созерцателем, либо деятелем, но как имманентно гностического (созерцательного) модуса:
    «Вл. Соловьев – гностик, может быть, последний великий гностик всего христианства» [Мережковский, 1908: 133].
    «Для него сущность догмата открывается не воле сначала и потом разуму, а, наоборот, сначала разуму, потом воле. Он – рационалист, как всякий гностик» [Там же].

    ПМО-формализация:
    \begin{center}
    \includegraphics[width=1\linewidth]{articles/lugowska/images/image19.png}
\end{center}
    Обозначим этот модус как \includegraphics[width=1\linewidth]{articles/lugowska/images/image16.png}для краткости. Модус \includegraphics[width=1\linewidth]{articles/lugowska/images/image16.png}обладает следующими сущностными структурными предикатами, которые принадлежат самому модусу, а не образуются как моды на различных моделях:
    \begin{itemize}
      \item \includegraphics[width=1\linewidth]{articles/lugowska/images/image20.png}: Приоритет разума (гнозис) над волей 
      \item \includegraphics[width=1\linewidth]{articles/lugowska/images/image21.png}: «Богоделание вытекает из богопознания» 
      \item \includegraphics[width=1\linewidth]{articles/lugowska/images/image22.png}: Рационалистичность 
      \item \includegraphics[width=1\linewidth]{articles/lugowska/images/image23.png}: Консервативность и ретроградность 
      \item \includegraphics[width=1\linewidth]{articles/lugowska/images/image24.png}: Реставраторство («Да будет снова то, что было») 
    \end{itemize}

    Определение моды модуса Y в контексте $\alpha$:
    \begin{center}
    \includegraphics[width=1\linewidth]{articles/lugowska/images/image25.png}
\end{center}
    То есть \includegraphics[width=1\linewidth]{articles/lugowska/images/image6.png} есть мода модуса \includegraphics[width=1\linewidth]{articles/lugowska/images/image16.png}в некотором контексте \includegraphics[width=1\linewidth]{articles/lugowska/images/image12.png}, если существуют условие (модель) \includegraphics[width=1\linewidth]{articles/lugowska/images/image8.png}, проектор \includegraphics[width=1\linewidth]{articles/lugowska/images/image9.png}, которые образуют \includegraphics[width=1\linewidth]{articles/lugowska/images/image6.png} из \includegraphics[width=1\linewidth]{articles/lugowska/images/image16.png}. 
    На этом этапе моделирование необходимо сделать отдельное замечание о связи ПМО-описания с ассоциативно-семантическими полями (АСП)\textsuperscript{\footnote{² О связи с ассоциативно-семантическими полями (АСП): Ассоциативно-семантические поля, выявленные лингвориторическим анализом (Луговская, Грудина), описывают те лексические, образные и риторические средства, которые Мережковский использует для артикуляции различных мод модуса. В ПМО эти поля переструктурируются через предикатный синтаксис как системы условий (моделей), на которых образуются моды. АСП остается как описательный инструмент; ПМО обеспечивает формальное описание их онтологической функции.}}: 
    В лингвориторическом анализе (Луговская, Грудина, 2024) выявлены два ассоциативно-семантических поля:
    \begin{itemize}
      \item АСП «философ-созерцатель» (созерцание, погруженность в идеальное, подполье, тайна)
      \item АСП «человек-деятель» (реформатор, деятельность, видимая сторона)
    \end{itemize}
    В ПМО-описании эти семантические поля переструктурируются как:
    \begin{itemize}
      \item АСП соответствует множеству вторичных предикатов, которые маркируют доступ к моде
      \item Каждый концепт в АСП указывает на условие (модель), при котором образуется мода
      \item Модус \includegraphics[width=1\linewidth]{articles/lugowska/images/image16.png}рассматривается как единственный источник, из которого проектируются различные моды 
    \end{itemize}
    Таким образом, АСП – это описание на естественном языке того, какие лексические и образные средства Д. С. Мережковский использует для артикуляции различных мод модуса. То есть ПМО предоставляет формальную структуру этого варьирования.
    Следующий шаг - собственные модели: темпоральная триада и образование аутентичных мод. 
    Модус не образует моды на всех мыслимых условиях. Множество собственных моделей \includegraphics[width=1\linewidth]{articles/lugowska/images/image14.png}модуса \includegraphics[width=1\linewidth]{articles/lugowska/images/image16.png}– это те условия (модели), на которых модус способен образовывать свои моды. 
    Базовая формула:
    \begin{center}
    \includegraphics[width=1\linewidth]{articles/lugowska/images/image26.png}
\end{center}
    Для модуса В. С. Соловьева как гностика-созерцателя Д. С. Мережковский выявляет три собственные модели – три условия, на которых этот модус образует свои аутентичные моды:
    \begin{center}
    \includegraphics[width=1\linewidth]{articles/lugowska/images/image27.png}
\end{center}
    Модель 1: Прошлое \includegraphics[width=1\linewidth]{articles/lugowska/images/image28.png}и мода реставратора определяется следующим условием модели\textbf{:} Рассмотрение \includegraphics[width=1\linewidth]{articles/lugowska/images/image16.png}в условиях прошлого, истории, традиции. 
    Текстовыми маркерами в этом случае будут выступать:
    «Розовый башмачок – безнадежная романтика прошлого, желание сделать прошлое не только настоящим, но и будущим, таково безумие этого печального рыцаря Прекрасной Дамы» [Мережковский, 1908: 131].
    «Былое надежно; будущее страшно» [Там же: 132].
    «Лучи заходящего солнца, лампадный свет вечерний любил он больше, чем дневной и утренний» [Там же: 132].
    «восстановить, реставрировать три исполинские развалины средневековья: вселенскую монархию… вселенскую церковь… вселенскую догматику» [Там же: 132].
    \textbf{ПМО-формализация}\textbf{:} 
    \begin{center}
    \includegraphics[width=1\linewidth]{articles/lugowska/images/image29.png}
\end{center}
    Читается: «\includegraphics[width=1\linewidth]{articles/lugowska/images/image30.png} есть мода модуса \includegraphics[width=1\linewidth]{articles/lugowska/images/image16.png}(гностика-созерцателя) в модели прошлого \includegraphics[width=1\linewidth]{articles/lugowska/images/image28.png}с проектором \includegraphics[width=1\linewidth]{articles/lugowska/images/image31.png}». 
    Характеристики моды:
    \begin{itemize}
      \item Содержание: романтик былого, тайный славянофил, реставратор
      \item Поле образования: отношение к историческому времени
      \item Символический репрезентант: розовый башмачок
    \end{itemize}

    Рассмотрим символический репрезентант розовый башмачок как материализованную модель и точку амбивалентности, для чего дадим некоторое уточнение к установке лингвориторического анализа\textsuperscript{\footnote{О двойственной функции розового башмачка: первоначальный анализ (Луговская, Грудина) предложил интерпретацию башмачка как имеющего двойную функциональность. Уточнение ПМО-анализа состоит в том, что эта двойственность не просто семантическая (башмачок одновременно указывает на прошлое и на реальность), но онтологическая: башмачок функционирует одновременно (а) как модель (условие образования моды), (б) как амбивалентная точка, где мода реставратора (чистое созерцание) потенциально может трансформироваться в направлении деятельности. Утрата башмачка – это утрата не только символа, но материального медиума, через который только и может образоваться мода в модели прошлого.}}: 
    Розовый башмачок функционирует в критическом тексте Мережковского двойственно:
    Во-первых, как материализованная модель\textbf{ }
    \includegraphics[width=1\linewidth]{articles/lugowska/images/image28.png}\textbf{ }– конкретное физическое условие, объект, через который проектируется мода реставратора. Башмачок – это материальный знак идеального чувства\textbf{ }(платонической любви к Софье Хитрово), что создает амбивалентность\textbf{: }вещь одновременно указывает на прошлое (его материальность, эмпирическое существование) и на идеальное (неосуществимое чувство, невозвратимость).
    Во-вторых, как точка переходности между модами – башмачок маркирует границу, на которой мода реставратора (чистое созерцание невозвратного) потенциально может трансформироваться в направлении деятельной включенности. Паника при потере башмачка («Пропал мой башмачок!») указывает на то, что утрата этого материального медиума грозит разрушением моды и необходимостью перехода в иную модель.
    ПМО-формализация этой амбивалентности:
    Пусть башмачок обозначен как объект \includegraphics[width=1\linewidth]{articles/lugowska/images/image32.png}. Тогда: 
    \begin{center}
    \includegraphics[width=1\linewidth]{articles/lugowska/images/image33.png}
\end{center}
    означает, что мода \includegraphics[width=1\linewidth]{articles/lugowska/images/image30.png}образуется из модуса \includegraphics[width=1\linewidth]{articles/lugowska/images/image16.png}в условии, где модель сама является материальным объектом \includegraphics[width=1\linewidth]{articles/lugowska/images/image32.png}, одновременно являющимся: 
    \begin{itemize}
      \item Репрезентантом прошлого \includegraphics[width=1\linewidth]{articles/lugowska/images/image28.png}
      \item Точкой амбивалентного перехода (граница между чистым созерцанием и деятельностью)
    \end{itemize}

    \begin{center}
    \includegraphics[width=1\linewidth]{articles/lugowska/images/image34.png}
\end{center}
    То есть утрата материального медиума (башмачка) приводит к блокировке проектора, невозможности образования моды в этой модели.
    Рассмотрим следующую модель: Настоящее \includegraphics[width=1\linewidth]{articles/lugowska/images/image35.png}и мода консерватора 
    Условием этой модели в структуре образа В. С. Соловьева у Д. С. Мережковского является рассмотрение \includegraphics[width=1\linewidth]{articles/lugowska/images/image16.png}в условиях современности, необходимости сохранения существующего. 
    Текстовые маркеры:
    «Не разрушать и не созидать, а сохранять и поддерживать, подпирать валящееся здание, чинить и замазывать трещины – таков его глубочайший инстинкт» [Там же: 131].
    «Остановить, запрудить всемирный поток разрушения – такова его заветная цель» [Там же: 132].
    ПМО-формализация:
    \begin{center}
    \includegraphics[width=1\linewidth]{articles/lugowska/images/image36.png}
\end{center}
    Обозначим характеристики этой моды:
    \begin{itemize}
      \item Содержание: консервирующая деятельность, поддержание существующего
      \item Поле образования: отношение к историческому процессу
      \item Риторический маркер: консервативность, ретрограднось
    \end{itemize}

    В локиге нашего анализа следующей к рассмотрению выступает модель: Будущее как эсхатон \includegraphics[width=1\linewidth]{articles/lugowska/images/image37.png}и мода эсхатолога 
    Ее условием выступает рассмотрение \includegraphics[width=1\linewidth]{articles/lugowska/images/image16.png}в условиях апокалиптического будущего, конца истории. 
    Текстовые маркеры:
    «Страх будущего – "антихристов страх". "Через двести-триста лет, какая будет жизнь на земле!" – воркуют чеховские герои. – Через двести-триста лет монголы завоюют Европу, начнется всемирная резня, придет антихрист, и наступит конец мира, – каркает Вл. Соловьев» [Там же: 132].
    «Последняя и единственная революция для него – переворот уже не исторический, а космический – кончина мира» [Там же: 132].
    ПМО-формализация:
    \begin{center}
    \includegraphics[width=1\linewidth]{articles/lugowska/images/image38.png}
\end{center}
    Важно отметить, что по Д. С. Мережковскому будущее для В. С. Соловьева – не пространство созидания\textbf{, }но\textbf{ }катастрофа, апокалипсис\textbf{.} Модель \includegraphics[width=1\linewidth]{articles/lugowska/images/image37.png}принципиально отлична от революционной, прагматической модели (см. раздел 4.3). Эсхатологическое будущее – это конец истории, а не ее продолжение и преобразование. 
    То есть характеристиками этой моды будут следующие:
    \begin{itemize}
      \item Содержание: апокалиптический пророк, предсказатель антихриста
      \item Поле образования: отношение к будущему времени (как апокалипсису, а не историческому прогрессу)
    \end{itemize}

    Рассмотрим означенную триаду мод как тройственное раскрытие модуса и предложим синтезирующую формулу:
    \begin{center}
    \includegraphics[width=1\linewidth]{articles/lugowska/images/image39.png}
\end{center}
    Как видим, темпоральная структура образа В. С. Соловьева у Д. С. Мережковского состоит в\textbf{ }трехмерном варьировании одного и того же модуса\textbf{ }на условиях трех темпоральных моделей. Каждая мода полнокровна, каждая содержит истину о Соловьеве, но \textit{ни одна не является полной.}

    \textbf{Несобственная модель: революционный прагматизм как структурная невозможность}

    В дальнейшей логике построения модели образа В. С. Соловьева в литературно-критическом тексте Д. С. Мережковского мы приходим к необходимости определения невозможной модели, ключевой тезой которой выступает утверждение, что \textit{модель }	extit{революционного прагматизма}	extit{\textbf{ }}
    \includegraphics[width=1\linewidth]{articles/lugowska/images/image40.png}\textit{не входит в множество собственных моделей}\textit{ модуса Соловьева:}
    \begin{center}
    \includegraphics[width=1\linewidth]{articles/lugowska/images/image41.png}
\end{center}
    Это означает, что не существует проектора, который мог бы произвести моду деятеля-революционера из модуса гностика-созерцателя в условии революционного прагматизма:
    \begin{center}
    \includegraphics[width=1\linewidth]{articles/lugowska/images/image42.png}
\end{center}
    Другими словами, это не выбор и не оценка\textbf{, }но\textbf{ }	extit{онтологическая невозможность}\textbf{.} Модус \includegraphics[width=1\linewidth]{articles/lugowska/images/image16.png}структурно несовместим с условием \includegraphics[width=1\linewidth]{articles/lugowska/images/image40.png}. 
    Текстовые маркеры:
    Мережковский выражает эту невозможность через указание на радикальное бессилие:
    «Стихия революционная чужда ему навеки и безнадежно, если не как человеку-деятелю, то как философу-созерцателю» [Мережковский, 1908: 132].
    «Не только революция, но и реформация, не могли бы вспыхнуть от соловьевского гнозиса, как самый плохенький пожар от самого великолепного, вечернего зарева» [Там же: 133].
    «Реальное действие соловьевской критики на Церковь поразительно ничтожно: критика эта для православия, как жало пчелы для гиппопотамовой кожи: православие, можно сказать, и не почесалось» [Там же: 134].
    «Л. Толстого все-таки отлучили от Церкви. Вл. Соловьева не отлучали и не благословляли, а просто не заметили» [Там же: 134].
    Мережковский визуализирует эту невозможность через серию метафор бессилия. Рассмотрим их подробнее. 
    Метафора\textit{  Зарево vs. пожар}
    \begin{center}
    \includegraphics[width=1\linewidth]{articles/lugowska/images/image43.png}
\end{center}
    \begin{adjustbox}{center}\begin{tabular}{ |l|l|l| }
      \hline
      \textbf{Зарево} & \textbf{Пожар} &  \\
      \hline
      Эстетическая красота & Практическая эффективность \\
      \hline
      «Великолепное» & «Плохенький» \\
      \hline
      Закат, вечер & Восходящее будущее \\
      \hline
      Энергия потенциальная & Энергия кинетическая \\
      \hline
    \end{tabular}\end{adjustbox}\


    Оппозиция указывает на качественное несоответствие\textbf{ }между интенсивностью гностического созерцания и требуемой интенсивностью исторического действия.
    Метафора\textit{ Жало пчелы vs. кожа гиппопотама}
    Критическая мощь В. С. Соловьева (жало) несоразмерна твердости институционального сознания (кожа гиппопотама). Глагол «не почесалось»\textbf{ }маркирует полное отсутствие эффекта, что говорит не о слабости жала, но о его неприложимости к этому материалу.
    Метафора \textit{Невидимость vs. отлучение}
    «не заметили» (Соловьева) vs. «отлучили» (Толстого)
    Отлучение Л. Н. Толстого – это признание его воздействия (даже если негативное). Невидимость В. С. Соловьева – это неспособность институции даже противостоять ему, потому что она его не видит.
    В анализируемой статье концепт двойственности локализуется не как симметричное сосуществование созерцателя и деятеля, но как асимметричный конфликт между двумя модами, образованными в различных моделях эпистемологического типа:
    Мода 1 (аутентичная, невыразимая):
    \begin{center}
    \includegraphics[width=1\linewidth]{articles/lugowska/images/image44.png}
\end{center}
    Мода 2 (выраженная, неаутентичная):
    \begin{center}
    \includegraphics[width=1\linewidth]{articles/lugowska/images/image45.png}
\end{center}
    где \includegraphics[width=1\linewidth]{articles/lugowska/images/image46.png}– модель скрытого бытия (недоступного внешнему наблюдению), \includegraphics[width=1\linewidth]{articles/lugowska/images/image47.png}– модель публичного дискурса (дискурсивно выраженного). 
    \textit{Мода аутентичного: скрытый пророк}
    Текстовые маркеры:
    «Тут уже другое, не явное, а тайное лицо его; не прошлое, а будущее, не реставрация, а революция. Об этой революции говорит уже не философ десятью томами, а немыми знаками немой пророк» [Там же: 135].
    ПМО-характеристики:
    \begin{itemize}
      \item Онтологически истинна (содержит подлинную профетическую истину)
      \item Эпистемологически недоступна (невыразима в публичном дискурсе)
      \item Семиотически немая (не артикулируется в философских трудах)
    \end{itemize}
    Замечание о связи с АСП\textsuperscript{\footnote{О связи с ассоциативно-семантическими полями (концепт скрытости): В лингвориторическом анализе концепт скрытости реализуется через лексемы и образы: скрывается, стальная решетка, подполье, невидимая сторона, невыраженное. Эти лексемы в ПМО интерпретируются как маркеры условия (модели) Z_"скрыт" , в котором образуется мода скрытого пророка. Они указывают не на сам образ (моду), но на условие его образования.}}: 
    В лингвориторическом анализе концепт скрытости реализуется через следующие лексемы и образы:
    \begin{itemize}
      \item скрывается за произведениями
      \item стальная решетка на лице
      \item подполье и чертовщина
      \item невидимая сторона медали
      \item то, что не показывают публично
    \end{itemize}
    Эти лексемы маркируют условие невыраженности, условие \includegraphics[width=1\linewidth]{articles/lugowska/images/image46.png}, в котором образуется мода скрытого пророка. 
    \textit{Мода публичного: философ «в десяти томах»}
    Текстовые маркеры:
    «В произведениях все стройно, ясно, гладко, даже слишком гладко, выглажено, вылощено» [Там же: 129, 118].
    Серия синонимичных адъективов с семой \textit{'гладкость'} и \textit{'поверхностность'}\textbf{ }(\textit{стройно} (упорядоченность), \textit{ясно} (отчетливость), \textit{гладко} (отсутствие шероховатости), \textit{выглажено} (механическая обработка), \textit{вылощено} (искусственная полировка) и семантическое накопление создает эффект избыточности. В синонимическом ряду возникает эффект катахрезы – столкновение положительной оценки (ясность, стройность) и негативной коннотации (искусственность, маска) в рамках одной семы \textit{'гладкости'}. 
    ПМО - характеристики:
    \begin{itemize}
      \item Дискурсивно выражена («десять томов философии»)
      \item Онтологически ложна (маска, публичный образ)
      \item Семиотически избыточна («слишком гладко, вылощено»)
    \end{itemize}

    На основании этих построений можно предложить основную формулу традегии «немого пророка» В. С. Соловьева по Д. С. Мережковскому:

    \begin{center}
    \includegraphics[width=1\linewidth]{articles/lugowska/images/image48.png}
\end{center}
    Профетическая истина Соловьева \textit{невыразима} в модели публичного дискурса \includegraphics[width=1\linewidth]{articles/lugowska/images/image47.png}, а выраженное в публичном дискурсе (философия в десяти томах) не содержит профетической истины.
    Текстовые маркеры:
    «Если Вл. Соловьев, действительно, – предтеча Новой Церкви, то не тем, что он говорил и жил, как мудрец, а тем, что молчал и "умер, как безумец"» [Там же: 135].
    Можно формализовать антитезу, фиксирующую апорию:

    \begin{adjustbox}{center}\begin{tabular}{ |l|l|l| }
      \hline
      \textbf{Говорил} & \textbf{Молчал} &  \\
      \hline
      Мудрец (рациональность) & Безумец (иррациональность) \\
      \hline
      Философия (десять томов) & Пророчество (немота) \\
      \hline
      Выраженное & Невыраженное \\
      \hline
      Неистинное (маска) & Истинное (аутентичная мода) \\
      \hline
    \end{tabular}\end{adjustbox}\


    \textbf{Трагедия «немоты» как онтологическая апория}

    Сейчас самое время обратить внимание на семантико-ассоциативную модель названия статьи о В. С. Соловьеве: «Немой пророк»\textbf{ }– это\textbf{ }оксюморонное сочетание, содержащее структурный код для представления образа В.С. Соловьева Д. С. Мережковским:
    \begin{itemize}
      \item Пророк (\includegraphics[width=1\linewidth]{articles/lugowska/images/image49.png}) = обладатель профетического знания, модус 
      \item Немой (\includegraphics[width=1\linewidth]{articles/lugowska/images/image50.png}) = неспособность к артикуляции, к выражению 
    \end{itemize}
    ПМО-формализация апории:
    \begin{center}
    \includegraphics[width=1\linewidth]{articles/lugowska/images/image51.png}
\end{center}
    То есть В. С. Соловьев одновременно:
    \begin{enumerate}
      \item Является пророком (обладает истиной)
      \item Немой (не способен выразить эту истину)
      \item Если бы выразил пророчество, оно бы перестало быть пророчеством (превратилось бы в философский дискурс, потеряло бы аутентичность)
    \end{enumerate}

    В этой связи рассмотрим оппозицию говорения и молчания в структуре образа, которая представлена формулой расщепления

    \begin{center}
    \includegraphics[width=1\linewidth]{articles/lugowska/images/image52.png}
\end{center}
    Это означает, что когда В. С. Соловьев говорит (\textit{философия в десяти томах}), он выражает нечто, но это нечто не является истиной (маска, неаутентичность); а когда В. С. Соловьев молчит (пророческое молчание), он хранит истину, но эта истина остается невыраженной

    \begin{center}
    \includegraphics[width=1\linewidth]{articles/lugowska/images/image53.png}
\end{center}
    По мере приближения моды к абсолютной истинности ее выразимость стремится к нулю. Предел – абсолютная немота при абсолютной истинности.
    В ПМО-терминах:
    Трагедия = 
        $\exists$X_истинное : X_истинное $\notin$ Z_публ     (истинное невыразимо)
        $\exists$X_выраженное : ¬(истинное)           (выраженное неистинно)
        X_истинное $\cap$ X_выраженное = $\varnothing$         (расщепление)

    \textbf{Омут как эсхатологический закон неизбежного синтеза}
    Не менее интересной и значимой для понимания структуры образа В. С. Соловьева в его представлении Д. С. Мережковским является общее название сборника критических статей, одна из которых посвящена образу В. С. Соловьева. Сборниу 1908 года назван Д. С. Мережковским «В тихом омуте». 
    Ассоциативно-семантический и контекстуальный анализ функционирования концепта «омут» в этом сборнике показывает необходимость коррекции его интерпретации как эсхатологического закона неизбежного синтеза: омут как (B) неизбежный синтез, \textit{недостижимый в историческом времени. } 
    Текстовые маркеры:
    Мережковский говорит о слиянии в \textit{футуральной модальности неизбежности}, а не актуального осуществления: «Рано или поздно эти два противоположные течения встретятся и сольются в одном бездонном омуте» [Мережковский, «Мистические хулиганы»].
    «Хотя в последнем пределе религиозное созерцание и религиозное действие сливаются в одно, но до этого слияния предстоит им исчерпать все мыслимые противоречия» [Мережковский, 1908: 123].
    Глаголы встретятся\textbf{, }сливаются\textbf{, }предстоит маркируют будущность и неизбежность синтеза, но не его наличность в настоящем.
    Такое понимание позволяет нам формализовать предельный переход:
    Пусть для момента времени \includegraphics[width=1\linewidth]{articles/lugowska/images/image54.png}определены две моды модуса \includegraphics[width=1\linewidth]{articles/lugowska/images/image16.png}: 
    \begin{center}
    \includegraphics[width=1\linewidth]{articles/lugowska/images/image55.png}
\end{center}
    где \includegraphics[width=1\linewidth]{articles/lugowska/images/image56.png}– спецификатор, зависящий от времени. 
    Текущее состояние (конечное время): В любой момент конечного времени \includegraphics[width=1\linewidth]{articles/lugowska/images/image57.png}две моды \textit{не пересекаются}:
    \begin{center}
    \includegraphics[width=1\linewidth]{articles/lugowska/images/image58.png}
\end{center}
    Они существуют параллельно, но не сливаются.
    Предельный переход (эсхатон): Введем оператор слияния \includegraphics[width=1\linewidth]{articles/lugowska/images/image59.png}(от греч. ὅμοιος – один, единство):
    \begin{center}
    \includegraphics[width=1\linewidth]{articles/lugowska/images/image60.png}
\end{center}
    Таким образом, по мере приближения к концу истории (к эсхатону) две противоположные моды асимптотически приближаются друг к другу и в пределе достигают абсолютного единства \includegraphics[width=1\linewidth]{articles/lugowska/images/image61.png}– полного синтеза созерцания и деяния. 
    Однако этот синтез остается недостижимым в исторической реальности. В. С. Соловьев умирает \textit{до слияния}:
    \begin{center}
    \includegraphics[width=1\linewidth]{articles/lugowska/images/image62.png}
\end{center}
    То есть В. С. Соловьев, по Д. С. Мережковскому, существует в промежутке \includegraphics[width=1\linewidth]{articles/lugowska/images/image63.png}, где синтез еще не произошел. Он исчерпывает противоречия, но не достигает их разрешения и как раз розовый башмачок в свете эсхатологии – это материальный знак недостижимости\textbf{.} Он фиксирует привязанность к уже прошедшему времени в момент, когда требуется прыжок в будущее. Паника при его потере отражает онтологическое замешательство: утрата того малого материального знака, который связывал модус с миром конечного времени.
     \textbf{Заключение}
    Таким образом, можно представить полную ПМО-структура образа В. С. Соловьева как «немого пророка» по Д. С. Мережковскому:

    \begin{adjustbox}{center}\begin{tabular}{ |l|l|l|l| }
      \hline
      \textbf{Компонент} & \textbf{Формализация} & \textbf{Интерпретация} &  \\
      \hline
      \textbf{Модус} & \begin{center}
    \includegraphics[width=1\linewidth]{articles/lugowska/images/image16.png}
\end{center} & Соловьев-гностик \\
      \hline
      \textbf{Сущностные предикаты модуса} & \begin{center}
    \includegraphics[width=1\linewidth]{articles/lugowska/images/image64.png}
\end{center} & Гнозис, рационализм, консервативность, реставраторство \\
      \hline
      \textbf{Собственные модели} & \begin{center}
    \includegraphics[width=1\linewidth]{articles/lugowska/images/image65.png}
\end{center} & Три темпоральные условия образования мод \\
      \hline
      \textbf{Несобственная модель} & \begin{center}
    \includegraphics[width=1\linewidth]{articles/lugowska/images/image66.png}
\end{center} & Революционный прагматизм структурно невозможен \\
      \hline
      \textbf{Аутентичная мода (скрытая)} & \begin{center}
    \includegraphics[width=1\linewidth]{articles/lugowska/images/image67.png}
\end{center} & Невыраженный пророк \\
      \hline
      \textbf{Выраженная мода (неаутентичная)} & \begin{center}
    \includegraphics[width=1\linewidth]{articles/lugowska/images/image68.png}
\end{center} & Публичный философ \\
      \hline
      \textbf{Апория} & \begin{center}
    \includegraphics[width=1\linewidth]{articles/lugowska/images/image69.png}
\end{center} & Расщепление истины и выражаемости \\
      \hline
      \textbf{Материальный медиум} & Розовый башмачок \includegraphics[width=1\linewidth]{articles/lugowska/images/image32.png} & Амбивалентная модель прошлого \\
      \hline
      \textbf{Эсхатологический горизонт} & \begin{center}
    \includegraphics[width=1\linewidth]{articles/lugowska/images/image70.png}
\end{center} & Неизбежный, но недостижимый синтез \\
      \hline
    \end{tabular}\end{adjustbox}\

    Тогда \textit{диаграмма модальных отношений} будет выглядеть следующим образом:
    Модус Y (гностик-созерцатель)
                                                        │
                    ┌─────────────┼─────────────┐
                    │                                     │                                      │
                Z_прошлое            Z_настоящее            Z_эсхатон
             (прошлое)          (настоящее)       (апокалипсис)
                │                                        │                                      │
                ▼                                      ▼                                      ▼
          X_реставратор         X_консерватор        X_эсхатолог
       (былое надежно)     (валится здание)   (конец мира)
             [Материальный медиум: розовый башмачок b₀]
                            └──────┬──────┘
                                              │
                        [Расщепление по эпист-оси]
                                              │
                    ┌───────────────┴───────────────┐
                    │                                                     │
                Z_скрыт (невыраж)              Z_публ (выраж)
                    │                                                   │
                    ▼                                                 ▼
             X_тайное (пророк)             X_явное (философ)
             Истинное, немое              Выраженное, ложное
             ¬Moda(X_явное, Y, Z_публ)∩Moda(X_тайное, Y, Z_скрыт)
    {\centering АПОРИЯ: расщепление истинности и выразимости \par}

    Настоящее исследование опирается на две методологически различные, но дополняющие друг друга аналитические процедуры. Первый этап (лингвориторический анализ, Луговская, Грудина, 2025) выявил концептуальные структуры образа В. С. Соловьева посредством анализа ассоциативно-семантических полей, риторических приемов и языковых механизмов генерации смысла в критическом тексте Д. С. Мережковского. Второй этап (настоящее исследование) предложил формализацию этих структур через аппарат ПМО, стремясь перейти от описательного анализа к онтологической архитектонике.
    Критически важный вопрос: остаются ли установки лингвориторического анализа в силе при переходе к ПМО-формализации? Или ПМО-аппарат привносит искажения, переинтерпретирует выявленные факты, преломляет их сквозь призму формального синтаксиса, теряя адекватность первоначальным наблюдениям?
    Верификация против лингвориторического анализа служит ответом на этот вопрос. Она позволяет убедиться в том, что формализация не есть искажение, что ПМО-синтаксис сохраняет истину о тексте Мережковского и о образе В. С. Соловьева, выявленную лингвориторическим методом.

    \textbf{Таблица 1. Согласование ключевых установок}

    \begin{adjustbox}{center}\begin{tabular}{ |l|l|l|l| }
      \hline
      \textbf{Установка Л-анализа} & \textbf{Подтверждение в ПМО-анализе} & \textbf{Статус} &  \\
      \hline
      \textbf{Л1:} Бинарная оппозиция «философ-созерцатель» / «человек-деятель» & Базовый модус \includegraphics[width=1\linewidth]{articles/lugowska/images/image16.png}(гностик) и несобственная модель \includegraphics[width=1\linewidth]{articles/lugowska/images/image40.png}(деятель)  & \textbf{Консистентно} \\
      \hline
      \textbf{Л2 (уточн.):} Башмачок – материальный знак идеального чувства + привязка к прошлому & Башмачок как материализованная модель \includegraphics[width=1\linewidth]{articles/lugowska/images/image28.png}и амбивалентная точка переходности  & \textbf{Консистентно и уточнено} \\
      \hline
      \textbf{Л3:} Омут – паремиологический образ слияния противоположностей & Омут как эсхатологический предельный переход \includegraphics[width=1\linewidth]{articles/lugowska/images/image71.png} & \textbf{Консистентно} \\
      \hline
      \textbf{Л4:} Расщепление на \includegraphics[width=1\linewidth]{articles/lugowska/images/image68.png}и \includegraphics[width=1\linewidth]{articles/lugowska/images/image67.png} & Два типа мод в разных моделях \includegraphics[width=1\linewidth]{articles/lugowska/images/image47.png}и \includegraphics[width=1\linewidth]{articles/lugowska/images/image46.png} & \textbf{Консистентно} \\
      \hline
    \end{tabular}\end{adjustbox}\

    Лингвориторический анализ выявил, что Д. С. Мережковский конструирует образ В. С. Соловьева через операционализацию бинарной оппозиции «философ-созерцатель» / «человек-деятель». Эта оппозиция не является поверхностной риторической фигурой, но выступает как концептуальная ось, вокруг которой организуется вся система ассоциативно-семантических полей. Концепт «философ-созерцатель» связан с лексемами и образами скрытости, идеальности, мистической погруженности; концепт «человек-деятель» – с видимостью, общественной активностью, реальным воздействием. Оба концепта присутствуют одновременно, но находятся в состоянии непримиримого напряжения.
    ПМО-анализ трансформирует эту бинарную оппозицию в онтологическую структуру. Первый член оппозиции («философ-созерцатель») соответствует базовому модусу, который Д. С. Мережковский эксплицитно определяет как гностика, то есть как существо, структурно ориентированное на созерцание, разум (гнозис), познание идеального. Второй член оппозиции («человек-деятель») соответствует попытке образования моды на модели (революционный прагматизм), однако эта попытка терпит неудачу: модель оказывается несобственной для модуса, то есть структурно несовместимой с ним. Бинарная оппозиция в ПМО-формализации становится асимметричной. Это не два равноправных аспекта одного существа, но основание (модус) и структурно невозможное расширение (несобственная модель). Лингвориторический анализ воспринял эту оппозицию как содержательное противоречие; ПМО-анализ показывает, что это не логическое противоречие, а онтологическая апория.
    Другими словами, ПМО не отменяет установку Л1, но углубляет ее, преобразуя феноменальное противоречие в онтологическую структуру.
    По пониманию розового башмачка как амбивалентного медиума лингвориторический анализ выявил, что этот артефакт функционирует в тексте двойственно: он одновременно является (а) материальным знаком идеального чувства – платонической любви к Софье Хитрово, которая овеществляется в этом предмете и становится доступна для созерцания, (б) материальной привязкой к прошлому – башмачок как вещь, с которой В. С. Соловьев «ни на миг не расставался», фиксирует его сознание на невозвратном времени. Парадоксально, что именно материальность башмачка (его конкретность, вещественность) служит медиумом для выражения идеального (любви, чувства), но одновременно эта материальность препятствует переходу к действию, она удерживает В.С. Соловьева в сфере чистого созерцания.
    Такая трактовка подтверждается ПМО-анализом, который  интерпретирует башмачок как материализованную модель (условие образования моды) – она сама является конкретным материальным объектом, не абстрактным условием, но вещью, которая проявляется в физической реальности. Башмачок одновременно функционирует в двух качествах: как модель в ПМО-смысле – он выступает условием, на котором образуется мода реставратора (Соловьев раскрывается как реставратор, как романтик былого, именно в том, что держит этот башмачок, созерцает его, возвращается мыслью в прошлое через этот материальный объект); и как точка амбивалентного перехода – башмачок маркирует границу, потенциальное место трансформации (паника при его потере («Пропал мой башмачок!») указывает на то, что башмачок служит не только символом, но онтологическим якорем образа. Его утрата грозит разрушением самой возможности образования моды реставратора в модели прошлого). Одновременно башмачок, будучи материальным, хранит в себе потенциал движения, деятельности, но этот потенциал остается нереализованным.
    Формализация ПМО показывает, что амбивалентность башмачка не случайна и не поэтическое украшение, но отражает глубокую онтологическую структуру: материальное выступает медиумом идеального, но только в условии его статичности, неподвижности. Как только материальный медиум исчезает (утрачивается башмачок), образ распадается.
    ПМО-анализ не только подтверждает наблюдение Л-анализа о двойственности башмачка, но и дает точную онтологическую формулировку этой двойственности через понятие материализованной модели и амбивалентной переходности.
    Для понимания омута как структурного закона слияния лингвориторический анализ выявил, что Д. С. Мережковский использует образ омута (редуцированную форму поговорки «в тихом омуте черти водятся») как паремиологический код, вынесенный в название сборника. Омут символизирует несоответствие между внешней спокойной поверхностью (тихий омут) и скрытой глубиной, таящей опасность и тайну. В контексте анализа образа В. С. Соловьева омут символизирует слияние противоположностей – созерцательности и деятельности, гностицизма и прагматизма – которые в настоящий момент находятся в конфликте, но в некотором будущем состоянии должны слиться в единство.
    ПМО-анализ переинтерпретирует \textit{омут} не как образное описание, а как эсхатологический закон неизбежного синтеза. Д. С. Мережковский говорит: «Хотя в последнем пределе религиозное созерцание и религиозное действие сливаются в одно, но до этого слияния предстоит им исчерпать все мыслимые противоречия». Глаголы в футуральной модальности («будут встречаться и сливаться», «сливаются», «предстоит») указывают на то, что синтез не актуален в настоящем, но неизбежен в пределе.
    ПМО-формализация представляет это как предельный переход. Смысл этой формализации в том, что синтез противоположностей является структурным законом бытия, но этот закон реализуется только в пределе, при приближении к бесконечности (к апокалипсису, концу истории). В конечном времени (в котором живет и умирает В. С. Соловьев) синтез остается недостижимым, хотя два противоположных потока (гнозис и действие) асимптотически приближаются друг к другу.
    Омут, таким образом, – это не просто красивый образ, но визуализация эсхатологического горизонта, точки, в которой все противоречия разрешаются, но которая остается трансцендентна историческому времени. ПМО-анализ сохраняет центральное интуитивное содержание Л-анализа (омут как символ слияния) и углубляет его математическую и онтологическую формулировку.
    Для реализации смысла расщепления явного и тайного лингвориторический анализ выявил своеобразное раздвоение образа В. С. Соловьева на две несовместимые ипостаси (публичный философ, десять томов философии, выглаженные и вылощенные произведения и невыраженный пророк, скрытое подполье, молчаливое пророчество). Это расщепление маркируется лингвистически: оксюморонами, катахрезами, апоретическими вопросами, которые создают семантическое напряжение и не позволяют читателю воспринять образ как одномерный, плоскостной.
    ПМО-анализ формализует это расщепление как образование двух различных мод на двух различных моделях эпистемологического типа. 
    Критически важно, что это не две стороны одной медали, не две грани одного образа, но две полностью несовместимые моды одного модуса, образованные на несовместимых моделях. Одна модель образует моду явного философа, которая онтологически ложна (маска, неаутентичность), но дискурсивно выражена, другая (скрытое бытие, невидимая сторона) образует моду тайного пророка, которая онтологически истинна, но эпистемологически недоступна, семиотически немая.
    Трагедия состоит в следующем: нет проектора, который мог бы одновременно образовать обе моды из одного модуса в одной модели. Они существуют в разных онтологических условиях. В. С. Соловьев как тайный пророк остается невидимым, неслышимым; В. С. Соловьев как явный философ представляется обществу в искусственной форме, в маске.
    ПМО-анализ предоставляет точную онтологическую формулировку того, что лингвориторический анализ описал как семантическое напряжение между двумя несовместимыми образами.

    Помимо подтверждения основных установок лингвориторического анализа, ПМО-подход добавляет ряд качественно новых инсайтов, которые не могли быть выявлены чисто лингвистическими методами (Таблица 2).
    Во-первых, это невозможность деятельной моды как онтологическая, а не логическая, апория. ПМО-анализ показал, что формализация через множество собственных моделей позволила выявить, что революционный прагматизм не входит в это множество. Это не означает, что В. С. Соловьев попросту выбрал не быть деятелем (как может, например, монах выбрать монашество вместо светской деятельности). Это означает, что структура модуса несовместима со структурой модели. Гностик-созерцатель не может образовать моду деятеля-революционера не потому, что он не хочет, а потому, что это онтологически невозможно (Проектор попросту не существует).
    Это уточнение качественно меняет интерпретацию трагедии В. С. Соловьева в ее понимании Д. С. Мережковским. Критик не говорит: «Соловьев мог бы быть деятелем, но выбрал созерцание», он говорит: «Революционная стихия чужда ему навеки и безнадежно». Слова «навеки» и «безнадежно» указывают на абсолютность, на невозможность в принципе, а не на выбор. ПМО-анализ формализует эту абсолютность через понятие несобственной модели.
    Во-вторых, лингвориторический анализ выявил, что башмачок как символ двойственнен. ПМО-анализ объясняет эту двойственность: башмачок – это не просто символ или метафора, но материализованная модель, то есть модель, которая сама является материальным объектом. Это создает специфическую амбивалентность: модель не просто условие (абстрактное), но конкретная вещь, которая может быть потеряна или утрачена.
    Это объясняет, почему паника при потере башмачка в тексте Д. С. Мережковского не является психологической случайностью. Башмачок удерживает образ в модели прошлого, а его утрата означает блокировку самой возможности образования моды реставратора. Одновременно с этим, башмачок, будучи материальным, хранит в себе потенциал движения, деятельности, но этот потенциал реализуется только как мучительное созерцание невозвратного, как фиксация на прошлом.
    В-третьих, Д. С. Мережковский раскрывает образ В. С. Соловьева через три собственные модели: прошлое (реставраторство), настоящее (консервация), апокалиптическое будущее (эсхатологизм). Каждая мода полнокровна, содержит истину об образе, но ни одна не исчерпывает модуса. ПМО-анализ показывает, что триада не является произвольным выбором Д. С. Мережковского, но отражает полноту раскрытия модуса во времени. Гностик-созерцатель не может раскрыться полностью в одном временном измерении, потому что его сущность требует трехмерной темпоральной архитектоники: отношение к прошлому (что было, архаика), отношение к настоящему (что есть, требующее консервации), отношение к будущему (что будет, апокалипсис). Каждое измерение дает полный, истинный аспект модуса, но вместе они не образуют синтеза, но остаются в напряжении.
    В-четвертых, лингвориторический анализ выявил, что В. С. Соловьев парадоксально описывается Д. С. Мережковским как «немой пророк», то есть субъект, который одновременно является пророком (обладает истиной) и оказывается немым (не может ее выразить). ПМО-анализ формализует эту апорию математически, где выразимость стремится к нулю по мере приближения к абсолютной истинности. Это не просто образное выражение, но структурное следствие расщепления между модой тайного пророка (онтологически истинной, эпистемологически недоступной) и модой явного философа (выраженной, но неистинной). Если нет проектора, который мог бы произвести моду, одновременно истинную и выраженную, то истина остается в немоте; выражение становится ложью.
    И наконец, в-пятых, Д. С. Мережковский говорит об омуте, о слиянии противоположностей в некотором конечном состоянии (в апокалипсисе, в конце истории). ПМО-анализ показывает, что этот синтез не является проекцией, надеждой или утопией, но структурно необходимым пределом. Это означает, что противоречия внутри образа В. С. Соловьева (между гностицизмом и прагматизмом, между пророчеством и философией) не являются случайными или результатом его индивидуальной недостаточности. Они отражают фундаментальное противоречие эпохи, которое может быть разрешено только в апокалиптическом синтезе. Немой пророк по Д. С. Мережковскому – это не просто неудавшийся синтез, но предвестник того синтеза, который должен произойти в конце истории.
    В настоящем исследовании мы использовали проектор и сюръектор как ключевые компоненты формализации, при этом оставили и проектор, и сюръектор на уровне интуитивного понимания, когда формализация показывает онтологическую архитектонику, но не предоставляет механического алгоритма для проверки результатов.
    Нами введено понятие множества собственных моделей на основании того, что В. И. Моисев говорит о том, что модус способен образовывать моды на различных моделях, но он не организует модели в явное множество и не вводит вопрос о том, какие модели являются собственными, а какие – несобственными. Введение множества собственных моделей является метатеоретическим расширением ПМО, а не прямой экстраполяцией. Однако это расширение оказалось критически важным для всего анализа – без невозможно формализовать идею о том, что революционный прагматизм онтологически невозможен для В. С. Соловьева как гностика.
    После этого весьма интересного опыта применения ПМО для формализации структуры образа объекта литературно-критического текста остались открытыми некоторые теоретические вопросы. Так, например, для концепта слияния мы ввели оператор слияния Ω\Omega, который должен выражать эсхатологическое объединение противоположных мод и критически важен для нашего анализа. Однако в синтаксисе ПМО этот оператор не определен –он может быть рассмотрен как модусная сумма или операция на модах или на модусах – в любом случае без явного определения Ω\Omega наша формализация эсхатологического синтеза остается на уровне метафоры, облаченной в математические символы – это не формальный результат, а интуитивное выражение интуиции Д. С. Мережковского, и мы отдаем себе в этом отчет. 
    Мы говорим, что башмачок является материализованной моделью и одновременно амбивалентной точкой переходности. Но как это точно выразить в синтаксисе ПМО?
    Другой важный вопрос касается выделенных нами трех собственных моделей: прошлого, настоящего, апокалиптического будущего, но ведь возможны и другие модели, на которых также образуются моды модуса образа В. С. Соловьева. 
    Например, можно рассмотреть этическую модель (Соловьев как моралист, как искатель добра), эстетическую (Соловьев как ценитель красоты, как философ красоты) и социальную модель (Соловьев в отношении к обществу, к церкви, к государству) – ведь Д. С. Мережковский касается этих измерений, хотя они и не развернуты в полную моду. Но если существуют и другие собственные модели, то образ раскрывается более полно, чем через одну темпоральную триаду. 
    Сборник «В тихом омуте» содержит множество (в математическом смысле) статей Д. С. Мережковского о русских писателях и мыслителях и применение ПМО-анализа к этим текстам позволило бы проверить, является ли метод Д. С. Мережковского универсальным способом экзистенциального портретирования или он специфичен только для образа В. С. Соловьева. Лингвориторический анализ сборника показывает, что экзистенциальное портретирование как метод универсален для Д.С.Мережковского-критика. В этом случае подтверждается гипотеза о том, что ПМО действительно отражает глубокую логику критического мышления Д. С. Мережковского. 
    \textbf{Заключение}
    Применение аппарата Проективно-модальной онтологии В. И. Моисеева к анализу образа В. С. Соловьева в критической статье Д. С. Мережковского «Немой пророк» позволило достичь качественного прогресса в понимании как структуры образа, так и метода критического портретирования в целом.
    На уровне формализации образа нам удалось преобразовать интуитивные наблюдения лингвориторического анализа в точные онтологические структуры. Бинарная оппозиция «созерцатель / деятель» предстала не просто как содержательное противоречие, но как асимметричное отношение между модусом и несобственной моделью. Розовый башмачок проявился не просто как символ, но как материализованная модель, несущая в себе амбивалентность между идеальным и материальным. Темпоральная структура образа (прошлое – настоящее – будущее) раскрылась как полнота раскрытия модуса во времени, где каждое измерение дает истинный, но неполный аспект. Трагедия В. С. Соловьева перестала быть просто психологическим или идеологическим конфликтом – она проявилась как три переплетающихся онтологических апории:
    Несовпадение модуса и эпохи (гностик-созерцатель существует в эпоху, требующую модуса-прагматика. Это не конфликт между личностью и обществом в психологическом смысле, но структурное рассогласование между экзистенциальным модусом и историческим требованием).
    Блокировка проектора на несобственной модели (революционный прагматизм онтологически невозможен для гностика не как выбор, а как структурная невозможность. Нет проектора, который мог бы произвести моду деятеля-революционера из модуса гностика).
    Расщепление аутентичной и выраженной мод (пророческая истина остается невыраженной, выраженное становится ложью. Нет моды, которая была бы одновременно истинной и выраженной, аутентичной и дискурсивной).
    Таким образом, экзистенциальное портретирование предстало не как субъективное впечатление или психологический анализ, но как систематическая онтологическая процедура. Д. С. Мережковский варьирует образ на множестве моделей (темпоральных, онтологических, эпистемологических), выявляя собственные и несобственные модели, локализуя точки блокировки проектора, фиксируя онтологические апории. Метод Д. С. Мережковского предвосхищает позднейшие герменевтические и экзистенциальные подходы Хайдеггера, Гадамера и других мыслителей XX века, для которых личность и ее смыслы раскрываются именно через анализ ее способов быть, ее модусов существования в различных условиях.
    Наше исследование приводит к одному фундаментальному выводу: язык и формальная структура – это не украшение смысла, но основные генераторы смысла и онтологической истины.
    Когда Д. С. Мережковский пишет о розовом башмачке, это не риторическое излишество, не поэтическое украшение – конкретность этого образа, его материальность, его история – все это несет онтологическую нагрузку. Башмачок является материализованной моделью прошлого. Когда критик пишет: «стройно, ясно, гладко, даже слишком гладко, выглажено, вылощено» – синонимический ряд создает не просто красивый эффект, но фиксирует катахрезу как столкновение положительной оценки и негативной коннотации, маркирующую искусственность маски.
    Язык Д. С. Мережковского устроен так, что семантическое напряжение между оксюморонами, апоретическими вопросами, катахрезами преобразуется в онтологическое напряжение между несовместимыми модами. Риторический уровень и онтологический уровень не просто параллельны – они изоморфны друг другу, форма языка отражает форму бытия образа.
    Метод экзистенциального портретирования, в его онтологической формализации через ПМО, открывает путь к новому типу критического мышления, который не ограничивается литературным анализом, но может быть применен к анализу культурных явлений, исторических фигур, идеологических движений в целом.  Экзистенциальное портретирование как критический метод – это не субъективное впечатление рассказчика, но систематическое варьирование личности на множестве онтологических моделей, выявление собственных и несобственных моделей, локализация онтологических апорий как точек структурной невозможности.
    Д. С. Мережковский в своей критике В. С. Соловьева осуществляет эту процедуру с исключительной полнотой. Он показывает личность не как статичный портрет, но как динамическую систему мод, которые реализуются в различных условиях, которые содержат в себе непримиримые противоречия, которые остаются неразрешимыми в рамках исторического времени.
    В.С. Соловьев в представлении Д. С. Мережковского – это не просто великий учитель, не просто философ, не просто революционер, но личность в ее полной экзистенциальной амбивалентности, личность, которая может быть понята, только если мы согласимся видеть в ней одновременно истину и маску, пророчество и молчание, созерцание и немощь деяния.
    И именно эта амбивалентность, эта невозможность однозначного прочтения – это и есть суть критического метода Д.С. Мережковского – не в том, чтобы дать окончательное представление образа, но в том, чтобы показать, что описываемая личность остается открытой для множественного прочтения, когда смысл раскрывается только в движении сквозь противоречия в вариации образа на различных условиях существования.
    Это – метод не психологический и не исторический, но онтологический, и в этом смысле Д. С. Мережковский опережает современное ему литературоведение, предчувствуя те философские подходы к анализу личности и смысла, которые будут развиты только в XX веке, его метод предвосхищает позднейшие герменевтические и экзистенциальные подходы к анализу личности в литературе и становится прообразом критических практик, где язык и формальная структура рассматриваются не как украшение, но как основные генераторы смысла и онтологической истины.

    \textbf{Таблица 2. Инсайты, добавленные ПМО-анализом}

    \begin{adjustbox}{center}\begin{tabular}{ |l|l|l|l| }
      \hline
      {\centering \textbf{Инсайт} \par} & {\centering \textbf{Источник в ПМО} \par} & {\centering \textbf{Эвристическое значение} \par} &  \\
      \hline
      \textbf{1. Невозможность деятельной моды} & \begin{center}
    \includegraphics[width=1\linewidth]{articles/lugowska/images/image66.png}
\end{center} & Показывает, что противоречие не логическое, но \textbf{онтологическое}: структура модуса несовместима со структурой модели \\
      \hline
      \textbf{2. Амбивалентность башмачка} & Башмачок как \includegraphics[width=1\linewidth]{articles/lugowska/images/image28.png}и как точка переходности  & Дает точную формулировку того, как \textbf{материальное} служит медиумом идеального и одновременно препятствует переходу к действию \\
      \hline
      \textbf{3. Темпоральная триада} & Три собственные модели & Систематизирует интуицию о том, что Соловьев раскрывается полностью только через три временных измерения, ни одно из которых не достаточно \\
      \hline
      \textbf{4. Немота как апория истины} & \begin{center}
    \includegraphics[width=1\linewidth]{articles/lugowska/images/image72.png}
\end{center} & Формулирует парадокс: истина пророка именно в ее невыразимости; выражение истины ее уничтожает \\
      \hline
      \textbf{5. Эсхатологический синтез} & \begin{center}
    \includegraphics[width=1\linewidth]{articles/lugowska/images/image73.png}
\end{center} & Показывает, что синтез не актуален в истории, но структурно необходим как предел; это не проекция, но объективный закон бытия \\
      \hline
    \end{tabular}\end{adjustbox}\

    {\centering \textbf{Литература} \par}
    Блок, А.А. (1906). Рыцарь-монах. \textit{О Владимире Соловьеве. Сборник первый.}\\\textbf{ }М.: Путь. С. 103–115.
    Луговская, Е.Г., Грудина, Е.К. (2024). Экзистенциальное портретирование как критический метод: лингвориторический анализ статьи Д.С. Мережковского «Немой пророк». \textit{[подготовлена к публикации]}
    Мережковский, Д.С. (1908/1991). Немой пророк. \textit{В тихом омуте: статьи и исследования разных лет.}\\\textbf{ }М.: Советский писатель. С. 128–135.
    Моисеев, В.И. (2002). Логика всеединства. М.: ПЕР СЭ.
    Моисеев, В.И. (2002). К аксиоматике Модальной Онтологии. \textit{Рационализм и культура на пороге 3-го тысячелетия: материалы 3-го Российского Философского конгресса.}\\\textbf{ }Ростов н/Д: СКНЦ ВШ. С. 283–284.
    Моисеев, В.И. (2002). Projectively Modal Ontology. \textit{Logical Studies}\\, 9. \underline{\href{http://www.logic.ru/LogStud/09/LS9.html}{http://www.logic.ru/LogStud/09/LS9.html}}
    Розанов, В.В. (1900). Еще о Вл. Соловьеве. \textit{Мир искусства}, 3–4, 87–93.
    Трубецкой, Е.Н. (1913). \textit{Миросозерцание Вл. С. Соловьева.} Т. 1–2. М.: Путь.

    \paragraph{Ключевые слова:} {\itshape Проективно-модальная онтология (ПМО), В.С. Соловьев, Д.С. Мережковский, экзистенциальное портретирование, модус-мода-модель, формализация критического метода}
    \end{russian}
\fi }
  }

% --- User-facing commands ---
% Wrapper for adding an article
\NewDocumentCommand{\addarticle}{m}
  {
    \journal_add_article:n {#1}
  }

% Sets a flag and prints all articles (assuming content switches on the flag)
\NewDocumentCommand{\printallabstracts}{}
  {
    \printabstracttrue
    \journal_print_articles:
  }

% Clears a flag and prints all articles (assuming content switches on the flag)
\NewDocumentCommand{\printallarticles}{}
  {
    \printabstractfalse
    \journal_print_articles:
  }

\ExplSyntaxOff % End expl3 syntax mode
