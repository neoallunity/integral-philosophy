% ============================================
% cfg-semantic.tex — расширенные семантические конструкции
% ============================================

\RequirePackage{xparse}
\RequirePackage{xcolor}

% ============================================
% СЕМАНТИЧЕСКИЕ АКЦЕНТЫ В ТЕКСТЕ
% ============================================

% Термин (например, для глоссария)
\NewDocumentCommand{\Term}{m}{%
  \textbf{\textcolor{JournalAccent!85}{#1}}%
}

% Важное понятие
\NewDocumentCommand{\Concept}{m}{%
  \textit{\textcolor{JournalAccent2!75}{#1}}%
}

% Цитируемый источник (inline)
\NewDocumentCommand{\Source}{m}{%
  \textcolor{JournalAccent!70}{\textit{#1}}%
}

% Акцент (замена для \emph с фирменным цветом)
\NewDocumentCommand{\Accent}{m}{%
  \textcolor{JournalAccent2!80}{\textit{#1}}%
}

% ============================================
% СЕМАНТИЧЕСКИЕ БЛОКИ
% ============================================

% Замечание редакции
\NewDocumentEnvironment{editorialremark}{}{%
  \vspace{0.5cm}%
  \noindent\colorbox{JournalAccent!8}{%
    \begin{minipage}{\dimexpr\textwidth-2\fboxsep-2\fboxrule}
      \small\textcolor{JournalAccent!80}{\textbf{Примечание редакции:}}\\[0.2cm]
}{%
    \end{minipage}%
  }%
  \vspace{0.5cm}%
}

% Врезка с дополнительной информацией
\NewDocumentEnvironment{sidebar}{o}{%
  \vspace{0.5cm}%
  \noindent\fcolorbox{JournalAccent2!50}{JournalAccent2!5}{%
    \begin{minipage}{\dimexpr\textwidth-2\fboxsep-2\fboxrule}
      \IfNoValueTF{#1}{}{%
        \small\textbf{\textcolor{JournalAccent2!85}{#1}}\\[0.2cm]%
      }%
}{%
    \end{minipage}%
  }%
  \vspace{0.5cm}%
}

% Цитата (с фирменным оформлением)
\RenewDocumentEnvironment{quote}{}{%
  \vspace{0.4cm}%
  \begin{list}{}{%
    \setlength{\leftmargin}{1.5cm}%
    \setlength{\rightmargin}{0.5cm}%
  }%
  \item\relax\small\itshape\textcolor{JournalAccent!75}%
}{%
  \end{list}%
  \vspace{0.4cm}%
}

% ============================================
% СЕМАНТИЧЕСКИЕ РАЗДЕЛИТЕЛИ
% ============================================

% Тематический разрыв (замена \rule)
\NewDocumentCommand{\ThematicBreak}{}{%
  \vspace{0.8cm}%
  \noindent\textcolor{JournalAccent2!60}{\rule{0.3\textwidth}{0.5pt}}%
  \vspace{0.8cm}%
}

% Разделитель секций (более яркий)
\NewDocumentCommand{\SectionBreak}{}{%
  \vspace{1cm}%
  \begin{center}%
    \textcolor{JournalAccent!70}{\Large\textbullet\quad\textbullet\quad\textbullet}%
  \end{center}%
  \vspace{1cm}%
}

% ============================================
% НОМЕРА И ССЫЛКИ
% ============================================

% Номер выпуска (с фирменным стилем)
\NewDocumentCommand{\IssueNumber}{}{%
  \textcolor{JournalAccent!85}{\textbf{№\,\journalissue}}%
}

% Год выпуска
\NewDocumentCommand{\IssueYear}{}{%
  \textcolor{JournalAccent!85}{\textbf{\journalyear}}%
}

% Полное обозначение выпуска
\NewDocumentCommand{\IssueLabel}{}{%
  \textcolor{JournalAccent!85}{%
    \textbf{№\,\journalissue\;•\;\journalyear}%
  }%
}

% ============================================
% АФФИЛИАЦИИ И АДРЕСА
% ============================================

% Аффилиация автора
\NewDocumentCommand{\Affiliation}{m m}{%
  \textsuperscript{#1}\,\textit{#2}%
}

% Email автора
\NewDocumentCommand{\AuthorEmail}{m}{%
  \textcolor{JournalAccent!70}{\texttt{#1}}%
}

% ORCID (с иконкой если есть)
\NewDocumentCommand{\AuthorORCID}{m}{%
  \textcolor{JournalAccent!70}{\textsc{orcid:}\,\texttt{#1}}%
}

% ============================================
% ДАТЫ И ВЕРСИИ
% ============================================

% Дата поступления статьи
\NewDocumentCommand{\SubmissionDate}{m}{%
  \small\textit{Поступила:}\;\textcolor{JournalAccent!70}{#1}%
}

% Дата принятия статьи
\NewDocumentCommand{\AcceptanceDate}{m}{%
  \small\textit{Принята:}\;\textcolor{JournalAccent!70}{#1}%
}

% Дата публикации
\NewDocumentCommand{\PublicationDate}{m}{%
  \small\textit{Опубликована:}\;\textcolor{JournalAccent!70}{#1}%
}

% Блок дат (объединяет три даты)
\NewDocumentCommand{\ArticleDates}{m m m}{%
  \vspace{0.5cm}%
  \noindent\small%
  \SubmissionDate{#1}\quad|\quad%
  \AcceptanceDate{#2}\quad|\quad%
  \PublicationDate{#3}%
  \vspace{0.5cm}%
}

% ============================================
% DOI И ЛИЦЕНЗИИ
% ============================================

% DOI статьи
\NewDocumentCommand{\ArticleDOI}{m}{%
  \noindent\small\textsc{doi:}\;\url{https://doi.org/#1}%
}

% Лицензия Creative Commons
\NewDocumentCommand{\CCLicense}{m}{%
  \noindent\small\textcolor{JournalAccent!70}{%
    \textcopyright\;Распространяется по лицензии CC\,#1%
  }%
}

% ============================================
% ПЕРЕКРЕСТНЫЕ ССЫЛКИ (СЕМАНТИЧЕСКИЕ)
% ============================================

% Ссылка на другую статью в выпуске
\NewDocumentCommand{\RefArticle}{m}{%
  \textcolor{JournalAccent2!75}{\textit{см. статью}}\;\cref{#1}%
}

% Ссылка на предыдущий выпуск
\NewDocumentCommand{\RefPreviousIssue}{m}{%
  \textcolor{JournalAccent!70}{\textit{см. выпуск}}\;№\,#1%
}

% ============================================
% АННОТАЦИИ (ДОПОЛНИТЕЛЬНЫЕ)
% ============================================

% Краткая версия для TOC (Table of Contents Abstract)
\NewDocumentCommand{\TOCAbstract}{m}{%
  \addtocontents{toc}{\protect\begin{quote}\protect\small\textit{#1}\protect\end{quote}}%
}
