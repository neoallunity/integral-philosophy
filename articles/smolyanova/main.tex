\ifprintabstract
    \begin{english}
    % Smolyanova N.A. - English abstract
    \subsection{INTEGRAL PHYSICS}
    \label{subsec:smolyanova-en

    \subsubsection{Smolyanova N.A. INTEGRAL PHYSICS AS AN APPROACH TO HOLISTIC UNDERSTANDING OF THE UNIVERSE: OVERCOMING THE EPISTEMOLOGICAL CRISIS IN CONTEMPORARY THEORETICAL KNOWLEDGE}
    \label{subsubsec:smolyanova-title-en

    Contemporary theoretical physics confronts a profound systemic crisis characterised by an increasingly divergent fragmentation between fundamental theoretical frameworks and an absence of coherent cosmological understanding. This unity crisis manifests across multiple interconnected dimensions, each presenting significant methodological and conceptual challenges to modern natural science. The crisis is exemplified most acutely in the fundamental incompatibility between two paramount theoretical constructs of contemporary physics: quantum mechanics, which governs phenomena at microscopic scales, and general relativity, which elucidates the nature of spacetime and gravitational interaction. Despite demonstrating exceptional empirical precision within their respective domains of application, these theories prove categorically incommensurable when unified integration is attempted, thereby indicating profound deficiencies in our foundational conceptions of physical reality.

    This predicament is compounded by the proliferation of alternative interpretations of quantum mechanics, each proposing distinct ontological characterisations of the world. From the Copenhagen interpretation through many-worlds formalism and hidden variable theories, all such approaches demonstrate equivalent empirical adequacy whilst simultaneously proposing fundamentally divergent explanations of quantum phenomena. Concurrently, theoretical physics encompasses multiple alternative research programmes—ranging from string theory to loop quantum gravity—that function as isolated investigative enterprises without constructive theoretical dialogue. These theoretical pluralities signal not merely provisional disagreement but structural fragmentation at the foundations of physical knowledge.

    An additional critical factor intensifying the crisis is deepening mathematical specialisation, whereby disparate domains of physics employ categorically distinct mathematical formalisms without established protocols for transition between them. This generates formidable impediments to knowledge synthesis and development of unified theoretical foundations. Consequently, contemporary physics constitutes a complex mosaic of discontinuous conceptual frameworks, each addressing particular aspects of reality whilst remaining incapable of articulating a holistic cosmological vision.

    In response to these challenges, the present article proposes the concept of integral physics—a fundamentally novel interdisciplinary methodology directed toward overcoming extant fragmentation in physical knowledge. This approach is grounded in the philosophical methodology of integral science, developed within the framework of integral philosophy, which furnishes the conceptual and epistemological foundations for integral physics as a research paradigm. Contrary to traditional research programmes that attempt reduction of all phenomena to a single fundamental theory, integral physics advances a systemic approach to organising physical knowledge predicated upon recognition of multi-scalar hierarchical organisation of physical reality, wherein each level maintains its operational autonomy whilst remaining fundamentally interconnected through deeper organisational principles.

    The cardinal element of the integral approach consists in developing a universal metatheoretical language capable of describing interconnections between disparate physical theories and organisational levels of matter. Such a language must synthesise mathematical rigour with conceptual flexibility, permitting expression of qualitative differentiation across distinct scales of physical phenomena whilst preserving their fundamental coherence. Integral physics accords significant emphasis to development of novel mathematical formalisms capable of transcending the dichotomy between discrete and continuous, local and global descriptions of reality.

    Integral physics construes physical reality as a dynamic totality wherein the observer constitutes not an external agent but an intrinsic systemic element. This approach necessitates transcendence of traditional philosophical dichotomies—such as subject-object dualism or determinism-probability opposition—and development of innovative conceptual frameworks adequate to the complexity and coherence of the physical cosmos. Crucially, the recognition of the observer as an internal system element carries profound methodological consequences: it requires fundamental reconceptualisation of measurement operations, the ontological status of physical properties, and the relationship between theoretical models and empirical reality. This shifts the epistemological foundation from objective observation of an independent external world toward understanding how theoretical constructions emerge from the recursive interaction between investigator, instrument, and phenomenon. Rather than pursuing further theoretical specialisation, this methodology seeks restoration of integrative understanding grounded in recognition of fundamental organisational principles.

    The practical implications of integral physics extend to possibilities for unprecedented technological applications across multiple domains. For instance, reconceptualising the fundamental relationship between quantum and relativistic effects—rather than treating them as incompatible theoretical regimes—opens novel pathways for quantum computation architectures capable of operational coherence at previously inaccessible energy scales, and for engineering advanced materials with properties governed by multi-scalar physical principles rather than single-regime approximations. Similarly, understanding deep organisational connections between microscopic and macroscopic phenomena enables development of energy technologies whose efficiency no longer depends upon incremental refinement within isolated theoretical frameworks but upon alignment with fundamental principles of material organisation. Such applications remain contingent upon prior achievement of theoretical integration and the resulting capacity to translate insights across distinct physical domains.

    Realisation of the integral physics programme demands cultivation of a new generation of researchers possessing not merely profound specialist knowledge but capacity for interdisciplinary reasoning and synthetic vision of scientific problems. This necessitates substantial transformation of physics pedagogy, directed toward transcendence of narrow specialisation and cultivation of comprehensive scientific worldview. The epistemological shift required extends beyond methodological modification to encompass fundamental reconceptualisation of the relationship between knower and known, theory and observation, mathematical formalism and physical intuition.

    The article addresses an extensive constituency of specialists—theoretical physicists, philosophers of science, mathematicians engaged in theoretical physics, and all researchers confronting problems of unity within natural-scientific knowledge. Development of integral physics as a novel research paradigm potentially inaugurates a scientific revolution of magnitude comparable to the creation of quantum mechanics or relativity theory, yet directed not toward further specialisation but toward restoration of coherence in our understanding of nature. Such a paradigm shift may constitute the necessary intellectual foundation for addressing contemporary crises—including the apparent limits of reductionist methodology and the fragmentation of knowledge characterising late-twentieth- and early-twenty-first-century science.

    \paragraph{Keywords:} \textit{Integral physics; quantum gravity; unity of physical knowledge; theoretical physics crisis; metatheoretical synthesis; multi-scalar models of reality; philosophy of physics; alternative theories in physics; mathematical foundations of physics; interdisciplinary research; quantum mechanics interpretation; spacetime ontology; holistic understanding; scientific paradigm shift; knowledge integration}
    \end{english}
\else
    % Смольянова Н.А. - русская статья
    \subsection{ЛАБОРАТОРИЯ ИНТЕГРАЛЬНОЙ ФИЗИКИ}
    \label{subsec:smolyanova-ru

    \subsubsection{Смольянова Н.А. ИНТЕГРАЛЬНАЯ ФИЗИКА КАК ПУТЬ К ЦЕЛОСТНОМУ ПОНИМАНИЮ МИРОЗДАНИЯ: ПРЕОДОЛЕНИЕ КРИЗИСА СОВРЕМЕННОГО ТЕОРЕТИЧЕСКОГО ЗНАНИЯ}
    \label{subsubsec:smolyanova-title-ru

    \textit{Современная теоретическая физика переживает глубокий системный кризис, характеризующийся нарастающим расхождением между различными фундаментальными теориями и отсутствием целостной картины мироздания. Этот кризис единства проявляется в нескольких взаимосвязанных аспектах, каждый из которых представляет серьезную методологическую и концептуальную проблему для современного естествознания. Центральным проявлением кризиса является фундаментальная несовместимость двух основных теоретических конструкций современной физики – квантовой механики, описывающей явления микромира, и общей теории относительности, объясняющей природу пространства-времени и гравитации. Эти теории, демонстрируя исключительную точность в своих областях применения, оказываются принципиально несовместимыми при попытке их объединения, что указывает на наличие глубинных проблем в наших базовых представлениях о физической реальности.}

    \textit{Ситуация усугубляется существованием множества альтернативных интерпретаций квантовой механики, каждая из которых предлагает свою онтологическую картину мира. От копенгагенской интерпретации до многомировой и теории скрытых параметров – все эти подходы одинаково хорошо согласуются с экспериментальными данными, но при этом предлагают принципиально разные объяснения природы квантовых явлений. Параллельно в теоретической физике развивается множество альтернативных направлений – от теории струн до петлевой квантовой гравитации, – которые существуют как изолированные исследовательские программы без конструктивного взаимодействия между собой.}

    \textit{Дополнительным фактором кризиса служит углубляющаяся математическая специализация, приводящая к тому, что разные области физики используют принципиально различные математические аппараты без четких правил перехода между ними. Это создает серьезные препятствия для синтеза физического знания и разработки единой теоретической платформы. В результате современная физика представляет собой сложную мозаику разрозненных концепций, каждая из которых претендует на описание определенных аспектов реальности, но не способна предложить целостную картину мироздания.}

    \textit{В качестве конструктивного ответа на эти вызовы в статье предлагается концепция интегральной физики – принципиально нового междисциплинарного подхода, направленного на преодоление существующей фрагментации физического знания. В отличие от традиционных исследовательских программ, интегральная физика не пытается свести все явления к какой-либо одной фундаментальной теории, а предлагает системный подход к организации физического знания, основанный на признании многоуровневости физической реальности.}

    \textit{Ключевым элементом интегрального подхода является разработка универсального метатеоретического языка, способного описывать взаимосвязи между различными физическими теориями и уровнями организации материи. Такой язык должен сочетать математическую строгость с концептуальной гибкостью, позволяя выразить качественное своеобразие разных масштабов физических явлений при сохранении их фундаментального единства. Важное место в интегральной физике занимает разработка новых математических формализмов, способных преодолеть разрыв между дискретными и непрерывными, локальными и глобальными описаниями реальности.}

    \textit{Интегральная физика предполагает рассмотрение физической реальности как динамической целостности, где наблюдатель является не внешним, а внутренним элементом системы. Такой подход требует выхода за рамки традиционных дихотомий (например, субъект-объект или детерминизм-вероятность) и разработки новых концептуальных схем, способных адекватно отразить сложность и целостность физического мира.}

    \textit{Практические перспективы интегральной физики связаны с возможностью принципиально новых технологических решений в различных областях – от квантовых вычислений и новых материалов до энергетики и космических технологий. Понимание глубинных связей между явлениями разных масштабов может открыть путь к созданию технологий нового поколения, основанных на согласовании с фундаментальными принципами организации материи.}

    \textit{Реализация программы интегральной физики требует подготовки нового поколения исследователей, обладающих не только глубокими специальными знаниями, но и способных к междисциплинарному мышлению и синтетическому видению научных проблем. Это предполагает существенную трансформацию системы физического образования, направленную на преодоление узкой специализации и формирование целостного научного мировоззрения.}

    \textit{\\textbf{Интегральная физика, квантовая гравитация, единство физического знания, кризис теоретической физики, метатеоретический синтез, многоуровневые модели реальности, философия физики, альтернативные теории в физике, математические основания физики, междисциплинарные исследования}}

    \paragraph*{Современное состояние теоретической физики: достижения и вызовы}
    На сегодняшний день теоретическая физика пребывает в уникальном и парадоксальном состоянии, характеризующемся беспрецедентными успехами в сочетании с глубокими концептуальными проблемами. Как убедительно демонстрирует С. Вейнберг в своем фундаментальном труде «Квантовая теория поля» [Вейнберг, 2003: 15], современная физика достигла невероятной точности в описании широкого спектра явлений – от квантовых процессов на субатомном уровне [Давыдов: 1973] до глобальной структуры и эволюции Вселенной [Вейнберг: 2013]. Эти достижения охватывают временные масштабы от фемтосекундных процессов в ядерной физике до миллиардов лет космологической эволюции, а пространственные – от $10^{-15}$ метров в квантовой хромодинамике до размеров наблюдаемой Вселенной.

    Однако, как отмечают Л. Д. Ландау и Е. М. Лифшиц в своем классическом курсе теоретической физики, «современная физика представляет собой совокупность разрозненных концепций, каждая из которых описывает лишь отдельные фрагменты мироздания» [Ландау, Лифшиц: 1973]. Этот парадокс – невероятная точность отдельных теорий при отсутствии целостной картины – становится все более очевидным по мере углубления наших знаний. Современная физическая наука напоминает величественный, но незавершенный собор, где отдельные капеллы построены с ювелирной точностью, но отсутствует общий архитектурный замысел, связывающий их в единое целое.

    \paragraph*{Концептуальные противоречия и их истоки}
    Наиболее ярким проявлением этого кризиса единства является фундаментальное противоречие между квантовой механикой и общей теорией относительности – двумя столпами современной физической картины мира. Как подчеркивает Вейнберг в «Гравитации и космологии» [Вейнберг: 32], «эти теории остаются принципиально несовместимыми, несмотря на десятилетия интенсивных исследований». Их концептуальные основания – вероятностная природа квантовых явлений и детерминированная геометрия пространства-времени – представляют собой, по сути, две различные метафизики физической реальности.

    Проблема усугубляется существованием множества интерпретаций квантовой механики, каждая из которых предлагает свою онтологическую картину мира. От копенгагенской интерпретации (Бор Н., 1927), подчеркивающей роль наблюдателя, до радикальной многомировой концепции (Девитт Б.С., Грэхем Н., 1973) и теории скрытых параметров Бома (Бом Д., 1959) – все эти подходы одинаково хорошо согласуются с экспериментальными данными, но предлагают принципиально разные объяснения природы физической реальности. Как отмечает Бом (1959), это разнообразие интерпретаций отражает не просто философские расхождения, а фундаментальную неполноту нашего понимания квантовых явлений.

    \paragraph*{Разобщенность современных исследовательских программ}
    Как отмечает А. Н. Васильев в «Квантовой теории поля в двух словах» [Васильев, 2020], параллельно с основными направлениями физики существует множество альтернативных подходов – от теории струн и петлевой квантовой гравитации до различных вариантов эфиродинамики. Эти теории развиваются в значительной степени изолированно, образуя своеобразные «интеллектуальные архипелаги» в океане физического знания. При этом, подчеркивает А. Н. Васильев, отсутствуют четкие методологические критерии для оценки и сравнения этих альтернативных подходов.

    Эта концептуальная разобщенность усугубляется углубляющейся математической специализацией. Разные области физики используют принципиально различные математические аппараты – от операторных алгебр в квантовой теории поля [Васильев, 2017] до дифференциально-геометрических методов в общей теории относительности [Голдстейн и др., 2022]. Как отмечают специалисты, отсутствие «словаря перевода» между этими математическими языками создает дополнительные барьеры на пути к объединению физических теорий.

    \paragraph*{Интегральная физика как ответ на вызовы}
    В ответ на эти системные проблемы возникает \textit{интегральная физика} – не просто как очередная теория в ряду других, а как принципиально новая парадигма организации физического знания. 

    По глубокому определению Д. Бома [Бом, 1959: 112], «интегральная физика представляет собой систематический поиск глубинных принципов, способных объединить разрозненные фрагменты нашего понимания реальности в целостную картину». Такой подход требует преодоления традиционных дисциплинарных границ и создания нового метатеоретического языка, способного описывать отношения между различными физическими теориями без редукции одной к другой.

    Основа интегрального подхода, как подчеркивает С. Вейнберг [Вейнберг, 2013: 45], заключается в признании многоуровневости физической реальности. Каждый уровень организации материи – от квантовых полей до космологических структур – требует своего специфического языка описания, но при этом все уровни связаны через систему глубинных, инвариантных принципов. Это понимание предполагает три взаимосвязанных направления работы:
    \begin{enumerate}
        \item Синтез достижений основных и альтернативных теорий через выявление их общих структурных элементов и принципов;
        \item Разработку новых математических методов [Васильев, 2017], способных адекватно описать переходы между разными уровнями физической реальности;
        \item Глубокий философский анализ оснований физического знания, включая переосмысление таких базовых категорий, как пространство, время и причинность.
    \end{enumerate}

    \paragraph*{Перспективы и практические приложения}
    Ключевая задача интегральной физики, по выражению Бома (1959), – преодоление искусственного разрыва между микро- и макромиром. Особое значение в этом контексте приобретает разработка новых математических структур [Васильев, 2020], способных объединить дискретные и непрерывные, локальные и глобальные описания реальности. Это направление работы уже сегодня дает плодотворные результаты в области квантовой гравитации и космологии.

    Философские основания интегрального подхода требуют радикального переосмысления традиционных категорий. Как отмечает С. Вейнберг [Вейнберг, 2000], физическая реальность должна рассматриваться как динамическая целостность, где наблюдатель является не внешним, а внутренним и активным элементом системы. Такой подход открывает новые перспективы для понимания проблемы измерения в квантовой механике и природы физических законов.

    Практические приложения интегральной физики [Голдстейн и др., 2022] простираются от квантовых вычислений и новых материалов до принципиально новых энергетических технологий. Реализация этого потенциала требует подготовки нового поколения исследователей, способных к синтетическому мышлению и преодолению узкоспециализированных подходов – именно такие «универсальные» ученые чаще всего становятся авторами прорывных открытий на стыке различных областей знания.

    \paragraph*{Заключение}
    Как убедительно демонстрируют работы ведущих исследователей [Бом, 1959; Вейнберг, 2000-2013; Васильев, 2017; Васильев, 2020], интегральная физика представляет собой не просто новую научную парадигму, а фундаментально иной способ осмысления физической реальности. Этот подход, зародившийся как ответ на глубокий кризис единства в современной теоретической физике, постепенно формируется в целостную исследовательскую программу, способную преодолеть существующие концептуальные разрывы и методологические тупики.

    Суть интегрального подхода заключается в принципиальном отказе от редукционистской парадигмы, доминировавшей в физике последние столетия, и переходе к холистическому пониманию природы физических явлений. В отличие от традиционных направлений, стремящихся свести все многообразие физических процессов к неким элементарным составляющим, интегральная физика рассматривает реальность как сложную, многоуровневую систему, где каждый уровень организации материи обладает своей качественной спецификой, но при этом связан с другими уровнями через систему глубинных принципов.

    Особую значимость интегральный подход приобретает в контексте решения ключевых проблем современной физики – от проблемы квантовой гравитации до загадки темной материи и темной энергии. Традиционные попытки решения этих проблем через экстраполяцию известных теорий на новые области оказываются недостаточными. Интегральная физика предлагает принципиально иной путь – не механическое объединение существующих теорий, а создание новой концептуальной основы, способной органично включить в себя достижения различных направлений как частные случаи.

    Важнейшей характеристикой интегрального подхода является его принципиальная междисциплинарность. В отличие от узкоспециализированных направлений современной физики, интегральная физика активно заимствует методы и идеи из смежных областей – математики, философии, когнитивных наук, что позволяет выработать более гибкий и многомерный взгляд на физическую реальность – именно такой синтетический подход может привести к прорыву в понимании природы сознания и его роли в физических процессах.

    Перспективы развития интегральной физики связаны с несколькими ключевыми направлениями. Во-первых, это разработка нового математического аппарата, способного адекватно описывать многоуровневые физические системы [Васильев, 2017; Васильев, 2020]. Во-вторых, создание концептуального языка, позволяющего выразить качественное своеобразие разных масштабов физической реальности. В-третьих, формирование новой методологии физического исследования, сочетающей точность математического описания с глубиной философского осмысления.

    Особое значение интегральная физика приобретает в контексте современных технологических вызовов. Как показывают исследования [Вейнберг, 2013], дальнейший прогресс в таких областях, как квантовые вычисления, нанотехнологии или исследования космоса, требует принципиально нового уровня понимания физических процессов. Интегральный подход, с его акцентом на целостность и взаимосвязь явлений, может стать теоретической основой для технологий следующего поколения.

    В образовательном аспекте развитие интегральной физики требует коренной перестройки системы подготовки физиков. Необходимо преодолеть сложившуюся узкую специализацию и воспитать новое поколение исследователей, способных к синтетическому мышлению и междисциплинарному диалогу – это предполагает не только изменение учебных программ, но и трансформацию самого стиля научного мышления.

    В исторической перспективе интегральная физика может стать основой для нового великого синтеза в физике, сравнимого по масштабу с научными революциями прошлого. Однако в отличие от предыдущих революций, приводивших к углублению специализации, новая синтетическая парадигма будет направлена на восстановление целостности физического знания. Такой синтез может привести не только к решению конкретных физических проблем, но и к формированию принципиально новой картины мира, в которой наука, философия и технология образуют органичное единство.

    Таким образом, интегральная физика представляет собой не просто одно из направлений современной науки, а принципиально новый этап в развитии физического знания. Ее становление и развитие может привести к преодолению кризиса современной физики и открыть новые горизонты в познании природы и создании технологий будущего. Как убедительно показывают все цитируемые авторы, именно такой целостный, синтетический подход может стать основой для следующего великого синтеза в физике, способного объединить разрозненные фрагменты нашего понимания реальности в целостную картину мироздания.

    \begin{center}
        \textbf{Литература}
    \end{center}

    \begin{enumerate}
        \item Бом, Д. Причинность и случайность в современной физике / Пер. с англ. – М.: ИЛ, 1959. – 248 с.
        \item Васильев, А. Н. Квантовая теория поле в двух словах. – СПб.: Лань, 2020. – 160 с.
        \item Васильев, В. А. Введение в топологию. – М.: МЦНМО, 2017. – 224 с.
        \item Вейнберг, С. Гравитация и космология. Принципы и приложения общей теории относительности / Пер. с англ. – М.: Едиториал УРСС, 2000. – 696 с.
        \item Вейнберг, С. Квантовая теория поля. Т. 1: Общая теория / Пер. с англ. – М.: Физматлит, 2003. – 648 с.
        \item Вайнберг, Стивен. Космология: Пер. с англ. / Под ред. и с предисл. И. Я. Арефьевой, В. И. Санюка. М.: УРСС: Книжный дом «ЛИБРОКОМ», 2013. – 608 с.
        \item Голдстейн, Г., Пул, Ч., Сафко, Дж. Классическая механика / Пер. с англ. – 3-е изд. – СПб.: Лань, 2022. – 720 с.
        \item Давыдов, А. С. Квантовая механика. – 2-е изд. – М.: Наука, 1973. – 704 с.
        \item Dewitt and Neill Graham. Princeton Series in Physics. Princeton University Press. Princeton, New Jersey, 1973.
        \item Ландау, Л.Д., Лифшиц, Е.М. Теоретическая физика. Том 1: Механика. – М.: Наука, 1973.
    \end{enumerate}
\fi