\ifprintabstract
    \begin{english}
    % Safonov A.A. - English abstract
    \subsection{INTEGRAL MATHEMATICS}
    \label{subsec:safonov-en

    \subsubsection{Safonov A.A. INNER EXPERIENCE AS AN APPROACH TO PHENOMENOLOGICAL MODELLING OF WORLD-LIKE SYSTEMS}
    \label{subsubsec:safonov-title-en

    The work represents research into the problem of modeling the inner world within the context of neo-all-inclusive philosophy. Instead of relying on traditional deduction based on mathematical structures, this paper offers another perspective grounded in direct perception of one's inner world via personal experiences—such as dreams, reflections, memories, and acts of active imagination. The investigation builds upon Johann Wolfgang von Goethe's scientific methodology, according to which every careful observation of nature inherently carries elements of theoretical reasoning. The main hypothesis asserts that the inner world manifests itself directly in everyday life and psychological experience, including dreams, visual imagery, and intuitive insights. The primary goal lies in identifying structural components of the inner world that could potentially be described mathematically, thus bridging this approach with V.I. Moiseev's theory of world-like systems using the mathematical apparatus of R-functions to describe similarity among various systems in the universe. Such an approach broadens conventional scientific horizons and allows us to perceive human inner life as a system incorporating holistic perceptions of the world.

    \paragraph{Keywords:} \textit{Inner world, modeling, neo-ontogony, selfhood, unconscious, integration, methodology, ideation, integrity, psychology.}
    \end{english}
\else
    % Сафонов А.А. - русская статья
    \subsection{ЛАБОРАТОРИЯ ИНТЕГРАЛЬНОЙ МАТЕМАТИКИ}
    \label{subsec:safonov-ru

    \subsubsection{Сафонов А.А. ВНУТРЕННИЙ ОПЫТ КАК ПУТЬ К ФЕНОМЕНОЛОГИЧЕСКОМУ МОДЕЛИРОВАНИЮ МИРОПОДОБНЫХ СИСТЕМ}
    \label{subsubsec:safonov-title-ru

    \textit{Настоящая работа представляет собой исследование проблемы моделирования внутреннего мира в контексте философии неовсеединства. Вместо традиционной дедукции, основывающейся на математических структурах, предлагается иной подход, опирающийся на прямое восприятие нашего внутреннего мира через собственные переживания – сновидения, рефлексии, воспоминания и активные акты воображения. Исследование основано на естественнонаучной методологии Иоганна Вольфганга фон Гете, согласно которой каждое пристальное наблюдение природы уже несет элемент теоретизации. Основное предположение состоит в том, что внутренний мир проявляется непосредственно в нашей повседневной жизни и психологическом опыте, таком как сны, зрительные образы и интуитивные прозрения. Основная задача заключается в выявлении структурных элементов внутреннего мира, которые могли бы быть описаны математически, тем самым сближая этот подход с теорией мироподобных систем В.И. Моисеева, использующих математический аппарат R-функций для описания подобия отдельных систем Вселенной. Такой подход расширяет границы традиционных научных подходов и позволяет увидеть внутреннюю жизнь человека как систему, включающую в себя элементы целостного восприятия мира.}

    \textbf{Внутренний мир, моделирование, неовсеединство, самость, бессознательное, интеграция, методология, идеация, целостность, психология}

    \paragraph*{Введение}
    В данной статье я бы хотел подойти к проблеме моделирования «внутренних миров», которая ставится в философии неовсеединства, с несколько неожиданной стороны – не дедуктивно, отталкиваясь от математических структур, а опираясь на тот материал, который дает нам непосредственное соприкосновение с нашим внутренним миром – в сновидениях, размышлениях, воспоминаниях, в опыте активного воображения. 

    Данный подход во многом опирается на естественнонаучную методологию Гете, которую можно выразить одной его фразой: «При всяком внимательном взгляде на природу мы уже теоретизируем» [Гете, 1957: 253]. Для этого образа мысли характерна особенная гносеологическая доверчивость к области исследований: искомые структуры, законы, смыслы уже присутствуют в интересующей нас области, общее уже присутствует в частном, необходимо лишь то особое внимание, которое позволило бы «видеть идеи».

    За основу наших исследований мы примем следующую гипотезу: внутренний мир, о котором говорит нам философия неовсеединства\footnote{Подробнее о философии неовсеединства см.:
    \begin{itemize}
        \item Моисеев В.И. Логика всеединства / Воронеж. гос. мед. академия. Воронеж, 1999.
        \item Моисеев В.И., Русская философия всеединства как прообраз интегральной науки, //Соловьевские исследования, 2024, вып. 3 (83), с. 38–48.
        \item Моисеев В.И. От логики всеединства к философии неовсеединства // Материалы междунар. науч. конф. «Философия В.С. Соловьева в межкультурной коммуникации: к 110-летию со дня смерти В.С. Соловьева и 20-летию праведной кончины протоиерея Александра Меня», Иваново, 23–25 сентября 2010 г. Иваново, 2010. С. 86–90.
    \end{itemize}}, дается нам непосредственно – в наших снах, видениях, воспоминаниях, в том, что дано нам на экране «первого лица» и потому находится за скобками традиционных естественнонаучных методов, не имеющих доступа к подобному опыту.

    В то же время нужно понимать, что, как и в случае с внешним опытом, внутренний опыт никогда не открывает нам реальность внутреннего мира целиком, мы всегда имеем дело с некоторыми фрагментами, «кадрами» на внутреннем экране. К таким фрагментам можно отнести локации, ситуации, увиденные нами во сне, навязчивые образы, к которым снова и снова возвращается воображение и т.д.

    Наша задача – попробовать сложить эти фрагменты в цельную мозаику, увидеть за отдельными субъективными данными объективную закономерность.

    Данную попытку нельзя назвать чем-то абсолютно новым – подобным путем идет глубинная психология К. Юнга, С. Грофа, К. Уилбера. Новым здесь будет то, что целью нашего исследования внутреннего мира будет выявление структурных, поддающихся математическому описанию закономерностей, и в этом отношении мы движемся навстречу теории мироподобных систем [Моисеев, 2022], которая выводит подобные структуры дедуктивным путем.

    Непостижимую эффективность математики в исследовании внешнего мира можно объяснить тем, что с ее помощью структурное эйдетическое ядро явления становится имманентным сознанию. Так, записав уравнение маятника, мы можем изучать движение маятника изнутри, не обращаясь к физическому маятнику. Конечно, математика упрощает, оставляет только предельно общее, но это упрощение дает возможность мысленному моделированию, связывающему воедино бесчисленное количество возможных явлений. Здесь происходит нечто вроде феноменологической идеации – для того, чтобы узнать что-то об определенном типе явлений, не обязательно индуктивно обобщать бесчисленное количество явлений, достаточно работать с их математическим эйдосом.

    История науки богата примерами, когда разработанные независимо от всякого опыта математические абстракции вдруг находили весьма конкретное опытное применение и, наоборот, когда эксперимент стимулировал математическое творчество. Поэтому предлагаемый подход является не альтернативным, но комплементарным по отношению к интегральной методологии неовсеединства. Оба этих подхода движутся в противоположных направлениях. Но если они движутся к истине – они должны встретиться.

    \paragraph*{Лиминарная зона}
    На заре моего детства – едва ли мне тогда исполнилось три года – во время прогулки с бабушкой и дедушкой я вдруг остановился перед таинственной, уходящей в темноту улицей. До определенной точки шла линия фонарей, потом она обрывалась и начиналась кромешная неизвестность. И я все смотрел туда как загипнотизированный.

    С тех пор эта улица то и дело появляется в моих снах – пустынная, тревожная, и в то же время непреодолимо притягивающая, и я все иду по ней в надежде узнать, что же находится в конце. Как-то раз я обнаружил в конце книжный шкаф, однажды увидел там море, но даже во сне было ощущение, что это пределы ненастоящие, это некая ширма, которая скрывает то важное, что я ищу.

    Улица появлялась во снах с такой закономерной регулярностью и сопровождалась таким нуминозным ощущением важности, что постепенно я пришел к тому, что она должна что-то означать в моем внутреннем мире. И только недавно пришла гипотеза, по своей простоте и ясности напоминающая инсайт, мгновенное понимание того как оно есть: улица символизировала движение к краю моего внутреннего мира. Освещенная часть соответствует осознанной его части, зоне эго. Темная – бессознательное, за которым заканчивается собственно мой мир и начинается нечто совершенно иное. География естественным образом способствовала такому символизму: в реальности эта улица упиралась в нейтральную территорию с Польшей, что для моего детского мира было равносильно краю земли.

    \paragraph*{Там стен кривизна…}
    Когда мне было около 20 лет, случился инсайт с другой улицей – мы шли с моим другом, калининградским уличным музыкантом Анатолием Троилиным, песни которого имели свойство «открывать двери» моего восприятия, к его дому, расположенному на старой немецкой улочке и в какой-то момент я представил, как будто она уходит далеко-далеко, в какое-то средневековое чудесное запределье.

    По мотивам того впечатления у меня потом появились следующие строчки:
    \begin{quote}
        \textit{Глаза закрываю}\\
        \textit{Вижу странную улицу}\\
        \textit{Пустынную и уходящую}\\
        \textit{В безумные головные дали}\\
        \\
        \textit{Там стен кривизна}\\
        \textit{И пространство сутулится}\\
        \textit{И кто-то в длинном плаще}\\
        \textit{Несет нам печали}\\
        \\
        \textit{Вдруг падает голова}\\
        \textit{Жуткая Птица}\\
        \textit{С глазами какими-то ненастоящими}\\
        \textit{Выглядывает из-за плеча и…}
    \end{quote}

    Образ уходящей в безумные головные дали улицы долго меня не покидал – было стойкое ощущение, что ей соответствует какая-то реальность. Конечно, не физическая, но как реальность воображения, психики, архетипа не менее, а может быть даже более реальная. Здесь, пожалуй, подойдет термин Р. Отто – нуминозный [Отто, 2008]. Образ внутренней улицы сопровождался ощущением священного трепета, ощущением значительности.

    Примечательно, что пространство улицы представлялось мне искривленным и как бы сжимающимся по мере движения к ее краю. Рядом с ней как-то постоянно «витал» график функции, сжимающий бесконечную область определения по иксу в конечный интервал по \textbf{y} (см. рис. 1).

    \begin{figure}[ht!]
        \centering
        \includegraphics[width=0.5\linewidth]{images/image097.png}
        \caption{График функции, сжимающий бесконечную область в конечный интервал}
        \label{fig:function-graph}
    \end{figure}

    \begin{figure}[ht!]
        \centering
        \includegraphics[width=0.495\linewidth]{images/image083.png}
        \caption{Дополнительная иллюстрация концепции сжатия пространства}
        \label{fig:space-compression}
    \end{figure}

    С удивлением я потом обнаружил эти функции в R–анализе, на котором В.И. Моисеев строит свою теорию мироподобных систем\footnote{Моисеев В.И. Мирология: Наука о мироподобных системах. – М.: ЛЕНАНД, 2022. – 600 с.}, в которой разного рода замкнутые целые (клетка, растение, человек и т.д.) рассматриваются как малые миры, обладающие своим временем, пространством, собственными локальными законами. Каждый из таких миров является финфинитом – сочетает в себе бесконечный и конечный статусы. Математической основой мирологии являются упомянутые выше R–функции, изоморфно сжимающие бесконечные пространства в конечные.

    \paragraph*{Куда ведет внутренняя улица?}
    Настойчивое появление таинственной внутренней улицы во снах, как будто явно было неслучайным, имело некую глубинную логику. В моем внутреннем мире она должна была что-то означать. И особенное значение имел не дававший мне покоя вопрос: что же в конце?

    Когда мне было примерно 20 лет из образа внутренней уходящей в «безумные головные дали» улицы постепенно как из ростка стал вырисовываться замысел, своего рода сверхзадача, которую я тогда себе поставил – написать свою \textit{«книгу книг»,} в которой в единый узел стягивались бы все смыслы мироздания. Несоизмеримость масштаба внутреннего опыта с масштабом задачи может свидетельствовать об инфантильной установке автора, в которой весь мир вращается вокруг него, однако в этом было и нечто другое – гипотеза, или скорее даже интуиция собственного мироподобия, микрокосмичности. Книга была моей попыткой проверить пифагорейский рецепт: познай себя и ты познаешь вселенную.

    Я представлял себе сюжет этой книги как бы уже существующим в мире идей и хотел сделать себя наживкой для него, сделав себя одновременно автором и главным героем книги, а процесс ее написания – ее красной нитью. Речь шла не о классической автобиографии, а о том, чтобы посредством этого сюжета как бы «вытянуть себя за волосы», прыгнуть в запредельное.

    Я взялся за дело и скоро столкнулся с интересным явлением – подобно некому аттрактору, книга как бы писала сама себя, собирая вокруг персонажей, события, символы, философские проблемы. Постепенно она становилась моим «все во всем», философским камнем.

    И символически эта книга находилась для меня как бы в конце заветной улицы.

    Идея подобной «книги книг» является универсальным архетипом целостности. Тут можно вспомнить строчки из последней главы Божественной комедии Данте «\textit{Я видел – в этой глуби сокровенной Любовь как в книгу некую сплела То, что разлистано по всей вселенной}» [Данте, 1982: 683], увиденную Декартом во сне книгу Scientia mirabilis (сумма всех наук), «Бесконечную историю» Михаэля Энде и др.

    Судьба вывернулась так, что тогда нам с женой пришлось снимать комнату, которая находилась в непосредственной близости от загадочной немецкой улочки, где жил Анатолий Гусляр. В то время начал вырисовываться сюжет, который в то же время был и моей жизнью: я снова и снова пытался осознавать себя во сне, выходить в т.н. «астрал», и, когда это получалось, вылетал из окна и отправлялся в сторону заветной мостовой. Я чувствовал, что именно там и должен был завертеться искомый сюжет. Но каждый раз мне что-то мешало. И в конце концов я оставил эти попытки.

    А потом мы уехали в другой город.

    Но спустя два года, совершенно неожиданно искомый сюжет сам нашел меня. Мы были в отпуске, вернулись поздно вечером в Калининград после путешествия и пошли в гости к Анатолию Троилину. Я тогда переживал глубокий внутренний кризис, в основе которого было ощущение своей легковесности по сравнению с мощью смыслов, с которыми я в то время сталкивался.

    И вот, не будучи религиозным, перед сном я стал отчаянно повторять молитву «Господи, Иисусе Христе, Сыне Божий, помилуй меня грешного». В какой-то момент по телу пошли вибрации и я «вышел из тела». И я понял, что нахожусь в сердцевине своей сказки: я был в таинственном мире по ту сторону своего бодрственного опыта и заветная улица была передо мной. И, будучи в реальности короткой, здесь она уходила в какие-то запредельные дали. И я полетел.

    Трудно описать подобные переживания адекватно. Могу сказать, что это было необыкновенно, сказочно и по своей реалистичности превосходило дневной опыт. По ощущениям все это длилось очень долго. Но удалось вынести оттуда только то, что в конце концов я оказался в необыкновенном космическом городе и встретил там миниатюрного сказочного старца с пронзительно синими глазами и белоснежной бородой. На его шее висел медальон. На медальоне появилась надпись «Ask me».

    Я стал было что-то говорить, но надпись стерлась и ее сменила другая: «directly». Кем бы ни была эта сущность, она дала понять, что требует предельной искренности и понимания даже легких намеков, как бы задавая рамки диалога.

    Так началась история, которую я подробно описал в романе «Плерома».

    Но в настоящем рассуждении я хочу посмотреть на данную историю сквозь призму мирологии.

    Абстрагировавшись от деталей, можно выделить следующие предельно общие черты в описанных выше опытах:
    \begin{enumerate}
        \item Таинственная улица стала для меня символом интроверсии, внутреннего измерения.
        \item Улица вела к краю моего мира, по мере продвижения по ней пространство стремительно сжималось, что соответствует моделям R–анализа, в которых бесконечное пространство изоморфно отображается в конечное, что приводит к страшному сжатию пространства по краям.
        \item Конец улицы – предел моего мира и вместе с тем символ центра психики, находящегося вне освященной эго зоны души и называемого в аналитической психологии, а впоследствии и в неовсеединстве – \textit{самостью}.
        \item Примечательно, что в конце улицы я обнаружил несколько характерных символов самости: «книгу книг», космический город, мудрого старца, напоминающий мандалу медальон, предлагающий задавать вопросы.
        \item Пожалуй, будет уместным говорить о естественных полярностях, возникавших в описанных опытах: у улицы есть освященное начало, соответствующее сознательному опыту, эго и т.д., есть темный, недостижимый полюс, который можно соотнести с бессознательным и самостью. Все смысловые движения определяются в первую очередь этой полярностью.
        \item Два базовых полюса задают своего рода две топологии: топологию отделимости, соответствующую разделяющей аналитической функции сознания. Это соответствует полюсу нуля или полюсу Многого Мирологии. И топологию неотделимости, где ни один из смысловых фрагментов не мыслим без тотальности всего. Это реализация принципа «все во всем», структура пространства, задаваемого полюсом Единого, отличающаяся состоянием повышенной когерентности. Архетипическими символами такой всесвязанности являются загадочный город в конце улицы, мудрый старец и «книга книг».
        \item Стремясь быть предельно искренним в своих исследованиях, не могу обойти вниманием один личный момент.
    \end{enumerate}

    В своих исследованиях я попадаю в классическую полярность разума и откровения. Будучи христианином, я верю в возможность знания, открываемого свыше Духом Святым. Это знание мы можем найти в Священном Писании или Символе веры. Очевидно, что само Евангелие как весть о том, что Бог воплотился в конкретном человеке Иисусе Христе, умер за наши грехи и воскрес на третий день, никоим образом невозможно вывести научно. В подобной перспективе в средоточии внутренней жизни всегда остается что-то непрозрачное для рацио.

    В тоже время я полагаю, что никто не мешает разуму идти в своих исследованиях сколь угодно долго, сколь в его силах, независимо от веры.

    И если исходить из базовой для философии всеединства гипотезе о целостности мира, разум и откровения должны сойтись в познании единой истины.

    \paragraph*{Заключение}
    Проведенное исследование демонстрирует, что внутренний опыт – сновидения, интуитивные прозрения, образы воображения – может служить ценным источником для феноменологического моделирования внутренних миров. Подход, основанный на методологии Гете и интегральной философии неовсеединства, позволяет выявить структурные закономерности внутреннего мира, которые в дальнейшем могут быть формализованы с помощью математического аппарата R-анализа и мирологии.

    Главное открытие состоит в том, что внутренний мир обладает собственной геометрией и топологией, которые могут быть описаны с помощью концепций из теории мироподобных систем. Улица как символ внутреннего измерения, ее сжимающаяся геометрия, полярность сознательного и бессознательного – все эти элементы образуют целостную систему, поддающуюся научному исследованию.

    Дальнейшие исследования в этом направлении могут быть связаны с:
    \begin{itemize}
        \item Разработкой более детальных математических моделей внутренних миров на основе R-функций.
        \item Сравнительным анализом символики внутреннего опыта в различных культурных традициях.
        \item Исследованием связи между структурой внутренних миров и нейрофизиологическими процессами.
        \item Разработкой практических методов использования внутреннего опыта для творчества и решения сложных задач.
    \end{itemize}

    Интегральный подход, сочетающий феноменологическое описание с математическим моделированием, открывает новые возможности для понимания природы сознания и внутренней реальности.

    \begin{center}
        \textbf{Литература}
    \end{center}

    \begin{enumerate}
        \item Гете И.В. Фон. Избранные произведения по естествознанию. Том I. – СПб.: Издательство АН СССР. – 1957.
        \item Данте Алигьери; «Божественная комедия» [пер. с итал. и примеч. М. Лозинского]. –Москва: «Правда», 1982. – 640 с.
        \item Моисеев В.И. Логика всеединства / Воронеж. гос. мед. академия. – Воронеж. –1999.
        \item Моисеев В.И., Русская философия всеединства как прообраз интегральной науки, //Соловьевские исследования, 2024. – Вып. 3 (83), с. 38–48.
        \item Моисеев В.И. От логики всеединства к философии неовсеединства // Материалы междунар. науч. конф. «Философия В.С. Соловьева в межкультурной коммуникации: к 110-летию со дня смерти В.С. Соловьева и 20-летию праведной кончины протоиерея Александра Меня», Иваново, 23–25 сентября 2010 г. Иваново, 2010. – С. 86–90.
        \item Моисеев В.И. Мирология: Наука о мироподобных системах. – М.: ЛЕНАНД, 2022. – 600 с.
        \item Отто Р. Священное. Об иррациональном в идее божественного и его соотношении с рациональным. Пер. с нем. СПб.: Изд-во СПбГУ, 2008.
        \item Юнг, Карл Густав. Человек и его символы. – Санкт-Петербург: Серебряные нити, 2017. – 320 с.
    \end{enumerate}
