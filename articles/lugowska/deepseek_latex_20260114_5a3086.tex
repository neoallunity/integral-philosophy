\documentclass{article}
\usepackage[utf8]{inputenc}
\usepackage[T2A]{fontenc}
\usepackage[russian,english]{babel}
\usepackage{amsmath, amssymb}
\usepackage{graphicx}
\usepackage{adjustbox}
\usepackage{cite}
\usepackage[style=verbose-trad2]{biblatex}

% Добавьте этот путь к вашему BibTeX-файлу
\addbibresource{references.bib}

\newif\ifprintabstract
\printabstracttrue % или \printabstractfalse

\begin{document}

\ifprintabstract
    \begin{english}
    % Lugowska H. - English abstract
    \subsection{Lugowska H. PROJECTIVE-MODAL ONTOLOGY OF V. SOLOVYOV'S IMAGE IN D.S. MEREZHKOVSKY'S CRITICAL ESSAY "THE SILENT PROPHET": AN APPLICATION OF FORMAL APPARATUS TO LITERARY-CRITICAL DISCOURSE}
    \label{subsec:lugowska-en}

    {\itshape }

    This article proposes an application of the projective-modal ontology (PMO) framework—a formal philosophical apparatus developed by V.I. Moiseev for examining the structural relationship between essence (modus), manifestation (mode), and constraining conditions (models)—to the analysis of existential portrayal as a critical method in D.S. Merezhkovsky's literary-critical discourse. Specifically, the study reconstructs Merezhkovsky's characterisation of Vladimir Solovyov in the critical essay "The Silent Prophet" (1908) through the systematic application of PMO's formal apparatus, demonstrating how existential portrayal functions as a disciplined procedure of projective variation of a subject's fundamental nature across a multiplicity of ontological, temporal, and epistemological models.

    The projective-modal ontology, whilst originally developed for philosophical-logical inquiry, provides a metalanguage capable of formalising the conceptual structures underlying literary criticism. The PMO framework operates through a seven-place predicate establishing relationships between: a modus (the source or generator of being—in this case, Solovyov's fundamental gnostic-contemplative nature); modes (particular manifestations or aspects of that source); models (limiting conditions under which modes do or do not emerge); projectors and surjectors (operations of restriction and expansion); and contexts of determination.

    Critically, the concept of proper models—the set of conditions upon which a modus is capable of generating modes—proves essential: a modus cannot generate authentic modes on all conceivable conditions. For instance, a revolutionary pragmatist model lies entirely outside Solovyov's ontological capacity, rendering any attempt to project his essence onto such a model structurally impossible rather than merely improbable.

    The study identifies Solovyov's base modus as immanently gnostic and contemplative, rather than a neutral subject oscillating between contemplation and action. This fundamental characterisation renders the revolutionary pragmatist model improper—incapable of generating authentic modes. Through systematic textual analysis, the investigation reveals three proper temporal models upon which Solovyov's authentic modes emerge: (1) the past, generating the mode of "restorer" and secret Slavophile; (2) the present, generating the mode of "conservator" and supporter of collapsing structures; and (3) an eschatological future, generating the mode of "eschatologist" and herald of apocalyptic catastrophe. Conversely, attempts to project Solovyov onto models of revolutionary action produce not legitimate modes but social masks—in particular, the artificial persona of the philosophical systematist composed of "ten volumes" of rigorous argumentation, a pseudo-mode bearing all markers of inauthenticity through its excessive smoothness and polish.

    The article formalises Merezhkovsky's critical method as an asymmetrical conflict between two second-order modes: (1) the hidden prophet—ontologically true yet epistemologically inaccessible, incapable of articulation within public discourse; and (2) the public philosopher—discursively expressed yet ontologically false, a mere mask concealing the authentic prophetic truth. This generates the fundamental ontological aporia structuring Solovyov's tragedy: the authentic prophetic mode remains structurally inexpressible within the model of public philosophical discourse, whilst anything expressed through such discourse fails to capture the prophetic truth. The concept of "quiet eddy" (omut) is reinterpreted not as an actual synthesis of opposites but as an eschatological limit—a fusion achievable only in finite time, perpetually deferred beyond historical reality. Solovyov, according to Merezhkovsky's interpretation, exists before the attainment of such synthesis, dying before unity is achieved.

    The tragedy of the "silent prophet" is formalised as an ontological aporia arising from three structural impossibilities: (1) a fundamental mismatch between the modus required by historical epoch (pragmatist activist) and the actual modus of the subject (gnostic contemplative); (2) blockage of the projector operation, rendering revolutionary action a structural impossibility rather than a moral choice; and (3) fission of authentic and inauthentic modes, whereby the only possible public expression necessarily betrays the underlying truth. The muteness of prophecy emerges as a limit function: as one approaches absolute prophetic truth, the capacity for expression approaches zero. Absolute authenticity correlates with absolute inexpressibility.

    The investigation demonstrates the heuristic potential of PMO for formalising existential portrayal as a critical method. Through coordination of PMO structures with linguistic-rhetorical analysis, the study reveals that Merezhkovsky's critical procedure is isomorphic to ancient dialectical method—a systematic procedure of projective variation capable of precise formal representation. The formal apparatus provides a metalanguage unifying macro-logical structures (associative-semantic fields and conceptual networks identified through philological analysis) with micro-logical structures (specific rhetorical devices and textual markers). This coordination permits rigorous diagnosis of the ontological nature of Solovyov's tragedy as portrayed by Merezhkovsky, transforming intuitive literary criticism into a formally specifiable procedure.

    The study concludes that PMO-based formalisation yields five principal findings: (1) identification of the base modus as immanently gnostic, rendering pragmatic revolution an improper model; (2) reconstruction of temporal architecture through three proper models generating authentic modes; (3) reconceptualisation of the "quiet eddy" as an eschatological limit-transition; (4) formalisation of prophetic muteness as an ontological aporia; and (5) diagnosis of three types of ontological aporias structuring Solovyov's tragic characterisation. The article thus establishes PMO as a productive framework for literary-critical analysis, extending its application beyond its original philosophical domain to encompass the formal structures underlying existential portrayal in twentieth-century critical discourse.

    \paragraph{Keywords:} {\itshape Projective-modal ontology, V.S. Solovyov, D.S. Merezhkovsky, existential portrayal, mode-modus-model, gnosticism-pragmatism, temporality, tragedy of muteness, formalisation of critical method, ontological aporia, silent prophet, twentieth-century Russian literary criticism}
    \end{english}
\else
    \begin{russian}
    % Луговская Е.Г. - русская статья

    \subsection{\texorpdfstring{\textbf{Луговская Е.Г., Е.К. Грудина. ПРОЕКТИВНО-МОДАЛЬНАЯ ОНТОЛОГИЯ ОБРАЗА В.С. СОЛОВЬЕВА В КРИТИЧЕСКОЙ СТАТЬЕ МЕРЕЖКОВСКОГО Д.С. «НЕМОЙ ПРОРОК»: ОПЫТ ПРИМЕНЕНИЯ ФОРМАЛЬНОГО АППАРАТА ПМО К ЛИТЕРАТУРНО-КРИТИЧЕСКОМУ ДИСКУРСУ}}{Луговская Е.Г., Е.К. Грудина. ПРОЕКТИВНО-МОДАЛЬНАЯ ОНТОЛОГИЯ ОБРАЗА В.С. СОЛОВЬЕВА В КРИТИЧЕСКОЙ СТАТЬЕ МЕРЕЖКОВСКОГО Д.С. «НЕМОЙ ПРОРОК»: ОПЫТ ПРИМЕНЕНИЯ ФОРМАЛЬНОГО АППАРАТА ПМО К ЛИТЕРАТУРНО-КРИТИЧЕСКОМУ ДИСКУРСУ}}
    \label{subsec:lugowska-ru}

    \textit{В статье предложена реконструкция образа В.С. Соловьева в критической статье Д.С. Мережковского «Немой пророк» на основе аппарата Проективно-модальной онтологии (ПМО) В.И. Моисеева. Продемонстрировано, что критический метод Мережковского может быть формализован как систематическое варьирование модуса личности философа на множестве темпоральных, онтологических и эпистемологических моделей. Установлено, что базовый модус Соловьева идентифицируется Мережковским как имманентно гностический (созерцательный), что делает революционный прагматизм невозможной моделью образования моды деятеля. Трагедия «немого пророка» интерпретируется как онтологическая апория, возникающая из структурной невозможности одновременной выражаемости и аутентичности пророческой истины. Розовый башмачок реконструируется как материализованная модель прошлого, функционирующая одновременно как условие образования моды реставратора и как точка амбивалентного перехода между созерцанием и деятельностью. Метафора «омута» переинтерпретируется как эсхатологический закон неизбежного синтеза противоположностей, недостижимый в историческом времени. Исследование демонстрирует эвристический потенциал ПМО как метаязыка для описания концептуальных структур, выявленных лингвориторическим анализом.}

    \textbf{Проективно-модальная онтология (ПМО), В.С. Соловьев, Д.С. Мережковский, экзистенциальное портретирование, модус-мода-модель, формализация критического метода}

    \textbf{Введение: проблема формализации философского портретирования}
    Владимир Сергеевич Соловьев (1853–1900) функционирует в метадискурсе русской культуры как фигура парадоксальная: образ «великого учителя» символистов одновременно маркируется как «двойственный», «нераскрытый», «невидимый» \cite{Блок1906, Розанов1900, Трубецкой1913}. Критическая рецепция Д.С. Мережковского в статье «Немой пророк» (1908) представляет попытку экзистенциального портретирования, которое не сводится ни к биографическому описанию, ни к философской экспликации, но предстает как развертывание личности сквозь противоречивые модусы бытия.
    Предшествующий лингвориторический анализ \cite{ЛуговскаяГрудина2024} выявил, что Мережковский конструирует образ Соловьева через операционализацию бинарной оппозиции \textit{«философ-созерцатель» / «человек-деятель»}, актуализированной в общественном дискурсе начала XX века в контексте идеологических дискуссий о революции и реформации. Анализ выявил также систему риторических приемов (оксюмороны, катахрезы, апоретические вопросы, символические детали), порождающих семантическое напряжение и вовлекающих читателя в активный процесс конструирования смысла образа.
    Однако остается нерешенной проблема формализации критического метода как систематической процедуры, позволяющей локализовать точки экзистенциальной трагедии личности в ее онтологической структуре.
    Настоящее исследование направлено на применение аппарата Проективно-модальной онтологии (ПМО) В.И. Моисеева \cite{Моисеев2002, Моисеев2002аксиоматика} к анализу образа Соловьева у Мережковского. ПМО рассматривается здесь не как универсальный метод анализа художественного текста, но как метаязык описания для концептуальных структур, выявленных лингвориторическим анализом. Допускается метатеоретическое расширение: введение понятия множества собственных моделей для описания условий, при которых модус (личность) способен образовывать свои моды (аспекты существования).
    
    \textbf{Цель исследования:} формализовать концептуальную структуру образа В.С. Соловьева посредством предикатного синтаксиса ПМО, выявив онтологическую архитектонику трагедии «немого пророка» и верифицировав результаты против установленных лингвориторическим анализом фактов.
    
    \textbf{Теоретические основания: Проективно-модальная онтология В.И. Моисеева}
    
    \textbf{2.1. Базовый аппарат ПМО}
    Проективно-модальная онтология В.И. Моисеева разработана на основе онтологии С. Лесьневского с использованием категориального синтаксиса \cite{Моисеев2002}. Центральное место в ПМО занимает \textbf{семиместный предикат Mod}, имеющий следующий категориальный тип:
    
    [ФОРМУЛА: семиместный предикат Mod]
    
    \textbf{Определение: семиместный предикат Mod (a,b,c,f,d,h,$\alpha$)}
    Предикат Mod выражает отношение между семью элементами:
    
    \begin{adjustbox}{center}
    \begin{tabular}{|l|l|l|l|}
      \hline
      \textbf{Позиция} & \textbf{Переменная} & \textbf{Название} & \textbf{Интерпретация} \\
      \hline
      1 & $a$ & \textbf{мода} & Аспект, проявление, образуется из модуса \\
      \hline
      2 & $b$ & \textbf{модус} & Источник, генератор бытия, основание \\
      \hline
      3 & $c$ & \textbf{модель} & Ограничивающее условие, обстоятельство \\
      \hline
      4 & $f$ & \textbf{проектор} & Операция ограничения модуса до моды \\
      \hline
      5 & $d$ & \textbf{модуль} & Начало расширения моды до модуса \\
      \hline
      6 & $h$ & \textbf{сюръектор} & Операция расширения моды до модуса \\
      \hline
      7 & $\alpha$ & \textbf{спецификатор} & Контекст определения \\
      \hline
    \end{tabular}
    \end{adjustbox}

    \textbf{Неформальное прочтение:}
    
    [ФОРМУЛА: Mod(a,b,c,f,d,h,$\alpha$)]

    \textbf{Визуальная схема отношений (Моисеев, 2002: 217):}
    
    \begin{center}
    \begin{tikzpicture}[node distance=2cm]
    \node (modus) {модус $b$};
    \node (mode) [right of=modus] {мода $a$};
    \draw[->] (modus) -- node[above] {$f$ (проектор)} (mode);
    \draw[->] (mode) -- node[below] {$h$ (сюръектор)} (modus);
    \node (model) [below of=modus] {модель $c$ (ограничение)};
    \node (module) [below of=mode] {модуль $d$ (расширение)};
    \draw[dashed] (modus) -- (model);
    \draw[dashed] (mode) -- (module);
    \end{tikzpicture}
    \end{center}

    На основе семиместного предиката Mod определяются сокращенные формы, позволяющие исключать кванторы по отдельным компонентам \cite{Моисеев2002}.

    \textbf{Гипотеза применения ПМО к литературно-критическому дискурсу}
    В настоящей работе авторами предлагается интерпретационное расширение ПМО в части применения ПМО к образу личности, представленной в литературно-критическом дискурсе.

    Базовая гипотеза нашего рассуждения состоит в том, что литературно-критический образ реальной личности может быть реконструирован как модус, реализующийся в различных модах (аспектах существования) на множестве моделей (темпоральных, онтологических, эпистемологических условиях).
    
    Вводим понятие множества собственных моделей $M_{\text{соб}}(b)$ как совокупности условий $c$, на которых модус $b$ способен образовывать свои моды. Формально:
    
    [ФОРМУЛА: определение собственных моделей]

    \begin{center}
    \begin{adjustbox}{center}
    \begin{tabular}{|l|l|l|}
      \hline
      \textbf{Уровень анализа} & \textbf{Инструментарий} & \textbf{Результат} \\
      \hline
      \textbf{Уровень 1: Текстовый анализ} & Лингвориторический анализ АСП, риторические приемы & Выявление концептов «философ-созерцатель» / «человек-деятель», символических деталей, семантического напряжения \\
      \hline
      \textbf{Уровень 2: Концептуальная структура} & ПМО-синтаксис (Mod-предикаты) & Формализация концептов как модусов и мод, определение моделей и условий \\
      \hline
      \textbf{Уровень 3: Онтологическая архитектоника} & Анализ множеств собственных моделей, валентные определения & Выявление точек трагедии, онтологических апорий, невозможных мод \\
      \hline
      \textbf{Уровень 4: Верификация} & Сопоставление результатов ПМО и Л-анализа & Подтверждение или коррекция интерпретаций \\
      \hline
    \end{tabular}
    \end{adjustbox}
    \end{center}

    % ... остальная часть статьи с аналогичными исправлениями формул ...

    {\centering \textbf{Литература} \par}
    
    % Библиография теперь будет выводиться через biblatex
    \printbibliography

    \paragraph{Ключевые слова:} {\itshape Проективно-модальная онтология (ПМО), В.С. Соловьев, Д.С. Мережковский, экзистенциальное портретирование, модус-мода-модель, формализация критического метода}
    \end{russian}
\fi

\end{document}