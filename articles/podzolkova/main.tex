\ifprintabstract
    \begin{english}
    % Podzolkova N.A. - English abstract
    \subsection{INTEGRAL ETHICS}
    \label{subsec:podzolkova-en

    \subsubsection{Podzolkova N.A. \\ THE CONCEPT OF DEVELOPMENT IN INTEGRAL PHILOSOPHY}
    \label{subsubsec:podzolkova-title-en

    In contemporary philosophy and science, the problem of integrating various forms of knowledge—from rational analysis to mystical experience—has become increasingly pressing. Traditional approaches often oppose the subjective and the objective, the empirical and the transcendent, making it difficult to form a holistic worldview. In this context, there is growing interest in integral methodologies capable of overcoming dualism and proposing new ways of understanding development as a universal process. The article by N.A. Podzolkova explores the fundamental polarities of development within the framework of the philosophy of neoallunity—a direction that synthesises ideas from Russian religious philosophy (V.S. Solovyov), modern integral theory (Ken Wilber), and mathematised ontology (V.I. Moiseev). The author aims to identify key vectors of development and demonstrate how the methods of neoallunity can reconcile opposing aspects of being—from logical and mathematical structures to lived experience. The primary objective of this study is to reveal the structure of development through the lens of neoallunity philosophy, highlighting fundamental polarities (deepening–ascent, experience–awareness) and proposing tools for their synthesis. The author seeks to: (1) systematise types of development (deepening as movement towards the One, ascent as movement towards the Many); (2) demonstrate the interconnection of experience and awareness in the context of integral science; and (3) illustrate the application of neoallunity methods in the analysis of development. The research relies on: (1) polar analysis—identifying and synthesising opposites (One–Many, internal–external); (2) projective-modal ontology—viewing development as an absolute mode uniting its projections; (3) polyquantitative mathematical tools—pleromal number, the regime of mixed unfolding (achieving infinity in a finite number of steps), L-analysis for modelling interworld interactions; (4) comparative philosophical...
    Key concepts of development, in light of neoallunity philosophy, acquire new meanings and introduce novel terms, such as: (1) "uniqueversality"—a synthesis of the unique and the universal; (2) "intersubjectivity"—the dimension of encounters between inner worlds; (3) "reverse matter"—the matter of life and consciousness, for which unity is real, whilst multiplicity of forms is merely potential; (4) "pleromal number"—a number possessing actual infinity and the potential to transcend its own limits. The structure of development is presented in the form of a quadrant table (modelled after Ken Wilber's "Four Corners of the Cosmos"), which becomes increasingly complex throughout the study, incorporating new categories. The primary emphasis is placed on synthesis categories, which the author calls "bridges" between quadrants. These "bridges" include: the category of subjectivity—overcoming the dualism of internal and external; the category of unity-in-multiplicity—synthesising the structural and the holistic; the category of finfinity—finite infinity, enabling transcendence of local world boundaries; and the category of world-similarity—the capacity of a finite fragment of reality to exist as an infinite world.
    The author envisions the future development of integral science in: (1) the refinement of mathematical models of inner worlds (polyquantitative mathematics); (2) the application of worldology in psychology, ethics, and epistemology; (3) the development of intersubjective ethics. The methods of neoallunity can be applied in education, psychotherapy, and interdisciplinary research. As a practical example, the article demonstrates how Kant's categorical imperative can be interpreted using R-analysis constructs. This study shows that the philosophy of neoallunity offers effective tools for synthesising the polarities of development. The work opens new perspectives for integral knowledge, where development is understood not as linear progress but as a multidimensional process of world encounters. The author advocates for collective research under the auspices of the Institute of Integral Science, where all laboratories operate in a spirit of collaboration and co-creation.

    \paragraph{Keywords:} \textit{Integral science, development, neoallunity, polar analysis, intersubjectivity, uniqueversality, worldology, world-similarity}
    \end{english}
\else
    % Подзолкова Н.А. - русская статья
    \subsection{ЛАБОРАТОРИЯ ИНТЕГРАЛЬНОЙ ЭТИКИ}
    \label{subsec:podzolkova-ru

    \subsubsection{Подзолкова Н.А. \\ ОТ КЕНИГСБЕРГСКИХ МОСТОВ ДО МОСТОВ МЕЖДУ МИРАМИ: \\ КАК МЕТОДЫ ФИЛОСОФИИ НЕОВСЕЕДИНСТВА \\ СОЕДИНЯЮТ БАЗОВЫЕ ПОЛЯРНОСТИ РАЗВИТИЯ}
    \label{subsubsec:podzolkova-title-ru

    \textit{В современной философии и науке остро стоит проблема интеграции различных форм познания – от рационального анализа до мистического опыта. Традиционные подходы часто противопоставляют субъективное и объективное, эмпирическое и трансцендентное, что затрудняет формирование целостной картины мира. В связи с этим возрастает интерес к интегральным методологиям, способным преодолеть дуализм и предложить новые способы осмысления развития как универсального процесса. Статья Н.А. Подзолковой посвящена исследованию базовых полярностей развития в рамках философии неовсеединства – направления, синтезирующего идеи русской религиозной философии (В.С. Соловьев), современной интегральной теории (Кен Уилбер) и математизированной онтологии (В.И. Моисеев). Автор ставит задачу выявить ключевые векторы развития и показать, как методы неовсеединства позволяют соединить противоположные аспекты бытия – от логико-математических структур до живого переживания.}

    \textit{Основная цель работы – раскрыть структуру развития через призму философии неовсеединства, выделив базовые полярности (углубление–восхождение, переживание–осознание) и предложив инструменты их синтеза. Автор стремится систематизировать виды развития (углубление как движение к единому, восхождение как движение к многому), показать взаимосвязь переживания и осознания в контексте интегральной науки и продемонстрировать применение методов неовсеединства для анализа развития. Исследование опирается на 1) полярный анализ – выявление и синтез противоположностей (единое–многое, внутреннее–внешнее); 2) проективно-модальную онтологию – рассмотрение развития как абсолютного модуса, объединяющего свои проекции; 3) инструменты поликвантической математики – плерональное число, режим смешанного размыкания (достижение бесконечности за конечное число шагов), L-анализ для моделирования межмировых взаимодействий; 4) сравнительно-философский анализ – сопоставление идей В.С. Соловьева, Кена Уилбера, Ауробиндо Гхоша, Мартина Бубера и др.}

    \textit{Ключевые концепты развития в свете философии неовсеединства приобретают новые смыслы и подразумевают введение новых терминов, таких как: 1) «универкальность» – синтез уникального и универсального, 2) «интерсубъектность» – измерение встречи внутренних миров, 3) «обратная материя» – материя жизни и сознания, для которой единство является реальным, а многообразие форм – лишь потенцией, 4) «плерональное число» – число, обладающее актуальной бесконечностью и потенциалом выхода за свои пределы. Структура развития представлена в виде таблицы квадрантов (по типу «четырех углов космоса» Кена Уилбера), которая на протяжении исследования усложняется и обрастает новыми категориями. Основной акцент делается на категориях синтеза, называемых автором «мостами» между квадрантами. В качестве таких «мостов» выделяются: категория субъектности – преодоление дуализма внутреннего и внешнего; категория многоединства – синтез структурного и целостного; категория финфинитности – конечной бесконечности, позволяющий выходить за границы локальных миров; и категория мироподобия – способность конечного фрагмента реальности бытийствовать в качестве бесконечного мира.}

    \textit{Перспективы развития интегральной науки автор видит в углублении математических моделей внутренних миров (поликвантическая математика), в применении подхода мирологии в психологии, этике, теории познания, в разработке интерсубъектной этики. Методы неовсеединства могут быть использованы в образовании, психотерапии, междисциплинарных исследованиях. В качестве практических результатов в статье показано, как можно интерпретировать кантовский категорический императив с помощью конструкций R-анализа. Статья демонстрирует, что философия неовсеединства предлагает эффективные инструменты для синтеза полярностей развития. Работа открывает новые перспективы для интегрального знания, где развитие понимается не как линейный прогресс, а как многомерный процесс встречи миров. Автор нацелен на коллективную работу под эгидой Института интегральной науки, где все лаборатории находятся в состоянии содружества и сотворчества.}

    \textbf{Интегральная наука, развитие, неовсеединство, полярный анализ, интерсубъектность, универкальность, мирология.}

    \paragraph*{Введение}
    Идея развития является ключевым концептом интегральной философии (см. Манифест интегральной философии «На пути к обществу развития» [Моисеев, 2025(1): 9]). Структура развития в философии неовсеединства подразумевает не просто наличие «правильного направления развития», но и более мощный онтологический статус этого направления. Для проективно-модальной онтологии это абсолютный модус, собирающий все свои моды-проекции, для полярного анализа – главный вектор синтеза. Важно, что этот вектор не возникает из своих проекций, но обнаруживается через них, служит для них своеобразным аттрактором. «Развитие, таким образом, не есть создание вектора синтеза, но есть стремление к нему. Понятие «эмерджентность» – это сюръекционное действие, указывающее на некоторый труднодоступный модус (источник синтеза). Кажется, что этот модус возникает из ниоткуда – чудесным образом, но само Чудо есть не что иное как расширяющее условие развития» [Подзолкова, 2025: 128].

    Понимание структуры развития, конечно, не обеспечивает и не гарантирует самого развития, но очень ему способствует. Если понимать логику развития, ориентироваться в базовых антиномиях, указывающих направление синтеза, то можно двигаться быстрее: фактически «со скоростью» научного исследования, а не только философского обобщения. В этом суть интегральной науки – стремиться к \textit{структурно выраженному многоединству} [Моисеев, наст. изд.: 31-54]. Задача данной статьи – выявить базовые составляющие развития и попытаться применить к ним методы, разработанные в рамках философии неовсеединства.

    \paragraph*{1. Основные полярности развития}
    В продолжение идей В.С. Соловьева, касающихся законов развития [Соловьев, 1990: 140-178], отметим несколько важных моментов с точки зрения полярной динамики процесса развития.

    Во-первых, развитие может проявить себя в ипостасях \textit{углубления} и \textit{восхождения}. Углубление связано с обнаружением скрытых взаимосвязей. Например, пресловутая способность «понять прежде, чем увидеть», в которой заключена суть научного поиска [Ровелли, 2020: 22]. Восхождение связано с обнаружением нового: новых территорий (внешних и внутренних), новых видов (растений, животных, частиц) или же новых составляющих какого-то сложного процесса. Развитие-восхождение движется от полюса единого к полюсу многого, а развитие-углублении – наоборот от полюса многого к полюсу единого. Многие ходили по кенигсбергским мостам, но только Леонард Эйлер нашел \textit{общие правила} уникурсальных маршрутов (движение к полюсу единого). Из этого открытия родилась новая область геометрии – топология – которая стала развиваться и расширяться (движение к полюсу многого). Так проявляется «мерцание» индукции-дедукции – от поиска единого правила к применению этого правила на всем многообразии изучаемого предмета. Но этими двумя ипостасями развитие, конечно, не исчерпывается.

    Например, этика может проявляться через полюс углубления, то есть усиления единства и взаимосвязанности (общности) живых сущностей, но очевидно, что суть этого единства отличается от общего математического доказательства. Мы имеем дела с разными проявлениями единства и связности элементов целого. В одном случае, это «парменидова связность» – незыблемая и воспроизводимая разумом, в другом случае «гераклитова связность» – текучая и переливающаяся субстанция переживания общности как взаимообусловленности, зависимости каждого от каждого.

    Оба типа связности очень важны, и было бы большой ошибкой идеализировать только один из них. Между тем, это регулярно происходит в человеческой культуре. В статье Г.С. Померанца «Две модели познания» отмечается, что эти типы связности имеют неравновесное признание: \textit{«Дети называют их «как будто» («понарошку») и «в самом деле» («взаправду»)} [Померанц, 1995: 41]. Здесь переливчатая субстанция глубинной связности живых сущностей («гераклитова связность») лишена статуса реальности. «Однако, \textit{– }продолжает Померанц, \textit{– «как будто» для ребенка не менее истинно, чем «в самом деле». В мире «как будто», играя, он входит в ритм жизни и становится маленьким чудотворцем: превращает палку – в лошадку, щепку – в Летучего Голландца. В этом мире он всеведущ и всеблаг} [Померанц, 1995: 41]\textit{. }Что же перед нами: всего лишь детское состояние сознания, не созревшее до «критериев демаркации», или же в этом переживании скрыто одно из базовых свойств Духа-творца? Для Померанца ответ однозначен: «Человеческое сознание с первых шагов создает и разрабатывает \textit{обе основные модели мира}: мир как текучее переливчатое Целое, охваченное единым ритмом (Дао, «вечно живой огонь»); мир как совокупность атомарных фактов, жестко расчлененных и затем связанных более или менее точно фиксированными отношениями (равенство и неравенство, причинность и вероятность и т.д.)» [Померанц, 1995: 41-42, \textit{курсив внутри цитаты мой – Н.П.}].

    Мощное подтверждение этой идеи можно найти в интегральной йоге Ауробиндо: «Истины вселенского существования бывают двух видов: первый – это истины духа, которые сами вечны и неизменны – великие истины, которые сами вызывают свое становление и постоянно реализуют в нем свои силы и значения, и второй – это игра сознания с этими истинами – диссонансы, музыкальные вариации, звучание возможностей, восходящие каденции, возвращения, повторы и постепенные переходы в более высокие сферы гармонии…»\textit{ }[Сатпрем, 2023: 45]. Ауробиндо считает этот второй вид истин (казалось бы, «иллюзорных» и «обманчивых» для мировосприятия йогина) не менее важным. Интересно, что он связывает их с традициями прагматичного и непоседливого западного сознания, которому хочется самому чувствовать себя Творцом, хочется видимых изменений и осязаемого становления. И Ауробиндо не осуждает западное сознание за эту тягу, поскольку она также важна и законна: «у Востока и Запада два взгляда на жизнь, которые являются противоположностями одной реальности. Между прагматической истиной, с одной стороны, на которую такое сильное и исключительное внимание обращает витальная мысль современной Европы [\textit{вспомним здесь А.Бергсона и Ф.Ницше, Ч.Пирса и У.Джеймса, а также экзистенциалистов и персоналистов – Н.П.}], очарованная энергией жизни и пляской Бога в Природе, и вечной, неизменной Истиной – с другой, в которой, в свою очередь, очарованный покоем и равновесием индийский разум любит обращаться с той же страстью к уникальным открытиям, нет тех раздоров и противоречий, о которых ныне заявляют пристрастный ум, рассудок, разделяющий все на части…» [Сатпрем, 2023: 44-45].

    Таким образом, мы можем ввести еще две полярные ипостаси развития, связанные с \textit{переживанием} и \textit{осознанием}. Мы видим, что для каждой из этих полярностей полюсы углубления и восхождения принимают свои очертания или, говоря языком проективно-модальных онтологий, свои моды-проекции.

    В случае разворачивания многообразия как переживания мы обнаруживаем за пределами своего мира множество новых миров. И это не метафора, а удивительный факт, усвоить который очень непросто, несмотря на его кажущуюся очевидность. Сложность заключается в том, что для подлинного обнаружения других миров мы должны покинуть территорию собственного мира, то есть совершить \textit{акт подлинного трансцендирования} – выхода за пределы своего опыта. И хотя умозрительно процедура эта может показаться элементарной, чтобы \textit{пережить} такой выход, нужно буквально дойти до границы внутреннего мира и пересечь ее. Иногда такой выход может случиться только после физической смерти, но даже она не гарантирует для индивидуального сознания настоящего преодоления. Просветленные учителя, йогины, святые самых разных конфессий находят свои уникальные методы, чтобы совершить этот акт развития-восхождения, продиктованный логикой интегрального развития, и выйти к многообразию миров, ощутить себя во вселенной живых актуально бесконечных измерений. Но в большинстве случаев уникальность выбранного пути остается достоянием только самого идущего или очень избранного круга учеников. В рамках R-анализа предлагается увидеть \textit{структуру} этого преодоления-трансцендирования (рассмотрим ее чуть позже). А понимание структуры, хоть и не гарантирует самого переживания, но, как мы уже отметили во Введении, помогает двигаться в нужном направлении.

    Что же касается моды-проекции развития как восхождения-к-многому-через-осознание, то здесь мы имеем более привычную картину познавательной деятельности. Кен Уилбер назвал ее – \textit{трансляционной}, связанной с раскрытием элементов данного мира («поверхностных структур») и не требующей трансцендирования. Уилбер пишет, что «трансляционный процесс будет не общим изменением уровня [в отличие от трансформации], а просто сменой «языка» или формы на том же уровне» [Уилбер, 2004: 73]. По Т.Куну именно так в пределах очередной парадигмы развивается «нормальная наука». Развитие это может быть мощным, стабильным и длительным. И нужно только помнить и понимать, что это не полное развитие (не развитие-как-таковое), а лишь одна из его необходимых и важных проекций.

    Обобщим сказанное в виде таблицы по типу уилберовских квадрантов (Таблица 1). Здесь «внутреннее-внешнее» трансформируется в «переживание-осознавание», а «индивидуальное-коллективное» – в «единомногое-многоединое». Тем не менее базовые уилберовские категории «внутреннее-внешнее» и «единое-многое» продолжают прослеживаться. Интересно, что среди категорий Канта нет «внешнего-внутреннего», потому что эта дихотомия относится у него не к категориям рассудка, а уже к идеям разума (Мир-Душа), где эта дихотомия негласно «снимается» вектором синтеза (Бог).

    \begin{table}[ht!]
    \centering
    \caption{Структура развития, представленная двумя парами базовых полярностей}
    \begin{tabular}{|c|c|}
    \hline
    \multicolumn{2}{|c|}{\textbf{Развитие как движение}} \\
    \hline
    \textbf{Переживание (эмпатия)} & \textbf{Осознавание (рефлексия)} \\
    \hline
    Углубление (к Единому) & Углубление (к Единому) \\
    \hline
    Восхождение (к Многому) & Восхождение (к Многому) \\
    \hline
    \end{tabular}
    \end{table}

    \paragraph*{2. Исследование полярностей развития с помощью средств философии неовсеединства}
    Мы только что задали две шкалы и рассмотрели возникающие в их пределах характеристики развития «в общем виде» (можно сказать, алгебраически). Попробуем теперь подставить те конкретные «значения», которые эти характеристики принимают в интегральной философии (Таблица 2).

    \subparagraph*{2.1. Универкальность и интерсубъектность}
    Действительно, все базовые полярности, когда мы говорим о них в контексте синтетического знания, должны приобрести как бы более насыщенную окраску, обогатиться новыми оттенками смыслов, одновременно захватывая эти смыслы у «соседних территорий» и углубляя свои собственные. Так на месте переживания внутренней связности возникает новый концепт – \textit{универкальность}, а на месте переживания всеобщей связности возникает новый концепт \textit{интерсубъектность}. Оба концепта взаимодополняют друг друга.

    В слове «универкальность» сливаются два понятия «универсальность» и «уникальность», которые не исключают, а усиливают друг друга, как и должно быть в настоящем синтезе. В первую очередь, термины отсылают нас к закону прямого отношения индивидуального и универсального В.С. Соловьева [Соловьев, 1993: 289], где работает та же логика всеединства. Если не связывать понятие «уникальность» с эгоическими переживаниями (которые неповторимы и единственны только на первый взгляд, а при более пристальном рассмотрении – банальны и типичны), то оно раскрывается через свободу некоторого нового внутреннего созерцания, превышающую привычные границы внутреннего виденья. Переживания подлинной уникальности затрагивают самое ценное и существенное в нас – то, ради чего действительно стоит быть, причем \textit{быть всем и быть всегда }(как в песне «Пусть всегда будет солнце...»). «Художник, забыв на время самого себя, становится способным увидеть в мире важное-для-всех-без-исключения и запечатлеть его во внешнем (по сути, инородном для внутреннего переживания) материале: слове, звуке, краске, мраморе... То что он видит, есть существенное. Но не безразлично-существенное формальной логики» [Подзолкова, 2021: 262]. Так через уникальность и безусловную ценность мы прикасаемся к существенному для каждого, то есть в полной мере универсальному. Это и есть модус «универкальности» для полярных векторов (мод) универсальное-уникальное. Этот модус тут же сам превращается в моду и участвует в новой динамике развития.

    Но куда направлен этот вектор «универкальности», к чему он движется, в чем его цель? Та самая цель, \textit{которая действительно оправдывает средства, и которую постоянно путают с внешней целью}, порождая тем самым чудовищные извращения. (Ведь даже Бог часто является только внешней целью, а потому недостижимой, далекой и трансцендентной). Внутренняя цель, о которой идет речь – не субъективная, не имманентная и даже не времен\textbf{а}я (в смысле физического времени), ее нельзя лишиться, но очень легко перестать замечать, \textit{потому что не она рождается внутри нас, а мы находимся внутри ее притяжения, несмотря на то, что она не снаружи }(без структур поликвантической математики это предложение остается логически противоречивым). Цель, которая неотделима от своих средств, и сама является их сущностью. Она не оправдывает, а наполняет собой, формирует и оживляет средства. Средства без этой внутренней цели – просто «скорлупа». Это нечто внутри нас, что превышает нас самих, и что можно найти, только углубляясь и не присваивая. Это и есть \textit{область} \textit{интерсубъектного, доступная пока большинству из нас лишь в редкие моменты универкальности}. Это выход из платоновской пещеры, который на самом деле не наружу, а внутрь (путь погружения и в то же время путь исхода), но это «внутреннее направление» оказывается еще более бескрайним, чем любой внешний космос. И, конечно, это не аберрация зрения, не бесплодная пустыня растянутой до бесконечности индивидуальной монады, но \textit{место Встречи} всего, имеющего внутреннее измерение, т.е. всего онтологически реального (не иллюзорного), живого.

    Сравним два высказывания. Протагоровское: «человек – мера всех вещей» (как начало морального релятивизма) и «Человек – общая мера для всех народов и рас...» Экзюпери [Сент-Экзюпери, 1980: 382] (как необходимости всем сбыться, перестать быть иллюзией). Если у Протагора человек (с маленькой буквы) – носитель навязываемой друг другу субъективности, то у Экзюпери Человек (с большой буквы) – внутренняя интерсубъектная связанность всех людей. «Если нет ничего, что было бы больше тебя, тебе неоткуда получать. Разве что от себя самого. Но что получишь от зеркала?» [Сент-Экзюпери, 2024: 187]. По Экзюпери сбыться означает перестать оставаться «мыльным пузырем» своей замкнутой субъективности. Простой эмпатии для этого недостаточно. Необходимо преодолеть укоренившийся эгоизм и всем вместе выйти в общее внутреннее измерение, превышающее каждого в отдельности, по-настоящему встретиться там с бесчисленными живыми мирами всех живых существ.

    Наше эго – изначально важная и полезная структура, отвечающая за целостность личности – на данном этапе развития, очевидно, превышает свои полномочия и серьезно препятствует познанию себя и мира. В связи с чем этика и гносеология очень тесно сплетаются. Сегодня сбывающегося человека – человека, переживающего свою уникальность именно как универкальность, как выход в интерсубъектное измерение – чаще всего называются мистиком, иногда святым, иногда поэтом или художником, но почти никогда – ученым. Когда-то маг совмещал в себе направление духовного (в том числе, этического) и научного развития, но, должно быть, со временем своими средствами познания маги дискредитировали себя. Маг привносил в интерсубъектное пространство законы внешнего мира (законы прямой материи), вместо того, чтобы делать наоборот – оживлять внешний мир любовью (инозаконностью) интерсубъектного измерения. Приходит время новых ученых, новых магов – которым будут чужды эгоические устремления, но радость узнавания и дарения ценного-для-каждого-существа станет для них основным мотивом деятельности. Вспомним сотрудников НИИЧАВО братьев Стругацких. Мечта о такой интегральной науке жила среди ученых во все времена, хотя и принимала свои вполне исторические формы (ученые Государства Платона, ученые «Утопии» Томаса Мора, ученые-просветители, ученые-атеисты…)

    Тем не менее, даже сплетаясь с этикой, концепты «универкальности» и «интерсубъектности» так и останутся «чисто гуманитарными», нестрогими, и как будто даже этически необязательными, если не возникнет инструментарий, связывающий их с внепроизвольной (то есть не зависящей от чьей-то субъективной воли) необходимостью, как это происходит, пожалуй, только в математике. И хотя вместить всю глубину сбывающегося мира математике пока не под силу, она непрерывно совершенствуется, создавая (наперегонки с реальностью) новые структуры для новых граней проявленного и недопроявленного сущего.

    В интегральной науке поэтическая «универкальность» и мистическая «интерсубъектность» уже становятся достояниями строгого дискурса. Как будто terra incognita «основного слова Я-ТЫ» Мартина Бубера [Бубер, 1993 (2)] и «узла бытия Я-МЫ» Антуана Сент-Экзюпери [Сент-Экзюпери, 2024] обрела «свое место на карте». В нахождении своего положения в структуре нет ничего уничижительного для глубинных переживаний. Еще Владимир Соловьев утвердил «мистический способ познания» как равноправный наряду с эмпирическим и рациональным [Соловьев, 1990]. Так постепенно возникает область апофатической гносеологии [Подзолкова, 2012], а поэзия оказывается тканью времени материи внутренних миров [Подзолкова, 2020].

    В правом столбце нашей таблицы (Таблица 2), где речь идет уже не столько о переживании, сколько об осмыслении переживаемого, как раз расположены новые базовые концепты развития, делающие возможным математизацию внутренних измерений.

    \subparagraph*{2.2 Плерональное число и обратная материя}
    Во-первых, новая математика, способная говорить о внутренних мирах, строится на понятии плеронального (пифагорейского) числа [Моисеев, 2022 (2): 270-273]. Это старое-новое архэ, до которого необходимо углубиться, чтобы начинать всякое восхождение.

    Если мы принимаем за аксиому наличие у всего живого внутреннего измерения\footnote{«Живой называется сущность, которая обладает собственным внутренним миром» – принципиальное для философии неовсеединства и всей интегральной науки определение феномена жизни [Моисеев, 2012: 53].}, (а в более сильной гипотезе наличие внутреннего измерения признается необходимым у всего реально существующего\footnote{Гипотеза об «интерсубъектности», которая обсуждалась выше, предполагает, что любое сущее в той или иной мере укоренено в интерсубъектном, т.е. существование прямой (средовой) материи, не подкрепленное обратной материей жизни (внутренним измерением), фактически равно небытию [Подзолкова, 2019: 73]. Схожие постулаты выдвигает в «Чтениях о богочеловечестве» В.С. Соловьев [Соловьев, 1989: 51], к тем же выводам приходит интегральная йога Ауробиндо: «Если бы хоть одна точка вселенной была лишена сознания, то вся вселенная была бы лишена его, потому что бытие должно быть единым» [Сатпрем, 2023: 86].}), то исследование этого измерения должно оперировать некоторыми числовыми сущностями, подобными живым мирам. Из безликого числа, редуцированного в одномерную потенциально бесконечную прямую, не возникнет эмерджент жизни.

    Вот почему сама мера количества для выражения структур живых миров должна быть особенной, обладающей собственной полнотой (плерональностью) и завершенностью (финитностью). Это число-мир, число-полнота, число-сила, способная создавать миры. Подобными характеристиками наделял числа Пифагор, поскольку был заворожен пронизанностью всего существующего числовыми соотношениями и гармониями.

    В.И. Моисеев вводит в структуру нового плеронального числа ненулевой угол наклона числовой оси, цикличность и финитность. Это далеко не полный перечень «волшебных преобразований», но даже этих трех характеристик достаточно, чтобы повысить уровень сложности единицы измерения с нуля до бесконечности. «Плерональное число обобщает обычное число и в простейшем случае выступает как финитный натуральный ряд, который занимает один виток спиральной числовой структуры...» [Моисеев, 2012: 136]. «Если на эту спираль смотреть сверху, мы увидим только цикл, если учитывать только линейную составляющую – только линию» [Моисеев, 2012: 135] (Рис. 1).

    \begin{figure}[ht!]
    \centering
    \includegraphics[width=0.547\linewidth]{articles/podzolkova/images/image088.png}
    \caption{Спиральная структура плеронального числа}
    \label{fig:pleromal-number}
    \end{figure}

    \begin{figure}[ht!]
    \centering
    \includegraphics[width=0.544\linewidth]{articles/podzolkova/images/image085.png}
    \caption{Дополнительная визуализация концепции плеронального числа}
    \label{fig:pleromal-number-2}
    \end{figure}

    Именно такими сильными числами-плеронами привычная материя (из которой строится видимый и воспринимаемый органами чувств космос и которая по всем законам термодинамики стремится к равномерному остыванию – «тепловой смерти»), разворачивается вспять, превращаясь в обратную и вскипая нарастающей сложностью сознания и чувствования. Так в правом нижнем квадранта нашей таблицы (Таблица 2) появляется концепт «обратной материи жизни», уже упоминавшийся ранее в нашей статье\footnote{Неизбежная особенность данной статьи в невозможности абсолютно последовательного изложения, ведь все векторы (квадранты) существуют одновременно, друг на друга влияют и являются проекциями одного и того же вектора синтеза на разные модели развития. Поскольку статья является исследованием взаимосвязей уже серьезно разработанных понятий, эта непоследовательность компенсируется автором ссылками на соответствующие источники.}. В этой материи стремление к единому преобладает над стремлением ко многому (в то время как в обычной средовой материи многое преобладает над единым), и мы расположили его в нижней («многоединой») части таблицы только потому, что рассматриваем эту материю по отношению к плерону, из которого она разворачивается.

    Есть гипотеза, что таким плероном является не число, а поэтическое слово [Подзолкова, 2020], но это тема для отдельного исследования совместно с лабораториями философии слова и интегральной математики. Здесь вряд ли заключено противоречие – скорее, еще более глубокий синтез. Любой плерональный виток является определенным ритмическим фрагментом – фракталом, узором, вибрацией, строфой… Тут каждый исследователь найдет синоним из своей области и будет по-своему прав. Главное, что структура плеронального числа – сама живая. Она «дышит», развивается, движется, а потому способна передать структуру по-настоящему живой ткани бытия – двуединой (полной) материи, способной удержать в себе и материальное, и духовное измерение.

    Логично предположить, что если для обычной материи активным началом континуума является пространство, то для обратной материи внутренних миров– время. Активность времени строго соответствует теории Времени Н.А. Козырева [Время и звезды, 2008]: «Н.А. Козырев ввел понятие активных свойств Времени. Заглавной буквой он указывал, что речь идет не о метрическом свойстве времени, а об активном, творческом и субстанциональном начале». [Время и звезды, 2008: 278]. Активное Время вносит во все процессы целесообразность и смысл, который берется изнутри внутренней бесконечности, невидимой для внешних органов чувств (свернутой в бесконечной полноте каждого плерона\footnote{Далее будет показано, как эта свертка разворачивается и начинает действовать (см. о процедуре смешанного размыкания по В.И.Моисееву).}): «Если в физических процессах причина лежит в прошлом по отношению к следствию, то Время словно меняет причину и следствие местами. Событие в будущем реализуется потому, что Время выстраивает цепочку неопределенностей таким образом, чтобы это событие смогло произойти» [Время и звезды, 2008: 279].

    Совершенно не случайным образом в наши базовые полярности проникла еще одна полярность: «время – смысл», проявив себя совершенно неожиданным образом. Очевидно, что переживание – это область временения, а осознавание – область вневременных смыслов. Однако описанные выше концепты интегральной философии содержат такой мощный «заряд» синтеза, что он не замедлил проявить себя в парадоксальной «пропорции инь-ян»: так временные характеристики универкальности и интерсубъектности пропитались смыслами, что превратило их в переживания ценности бытия всего существующего, а смысловые характеристики плеронального числа и обратной материи пропитались временем, что сделало их живыми и динамично развивающимися сущностями. Точно так же, как на известном китайском символе в темной области инь светится кружок ян, в светлой области ян темнеет горошинка инь. Подобное взаимопроникновение полярностей мы рассматривали уже между универкальностью и интерсубъектностью, между плерональным числом и обратной материей. Вообще, процедура взаимного проникновения понятий есть не что иное как проецирование полярных векторов на вектор синтеза, отображающего на себе меру развития. В данном случае – меру развития самого развития…

    \begin{table}[ht!]
    \centering
    \caption{Структура развития, представленная концептами философии неовсеединства}
    \begin{tabular}{|c|c|}
    \hline
    \multicolumn{2}{|c|}{\textbf{Развитие}} \\
    \hline
    \textbf{Переживание (область временения)} & \textbf{Осознавание (область смыслов)} \\
    \hline
    Универкальность & Плерональное число \\
    \hline
    Интерсубъектность & Обратная материя \\
    \hline
    \end{tabular}
    \end{table}

    \paragraph*{3. Наведение мостов между базовыми полярностями}
    Наблюдая взаимопроникновение наших концептов как в символе инь-ян, мы подошли к пониманию того, что важнее не просто классифицировать виды развития и описывать проекции главного вектора на условные плоскости разных языков бытия, но найти способы наведения мостов между самими проекциями-языками. Деятельность Института интегральных наук как раз нацелена на создание таких мостов – языков, понятных по обе стороны. Кен Уилбер, например, не ставит в своих исследованиях подобной задачи. Он считает, что в рамках каждой модели существует свой язык, а суть интегрального подхода заключается в учете всех возможных языков, в умении переключаться между ними, поднимаясь (или спускаясь) от уровня к уровню.

    В интегральной науке главным связующим звеном и языком для описания всех полярностей развития постепенно становится язык поликвантической математики. Его особенность в том, что он, будучи по-научному структурным и строгим, остается по-философски объемным и антиномичным, и даже поэтически метафоричным, что мы уже почувствовали, говоря о характеристиках числа-плерона, лежащем в основании поликвантической математики. Но рассмотрим более подробно те области знания, которые разрабатываются сейчас средствами философии неовсеединства (Таблица 3).

    Мостом между переживанием единства и осознаванием единства является концепт «субъектности», снимающий дуализм внешнего и внутреннего, субъекта и объекта, эмпатии и рефлексии. В интегральной этике субъект рассматривается в качестве носителя внутреннего мира (субъектной онтологии), а не как противоположность объекту. Ярлык «субъективности», долгое время мешающий развиваться этическим теориям, заменяется структурами субъектных онтологий, которые, как и любые другие математические структуры, подлежат исчислению и анализу. Теперь субъектность – это характеристика материи жизни, а не показатель произвольности теории (см. Постулаты интегральной этики https://allunity.ru/etics.shtml). Кстати, в индийской философии пытались решить это противоречие, вводя понятие «недуальной» философии (адвайты), но здесь не был достаточно развит концепт «объектности», поэтому синтез касался, скорее, левых квадрантов Атмана и Брахмана (в нашей интерпретации: «универкальности» и «интерсубъектности»). Только в XIX-XX веках западноевропейские концепты попадают в поле зрения восточных Учителей, и возникает почва для более глубокого синтеза. Ауробиндо отмечал, что «противоречие между внутренним и внешним не более, чем еще одна догма нашей ментальности» [Сатпрем, 2023: 71].

    Концепт «многоединства», очевидно, является базовым концептом интегральной философии и, дополняясь структурностью, входит в МЕС: многое – единое – структурное [Моисеев, наст. изд.]. Многоединство – это развитие концепта «всеединство» В.С. Соловьева, мост между верхними и нижними квадрантами, задающий главные полюсы для встречного движения прямой и обратной материи, о которых речь пойдет дальше.

    На схеме (Таблица 3) мы разместили «многоединство» в качестве «левого» моста только для того, чтобы отдельно высветить еще один важный аспект многоединства – концепт «финфинитности» (конечной бесконечности), обеспечивающий процедуру смешанного размыкания. Если «многоединство» позволяет работать с полярной динамикой процессов – видеть работу единого и многого в рамках той или иной субъектной онтологии или мира-в-целом (Всемира), то благодаря процедуре смешанного размыкания происходит выход за границы малого мира – достижение бесконечности за конечное число шагов [Моисеев, 2025: 98-106]. Эта процедура огромной важности для современной науки, позволяющая преодолеть множество «безвыходных» ситуаций в математике, физике, этике и многих других отраслях знания.

    Для примера рассмотрим две формулировки кантовского категорического императива в свете концепта финфинитности. Первая формулировка – «поступай так, чтобы максима твоей воли могла в любое время стать всеобщим законодательством» – прекрасно иллюстрируется вертикальной моделью несозмеримости бесконечно малых и конечных величин в моноквантической модели исчисления (Рис.1) [Моисеев, 2025 (2): 94-98].

    \begin{figure}[ht!]
    \centering
    \includegraphics[width=0.529\linewidth]{articles/podzolkova/images/image094.png}
    \caption{Вертикальная модель несозмеримости (Кант, формулировка 1)}
    \label{fig:kant-vertical}
    \end{figure}

    \begin{figure}[ht!]
    \centering
    \includegraphics[width=0.555\linewidth]{articles/podzolkova/images/image093.png}
    \caption{R-функция для достижения соизмеримости}
    \label{fig:r-function}
    \end{figure}

    При перпендикулярности осей Х и У было бы невозможно говорить о соизмеримости индивидуальной воли (измерение абсолютно малых – ось У) с всеобщим законодательством, применимым для всего человечества во всякой ситуации (измерение конечных величин – ось Х). Тем не менее, Кант говорит об этих понятиях, как будто их возможно и даже необходимо привести в соответствие друг другу. Это можно осуществить, благодаря наклону оси бесконечно малых (то есть индивидуальной воли) с помощью R-функции (как показано на рисунке), осуществляя таким образом все более заметную проекцию на ось конечных величин (всеобщее законодательство, совокупная воля человечества) гипотетически вплоть до их полного совпадения. Во всяком случае, категорический императив задает стремление к соизмеримости и сопоставимости, заявляет об этом как о должном (и вовсе не бессмысленном) векторе человеческого нравственного усилия.

    Вторая формулировка императива – «поступай так, чтобы ты всегда относился к человечеству и в своем лице, и в лице всякого другого, также как к цели и никогда не относился бы к нему только как к средству» – прекрасно передается горизонтальной интерпретацией сопоставимости бесконечно малых и конечных величин (Рис.2).

    \begin{figure}[ht!]
    \centering
    \includegraphics[width=0.586\linewidth]{articles/podzolkova/images/image095.png}
    \caption{Горизонтальная модель соизмеримости (Кант, формулировка 2)}
    \label{fig:kant-horizontal}
    \end{figure}

    \begin{figure}[ht!]
    \centering
    \includegraphics[width=0.604\linewidth]{articles/podzolkova/images/image081.png}
    \caption{Развертывание точки в новое измерение}
    \label{fig:point-unfolding}
    \end{figure}

    Здесь то, что было неразличимой лишенной размерности точкой в моноквантической системе исчисления (человек как средство, не имеющее принципиальной ценности), вдруг разворачивается в отрезок, имеющий свой размер, а при более близком «наведении» превращается в новую бесконечную числовую ось (собственный порядок исчисления, новое измерение), с которым уже нельзя не считаться в прямом (математическом) и переносном (этическом) смыслах.

    Важно, что в рамках этики неовсеединства предлагается собственная формулировка категорического нравственного императива, исходящая из постулата развития: «То обладает подлинно нравственным характером, что направляет жизнедеятельность сообщества разумных существ на усиление многоединства бытия» [Моисеев, 2018: 69].

    Наконец, мостом между переживаемым как встреча в интерсубъектном пространстве многоединством и осознанием этого многоединства в качестве взаимодействия прямой и обратной материи служит концепт «мироподобия» – способность частей мира бытийствовать в качестве целых миров. Мироподобием занимается наука «мирология» – новая область знания, изучающая все многообразие сущих как множество малых миров, обладающих своим пространством-временем, своими сущими характеристиками, своей полнотой и относительной (не абсолютной!) замкнутостью. Например, внутренние миры живых существ рассматриваются в мирологии как безусловно полноценные миры, а не только ускользающе незначительные свойства материальной Вселенной. Проблемам мирологии посвящена отдельная монография В.И. Моисеева [Моисеев, 2022 (1)] и выпуск журнала «Интегральная философия» №12 (2022 г.). В качестве моста между полюсами развития нас интересует, в первую очередь, мироподобная математика, которую ее создатель В.И. Моисеев определяет как «математику, оперирующую, с одной стороны, структурами мир-бытия, т.е. структурами пространства и времени, материи, сущих и законами; с другой, – работающей со структурой мир-бытия как такового, где все его составляющие дают единую целостность» [Моисеев, 2022 (3): 26].

    Мирология рассматривает не только существование рядоположенных миров, но и возможность существование миров на разных слоях реальности (см. процедуру смешанного размыкания), что ставит вопрос взаимодействия миров в принципиально новый контекст. «Область встречи миров с необходимостью находится в другом измерении, попасть в которое можно только изнутри мира как живого и целого. <…> Мирология как новая отрасль знания точнее может описать «окрестности» этого погружения. Во всяком случае, новый математический аппарат L-анализа («L» от анг. «layer» – слой), который разрабатывает В.И. Моисеев [Моисеев, 2022 (1):25-27], описывая межслойные взаимодействия, позволяет исследовать не только поверхностные структуры живых миров, но и глубинные структуры» (Рис.4) [Подзолкова, 2022: 71].

    \begin{figure}[ht!]
    \centering
    \includegraphics[width=0.883\linewidth]{articles/podzolkova/images/image109.png}
    \caption{Модели взаимопонимания в рамках мирологического подхода}
    \label{fig:worldology-models}
    \end{figure}

    Объединим все сказанное, дополнив нашу таблицу соответствующими «мостами» (Таблица 3).

    \begin{table}[ht!]
    \centering
    \caption{Структура развития с «интегральными мостами» между базовыми полярными концептами и «инструментарием» математики и философии неовсеединства}
    \begin{tabular}{|c|c|}
    \hline
    \multicolumn{2}{|c|}{\textbf{Развитие}} \\
    \hline
    \textbf{Переживание времени} & \textbf{Осознавание смыслов} \\
    \hline
    Универкальность & Плерональное число \\
    \hline
    Интерсубъектность & Обратная материя \\
    \hline
    \multicolumn{2}{|c|}{\textbf{Мосты (синтез)}} \\
    \hline
    \multicolumn{2}{|c|}{Субъектность • Многоединство • Финфинитность • Мироподобие} \\
    \hline
    \multicolumn{2}{|c|}{\textbf{Инструментарий}} \\
    \hline
    \multicolumn{2}{|c|}{Поликвантическая математика • Проективно-модальная онтология} \\
    \hline
    \multicolumn{2}{|c|}{Полярный анализ • R-анализ • Мирология} \\
    \hline
    \end{tabular}
    \end{table}

    В центре таблицы – краткий перечень новых методов, с помощью которых решаются задачи синтеза в философии неовсеединства: поликвантическая математика в целом, как способ работать с множеством числовых измерений, не редуцируя их многообразие к одной числовой прямой; проективно-модальная онтология; полярный анализ; R-анализ (релятивистский анализ количества). Более подробно эти средства описаны в работах В.И. Моисеева.

    Кроме того, каждая лаборатория также является связующим «мостом» между базовыми полярностями развития. Ведь смысл интегральной науки не в том, чтобы каждый работал в рамках «своего квадранта» (дифференциация знания в современной науке), но в том, чтобы научиться работать между отраслями знания. По Мартину Буберу «между» – «не вспомогательная конструкция, но истинное место и носитель межчеловеческого события» [Бубер, 1993 (1): 154].

    Предлагаю подумать, какие именно полярности связывает та или иная лаборатория, а также предлагаю свое первоначальное видение задач лабораторий ИИН для разработки инструментария и основных связующих концептов базовых полярностей развития. Буду рада предложениям и замечаниям. Надеюсь, что наша работа действительно поможет рождению интегрального знания.

    \begin{table}[ht!]
    \centering
    \caption{Задачи лабораторий ИИН для разработки инструментария и основных связующих концептов базовых полярностей развития}
    \begin{tabular}{|c|c|}
    \hline
    \multicolumn{2}{|c|}{\textbf{Развитие}} \\
    \hline
    \textbf{Переживание времени} & \textbf{Осознавание смыслов} \\
    \hline
    ЛФС (Лаборатория философии слова) & ЛИМ (Лаборатория интегральной математики) \\
    \hline
    ЛИЭ (Лаборатория интегральной этики) & ЛИФ (Лаборатория интегральной физики) \\
    \hline
    \multicolumn{2}{|c|}{\textbf{Мосты (синтез)}} \\
    \hline
    \multicolumn{2}{|c|}{ЛФИ (Философия истории) • ЛИФТП (Философия права)} \\
    \hline
    \multicolumn{2}{|c|}{ЛМЭ (Музыкальная эстетика) • Все лаборатории в содружестве} \\
    \hline
    \end{tabular}
    \end{table}

    ЛИМ – лаборатория интегральной математики\\
    ЛИФ – лаборатория интегральной физики\\
    ЛИФИ – лаборатория интегральной философии истории\\
    ЛИФТП – лаборатория интегральной философии и теории права\\
    ЛИЭ – лаборатория интегральной этики\\
    ЛМЭ – лаборатория музыкальной эстетики\\
    ЛФС – лаборатория философии слова

    \paragraph*{Заключение}
    Исследование базовых полярностей развития в рамках философии неовсеединства демонстрирует, что развитие – это не линейный прогресс, а многомерный процесс встречи миров, синтеза противоположностей, построения «мостов» между различными измерениями реальности. Предложенная методология, включающая полярный анализ, проективно-модальную онтологию, поликвантическую математику и мирологию, предоставляет эффективные инструменты для понимания и осуществления развития как на индивидуальном, так и на коллективном уровнях.

    Ключевые выводы статьи:
    \begin{enumerate}
        \item Развитие осуществляется через взаимодействие четырех базовых полярностей: углубление–восхождение и переживание–осознание.
        \item Концепты «универкальность» и «интерсубъектность» выражают синтез уникального и универсального, индивидуального и коллективного в переживании развития.
        \item Концепты «плерональное число» и «обратная материя» представляют математические и онтологические основания для осознания развития.
        \item «Мосты» между полярностями – субъектность, многоединство, финфинитность, мироподобие – обеспечивают синтез различных аспектов развития.
        \item Инструменты философии неовсеединства позволяют переинтерпретировать классические философские проблемы (например, категорический императив Канта) в рамках интегрального подхода.
        \item Деятельность лабораторий ИИН может быть организована как система взаимодополняющих «мостов» между различными полярностями развития.
    \end{enumerate}

    Перспективы дальнейших исследований связаны с углублением математических моделей внутренних миров, разработкой интерсубъектной этики, применением мирологии в психологии и теории познания, а также с созданием образовательных программ, формирующих интегральное мышление. Коллективная работа в рамках Института интегральной науки открывает новые возможности для синтеза знания и построения целостной картины мира, в которой развитие понимается как универсальный процесс встречи и взаимодействия миров.

    \begin{center}
        \textbf{Литература}
    \end{center}

    \begin{enumerate}
        \item Бубер М. Проблема человека / Бубер М. Я и Ты. – М.: Высшая школа, 1993. – С. 73-159.
        \item Бубер М. Я и Ты / Бубер М. Я и Ты. – М.: Высшая школа, 1993. – С. 5-72.
        \item Время и звезды: к 100-летию Н.А. Козырева. – СПб.: Нестор-История, 2008. – 790 с.
        \item Моисеев В.И. Мирология: Наука о мироподобных системах. – М.: ЛЕНАНД, 2022. – 600 с.
        \item Моисеев В.И. На пути к обществу развития: Манифест философии неовсеединства // Альманах Единая философия «Пределы мышления и их преодоление»: научно-публицистическое издание. При поддержке Объединенного движения «Русская философия». – № 1, март 2025. – С.9-12.
        \item Моисеев В.И. О пифагорейском числе // XXII Всероссийская научно-практическая конференция «Дни науки – 2022»: Материалы конференции. – Озерск: ОТИ НИЯУ МИФИ, 2022. – С. 270-273.
        \item Моисеев В.И. Образы мироподобного знания // Интегральная философия №12, 2022. – С. 13-36.
        \item Моисеев В.И. Основы R-анализа. – М.: Изд-во «Перо», 2025. – 320 с.
        \item Моисеев В.И. Очерки по философии неовсеединства: Опыт математического прочтения философии. Аксеология. Логика. Феноменология. – М.: ЛЕНАНД, 2018. – 632 с.
        \item Моисеев В.И. Человек и Общество: образы синтеза. Книга первая. – М.: ИД «Навигатор», 2012. – 711 с.
        \item Подзолкова Н.А. В поисках вектора человечности // XXV всероссийская научно-практическая конференция «Дни науки – 2025»: Сборник статей. В 2-х томах. – Озерск: ОТИ НИЯУ МИФИ, 2025. – Т.2. – С.127-131.
        \item Подзолкова Н.А. Апофатическая гносеология // Интегральная философия №1, 2012. – С. 89-93.
        \item Подзолкова Н.А. Мирология как путь к диалогу между мирами // Интегральная философия №12, 2022. – С. 69-98.
        \item Подзолкова Н.А. О соразмерности внутреннего и внешнего как важном принципе биологоса // Философские проблемы биологии и медицины: Феномен биорациональности. Вып. 13. – М.: ЛЕНАНД, 2019. – С. 70-74.
        \item Подзолкова Н.А. Поэтическое слово как время внутреннего мира // XX всероссийская научно-практическая конференция «Дни науки – 2020»: Материалы конференции.– Озерск: ОТИ НИЯУ МИФИ, 2020. – С. 231-235.
        \item Подзолкова Н.А. Универкальность как новая оптика видения универсального и уникального // XXI всероссийская научно-практическая конференция «Дни науки – 2021»: Материалы конференции. – Озерск: ОТИ НИЯУ МИФИ, 2021. – С. 260-264.
        \item Померанц Г.С. Две модели познания / Померанц Г.С. Выход из транса. – М.: Юрист, 1995. – С. 41-53. – (Серия «Российские пропилеи»).
        \item Ровелли К. Срок времени. – М: Издательство АСТ: CORPUS, 2020. – 224 с.
        \item Сатпрем Шри Ауробиндо, или путешествие сознания. – СПб., 2023. – 416 с.
        \item Сент-Экзюпери А. Военный летчик / Сент-Экзюпери А. Избранное. – Челябинск: Юж.-Урал. кн. изд-во, 1980. – 480 с.
        \item Сент-Экзюпери А. Цитадель. – М.: Эксмо, 2024. – 320 с.
        \item Соловьев В.С. София // Логос №4, 1993. – С. 274-296.
        \item Соловьев В.С. Философские начала цельного знания // Соловьев В.С. Соч. в 2 т. – М.: Мысль, 1990. – Т.2. – С. 139-288.
        \item Соловьев В.С. Чтения о Богочеловечестве / Соловьев В.С. Соч. в 2 т. – М.: Правда, 1989. – Т.2. – С. 3-172.
        \item Уилбер К. Проект Атман: Трансперсональный взгляд на человеческое развитие. – М.: ООО «Издательство АСТ» и др, 2004. – 314 с.
    \end{enumerate}
\fi