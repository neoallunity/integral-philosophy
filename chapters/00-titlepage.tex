\thispagestyle{empty}

% ЕЩЁ ЖАРЧЕ: динамика, второй акцент и диагональная фактура
\begin{tikzpicture}[remember picture,overlay]

  % Углы (рамка без полной рамки) + «внутренний» второй слой
  \draw[JournalAccent!70,line width=1.2pt]
    ([xshift=16mm,yshift=-16mm]current page.north west) -- ++(18mm,0) -- ++(0,-12mm);
  \draw[JournalAccent!70,line width=1.2pt]
    ([xshift=-16mm,yshift=-16mm]current page.north east) -- ++(-18mm,0) -- ++(0,-12mm);
  \draw[JournalAccent!70,line width=1.2pt]
    ([xshift=16mm,yshift=16mm]current page.south west) -- ++(18mm,0) -- ++(0,12mm);
  \draw[JournalAccent!70,line width=1.2pt]
    ([xshift=-16mm,yshift=16mm]current page.south east) -- ++(-18mm,0) -- ++(0,12mm);

  \draw[JournalAccent2!45,line width=0.8pt,opacity=0.7]
    ([xshift=20mm,yshift=-20mm]current page.north west) -- ++(14mm,0) -- ++(0,-9mm);
  \draw[JournalAccent2!45,line width=0.8pt,opacity=0.7]
    ([xshift=-20mm,yshift=-20mm]current page.north east) -- ++(-14mm,0) -- ++(0,-9mm);
  \draw[JournalAccent2!45,line width=0.8pt,opacity=0.7]
    ([xshift=20mm,yshift=20mm]current page.south west) -- ++(14mm,0) -- ++(0,9mm);
  \draw[JournalAccent2!45,line width=0.8pt,opacity=0.7]
    ([xshift=-20mm,yshift=20mm]current page.south east) -- ++(-14mm,0) -- ++(0,9mm);

  % Нижняя плашка
  \fill[JournalAccent!10]
    ([xshift=0mm,yshift=0mm]current page.south west) rectangle
    ([xshift=0mm,yshift=\JournalTitleFooterH]current page.south east);

  % Диагональная фактура на нижней плашке
  \JournalIfTexture{%
    \foreach \x in {-1,0.7,...,22} {
      \draw[JournalAccent2!25,line width=0.4pt,opacity=0.6]
        ([xshift=\x cm,yshift=0mm]current page.south west) -- ++(\JournalHatchDx,\JournalTitleFooterH);
    }%
  }

  \draw[JournalAccent!65,line width=\JournalRuleMed]
    ([xshift=\JournalMarginX,yshift=\JournalTitleFooterH]current page.south west) --
    ([xshift=-\JournalMarginX,yshift=\JournalTitleFooterH]current page.south east);
\end{tikzpicture}

\begin{center}
\vspace*{1.1cm}

% Верхний «бейдж»
\setlength{\fboxsep}{7pt}
\noindent\fcolorbox{JournalAccent!60}{white}{%
  \textcolor{JournalAccent!90}{\Large \textsc{Выпуск \journalissue}\;—\;\journalyear}%
}\\[0.35cm]

{\large \textcolor{JournalAccent!85}{\journalurl}}\\[0.85cm]

% Центральная «плашка»-заголовок (слоистая: тень + основная)
\setlength{\fboxsep}{12pt}

% Тень/смещение (без дублирования текста: рисуем только подложку)
\noindent\hspace*{2mm}\fcolorbox{JournalAccent2!35}{JournalAccent2!10}{%
  \begin{minipage}{0.86\textwidth}
    \vspace*{1.45cm}
  \end{minipage}%
}\\[-0.15cm]

% Основная карточка
\noindent\fcolorbox{JournalAccent2!80}{JournalAccent!4}{%
  \begin{minipage}{0.86\textwidth}
    \centering
    {\Huge\bfseries \textcolor{JournalAccent2!92}{ИНТЕГРАЛЬНАЯ}\;ФИЛОСОФИЯ}\\[0.35cm]
    {\large\textsc{ЖУРНАЛ ИНТЕГРАЛЬНОГО СООБЩЕСТВА}}\\[0.35cm]
    {\Large\bfseries \textcolor{JournalAccent!85}{ВЫПУСК \journalissue}}
  \end{minipage}%
}\\[0.9cm]

% Подстрока с боковыми линиями
\noindent\textcolor{JournalAccent2!60}{\rule{0.16\textwidth}{0.6pt}}\hspace{0.03\textwidth}%
{\large\textit{The "Integral Philosophy" periodical\\
the \journalissue-th publication, \journalyear}}\hspace{0.03\textwidth}%
\textcolor{JournalAccent2!60}{\rule{0.16\textwidth}{0.6pt}}

\vfill

% Низ — поверх плашки
{\small © Коллектив авторов, \journalyear}\\
{\small \url{\journalurl}}
\end{center}
\clearpage
