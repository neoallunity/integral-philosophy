\ifprintabstract
    \begin{english}
    % Moiseev V.I. - English abstract
    \subsection{THE INSTITUTE OF INTEGRAL SCIENCE}
    \label{subsec:moiseev-en

    \subsubsection{Moiseev V.I. \\ POSSIBLE RESEARCH AREAS FOR THE INSTITUTE OF INTEGRAL SCIENCE}
    \label{subsubsec:moiseev-title-en

    On 4 January 2025, the public association "Institute of Integral Science" (IIS) was established by unanimous decision within the framework of the Integral Community. The Institute's mission and potential organisational structure, comprising various laboratories, were presented. 

    The following laboratory projects were announced immediately upon the Institute's foundation:
    \begin{itemize}
        \item Laboratory of Integral Mathematics (directed by A.A. Safonov),
        \item Laboratory of Integral Ethics (directed by N.A. Podzolkova),
        \item Laboratory of the Philosophy of Language (directed by E.G. Lugovskaya),
        \item Laboratory of the Philosophy of History (directed by I.A. Zhubrin).
    \end{itemize}

    Subsequently, projects for the Laboratory of Integral Law (directed by O.M. Kholkin), the Laboratory of Integral Music (also directed by O.M. Kholkin), and the Laboratory of Integral Physics (directed by N.A. Smolyanova) were initiated.

    The activities of the laboratories have gradually acquired more definite form. Proposals for projects and work plans have been submitted, implementation of the plans has commenced, and new participants, including representatives from the younger generation, are being actively recruited.

    The Institute of Integral Science is, first and foremost, a scholarly association, wherein a fundamental conceptual framework for the operations of both the Institute as a whole and its constituent laboratories is essential. The principal objective is the construction of the initial contours of integral scientific knowledge, coordinating deductive (from the general to the particular) and inductive (from the particular to the general) trajectories of inquiry.

    The article provides an overview of a number of areas of work of the Institute of Integral Science (IIS), established as a public association by the Integral Community in January 2025. The principles and possible research topics in such areas as integral mathematics, logic and physics, integral chemistry, biomedicine and psychology, integral ethics, sociology and economics, integral history and linguistics, integral law and integral art, integral spirituality are considered. In each direction, the main problems are highlighted and possible ways of further research within the framework of the IIS are outlined.

    \paragraph{Keywords:} \textit{Institute of Integral Science, integral science, integral culture, development}
    \end{english}
\else
    % Моисеев В.И. - полная русская статья
    \subsection{ИНСТИТУТ ИНТЕГРАЛЬНОЙ НАУКИ}
    \label{subsec:moiseev-ru

    \subsubsection{Моисеев В.И. \\ К ВОЗМОЖНЫМ НАПРАВЛЕНИЯМ РАБОТЫ ИНСТИТУТА ИНТЕГРАЛЬНОЙ НАУКИ}
    \label{subsubsec:moiseev-title-ru

    Предлагается обзор ряда направлений работы Института интегральной науки (ИИН), образованного как общественное объединение Интегральным сообществом в январе 2025 г. Рассматриваются принципы и возможные темы исследования таких направлений, как интегральная математика, логика и физика, интегральная химия, биомедицина и психология, интегральная этика, социология и экономика, интегральная история и лингвистика, интегральное право, интегральное искусство и интегральная духовность. В каждом направлении выделяются основные проблемы и намечаются возможные пути дальнейшего исследования в рамках работы ИИН. 

    \paragraph{Ключевые слова:} \textit{Институт интегральной науки, интегральная наука, интегральная культура, развитие}

    \subsubsection*{Введение}
    4 января 2025 г. в рамках Интегрального сообщества единогласным решением было создано общественное объединение «Институт интегральной науки» (ИИН). Представлены миссия Института, возможная его структура в лице различных лабораторий. Сразу же были заявлены проекты следующих лабораторий Института: 
    \begin{itemize}
        \item Интегральной математики (руководитель А.А. Сафонов),
        \item Интегральной этики (руководитель Н.А. Подзолкова),
        \item Философии слова (руководитель Е.Г. Луговская),
        \item Философии истории (руководитель И.А. Жубрин).
    \end{itemize}

    Позднее возникли проекты Лаборатории интегрального права (руководитель О.М. Холкин), Лаборатории интегральной музыки (также руководитель О.М. Холкин) и Лаборатории интегральной физики (руководитель Н.А. Смольянова).

    Постепенно деятельность лабораторий все более оформляется, были предложены проекты и планы их работы, начата реализация планов, активно привлекаются новые участники, в том числе со стороны молодежи.

    В первую очередь Институт интегральной науки – это научное объединение, и здесь важна некоторая базовая концепция деятельности как всего Института в целом, так и его лабораторий. Главная задача – построение первых контуров интегрального научного знания, координируя здесь дедуктивный (от общего к частному) и индуктивный (от частного к общему) пути движения.

    \subsubsection*{Интегральное научное знание}
    Научное знание – это знание, во-первых, структурное, опирающееся на построение математических структур и структурное мышление. Во-вторых, как и во всяком знании, это координация эмпирического и теоретического знания, где первое дает в большей мере многое без единого (факты), а второе – единое без многого (законы). Координация многого и единого формирует образы разного рода теоретических многоединств – синтеза законов и фактов в тех или иных видах теоретического знания.

    В целом, феномен науки есть синтез трех базовых полярностей – многого М, единого Е и структурного С:

    \[
    \text{М} + \text{Е} + \text{С} = \text{МЕС} \quad \text{(структурно выраженное многоединство)}
    \]

    В свою очередь, могут возникать все более интегродифференциальные виды многоединств, все более структурно представленных, – и конца этому движению нет.

    Главная задача ИИН – создание интегральной науки. К сожалению, современная наука разорвана на множество частных дисциплин и направлений, и главный раскол проходит между естественными и гуманитарными науках. Интегральная наука призвана заполнить этот разрыв и создать интегральные образы знания. Каждая лаборатория должна всегда иметь в виду эту главную задачу Института и рассматривать свою деятельность как часть общей цели.

    Ниже я постараюсь, используя идеи философии неовсеединства, представить возможные образы интеграции знания для каждой из лабораторий. Не следует рассматривать их как обязательные предписания, но скорее как некие гипотезы, которые полезно иметь в виду в первую очередь руководителям и сотрудникам лабораторий. Кроме того, попробую затронуть темы не только существующих, но и возможных будущих направлений исследования и лабораторий в рамках ИИН.

    \subsubsection*{1. Интегральная математика и логика}
    Математика – царица наук. Она работает с чистыми структурами, создает/открывает и все более оформляет их. Эти структуры используют все остальные науки.

    В современной математике господствует теория множеств как базовое учение о многообразии, многоединстве. Но подобный подход, кроме несомненных преимуществ, имеет и ряд недостатков, в первую очередь связанных с тем, что множество слабо выражает идею целого, выступая более редукционистским образом многообразия.

    Поэтому одна из главных задач интегральной математики – создание нового учения о многообразии, более холистического толка. Здесь за основу можно взять конструкции Проективно Модальной Онтологии (ПМО), рассматривая в качестве базового состояния многообразия не множество, но модус с некоторым набором его мод.

    На этой основе возникает большая научно-исследовательская программа переинтерпретации всей математики на основе ПМО. В первую очередь центральное понятие структуры должно быть переосмыслено в этом стиле.

    Различные структуры – это теперь не множества, но различные модусы с теми или иными наборами своих мод – модами-элементами, модами-операциями и модами-предикатами. Более того, логические теории со своими языками и логикой, семантика этих теорий, – все это также должно быть представлено как разного рода модус-модальные структуры.

    Модус вместе со своими модами – это более строгое структурное выражение категории многоединства, где полюс единого представлен модусом, его самобытийной частью, а полюс многого – множеством его мод. Таким образом, подводя все структуры под модусы и моды, мы тем самым будем представлять их как разного рода многоединства, выстраивая математику как структурно выраженную теорию многоединства. Тем самым начнет получать свое все большее подкрепление теория интегрального знания как МЕС – структурно выраженного многоединства.

    Следующий важный шаг в построении интегральной математики – развитие и введение новых субъектных структур, субъектных многоединств. Структуры и их теории не висят в ментальном вакууме, но создаются субъектами – отдельными учеными и их сообществами. Поэтому пора ввести субъекта в математику – как некую фундаментальную субъектную структуру, имеющую статус на уровне фундаментальных структур пространства и времени.

    Субъектные структуры уже активно вводятся в математику, начиная с создания математической логики – как структурно выраженного математического мышления. Также в исследовании операций, в теории игр вводятся целевые функции, методы оптимизации и т.д. Так что начало субъектной математики уже положено с конца 19 века. Нужно активно идти далее по этому пути. \textit{Пришло время ввести фундаментальную структуру внутреннего мира и сознания}.

    В философии неовсеединства есть определенные наработки на пути создания этой фундаментальной структуры. Это конструкции субъектных онтологий, строгой феноменологии (так называемые экранные онтологии или Windows-онтологии), обратной топологии, квантовые модели сознания и т.д.

    Наконец, новая математика должна все более становиться рефлексивной. Создание математической логики было первой рефлексией математики над собой, когда она сделала предметом своего исследования свое мышление. Введение субъекта в математику – это продолжение той же рефлексивной линии. И ее нужно все более расширять и развивать. Например, математическое мышление далеко выходит за границы только формальной логики, и все более адекватно выражается сегодня в разных видах неклассических логик. Но по-прежнему большой проблемой является создание диалектической логики как математического направления – математической диалектики. Здесь необходимо дать определение диалектических противоречий, отличить их от формально-логических противоречий (противоречий-ошибок) – дать формулировку \textit{критерия логической демаркации} этих видов противоречий. В философии неовсеединства сделаны первые шаги в этом направлении (теория L-противоречий, объединяюще-ограничивающий механизм разрешения антиномий и т.д.), и нужно здесь идти дальше, выстраивая метаматематику на принципах не только формальной, но и диалектической логики.

    Большое новое направление математики – R-анализ (релятивистский анализ количества), в рамках которого пересматриваются фундаментальные концепты конечного и бесконечного, устанавливается их относительность, зависимость от количественных систем, и предлагается к рассмотрению новый тип инвариант – \textit{финфинитов} (конечно-бесконечных состояний). Предполагается, что R-анализ более адекватно должен выражать те состояния бытия, где конечное и бесконечное более соизмеримо соотносятся между собой, в том числе в разного рода органических онтологиях – онтологиях жизни, сознания и разума. На основе R-анализа также можно пытаться выражать задачу фундаментальной субъектной структуры, понимая ее как \textit{поликвантическое многообразие} – многообразие различных количественных систем, которые могут находиться между собой в разных пропорциях не/соизмерения.

    Важное направление дальнейшего развития интегральной математики и логики – создание математики и логики целого. На сегодня хорошо развита логика общего и частного, но категории целого и части не сводятся к категориям общего и частного и обладают своей спецификой. Представляется, что здесь могут помочь конструкции R-анализа, поскольку эмерджентное качество целого относится к качеству своих элементов как вид бесконечного к конечному (или конечного к бесконечно малому), и целые, возникающие на конечном числе элементов (а таковы все реальные целые), – это вид бесконечности, достигаемый на конечном шаге, что как раз хорошо может быть представлено средствами R-анализа.

    Наконец, с логикой целого тесно связано и учение о новом типе числа, которое я условно называю «пифагорейским». Это число, которое, как можно предполагать, знали пифагорейцы, но позднее его идея была утрачена и заменена современным типом «фаустовского» числа, если следовать здесь терминологии Шпенглера. Пифагорейское число – тип числа, которое достигает бесконечности на конечном шаге, в то время как в фаустовском числе бесконечное достигается только на бесконечном шаге. Учение о пифагорейском числе – важное направление построения интегральной математики, которое сближает математику не только с науками о жизни и сознании, но и с искусством.

    \subsubsection*{2. Интегральная физика}
    Сегодня тема интеграции в современной физике весьма актуальна, и не нужно слишком много доказывать, что синтез и интеграция физического знания насущно необходимы. Таковы задачи великого объединения 4-х фундаментальных взаимодействий, синтез теории относительности и квантовой теории и т.д. Поэтому важное направление исследований в Лаборатории интегральной физики – изучение и более глубокое понимание уже существующих мощных интегративных тенденций современной физики.

    Но этим дело интегральной физики далеко не исчерпывается.

    Сегодня рядом с конвенциональной физикой существует и все более крепнет множество направлений так называемой альтернативной физики. Это разного рода теории эфира, неклассические концепции электромагнетизма, альтернативные системы механики, неклассические версии и интерпретации квантовой механики и т.д.

    Представители конвенциональной физики не хотят идти на диалог со сторонниками альтернативных направлений и обычно относят их к ненаучным и даже лженаучным течениям, что уже говорит об определенном неблагополучии в универсуме физического знания на планете. Часто сторонники альтернативных концепций приводят вполне рациональные аргументы, ставят свои эксперименты и получают интересные результаты, строят оригинальные теории. Почему бы непредвзято не оценить весь этот материал и не пойти с ними на диалог? Ведь вполне возможно, что кроме в самом деле ненаучных подходов или фальсификаций здесь могут найтись и верные достижения, которые выступят в качестве антитезисов к тезисам официальной физики. А там, где тезисы и антитезисы, там и возможность синтезов.

    Поэтому в рамках Лаборатории интегральной физики, как и вообще это должно относиться к деятельности любых лабораторий и Института вообще, следует проводить более разумное отношение к многообразию знания в нашей культуре, проверяя его и отбирая обоснованные результаты, независимо от их принадлежности к официальной парадигме. В связи с этим следует обратить внимание и на множество альтернативных подходов в физике, трезво и непредвзято оценивая их и не боясь использовать имеющиеся здесь положительные результаты, координируя их со структурами конвенциональной физики.

    В рамках «R-физики» [Моисеев, 2012: Т. 1–2] сделана попытка построить теоретический аппарат физики, используя не структуры стандартного математического анализа, но средства R-анализа. Это интересное и перспективное направление, которое можно и нужно активно развивать далее.

    В «Мирологии» [Моисеев, 2022] предложены наброски новой космологической структуры, которая постепенно спускается от Абсолютного через Всемир к множеству малых миров и их онтологиям. Концепт материи предлагается рассмотреть как результат финитного R-сжатия прообразного бесконечного пространства, в связи с чем многообразие R-пространств одновременно представляет многообразие соответствующих видов материальности.

    И конечно же физика должна быть дополнена высшим законом становящегося бытия – \textit{законом развития}. Средствами полярного анализа в философии неовсеединства дано первое определение понятия и закона развития, введена количественная мера развития. Это направление исследований крайне важно, и нужно далее развивать теорию развития. В частности, очень актуальна и важна тема соотношения меры развития и меры энтропии. Последняя также может быть представлена как полярная мера – мера на полярностях, но она определена таким образом, что всегда остается в ортогональном к вектору синтеза подпространстве, так что мера развития для энтропийных процессов в современной физике оказывается всегда постоянной. Это очень интересный результат, который требует специального осмысления и представления в более широком контексте.

    \subsubsection*{3. Интегральная химия}
    Хотя в составе ИИН на сегодня (июль 2025 г.) пока нет Лаборатории интегральной химии, и я, к сожалению, не особенно занимался химической проблематикой, но все же некоторые соприкосновения с этой темой были, и определенные соображения в этой области хотелось бы высказать.

    Химия сегодня добилась больших успехов, сблизилась с физикой через конструкции квантовой теории в лице квантовой химии и, казалось бы, эволюционирует к тому, чтобы стать просто прикладной физикой. Но в химии есть, по крайней мере, один очень важный аспект, пока не вполне ассимилированный физикой и центральный для химического логоса. Это аспект \textit{качественных} состояний вещества и их трансформаций. Вершиной этого подхода стала Периодическая система химических элементов Менделеева, которая задала образец естественной систематики для всех наук. И это первый важный путь развития идей интегральной химии.

    Последовательность химических элементов можно представить как последовательность развития, в которой нарастает мера развития, и в каждом элементе сочетаются как линейный (заряд ядра), так и циклический (степень заполненности внешней электронной орбитали) параметры. Далее над этой спиральной структурой надстраивается производное многообразие молекул. Подобный тип организации, как можно предполагать, далеко выходит за границы только химии и может послужить прообразом для построения спиральных естественных систем в самых разных науках – биологии, психологии, социологии, логике и т.д.

    Аналогично поиску естественной системы химических элементов необходимо искать и формулировать подобные естественные системы в других науках. И они также должны соединять в себе линейные и циклические параметры организации, выражая базовую последовательность качеств, над которой затем надстраиваются производные многообразия. Подобная формула, по-видимому, является универсальной и далеко выходит за пределы химии, но именно последняя является исторически первым образцом для ее выражения.

    Таковы первые контуры интегральной химии как универсальной науки о качествах, их трансформациях и естественных спиральных системах организации.

    Если физика пытается свести качественные аспекты химии к своим количественным определениям, то химия должна удерживать их и развивать более глубокую науку о качествах и мерах. Встреча химии и физики – это встреча качества и количества в более интегральной категории меры. И здесь возникают явные переклички с идеями R-анализа и поликвантической математикой.

    Подобно учению о химических реакциях, можно представлять в будущем создание некоторой более универсальной теории трансформации – своего рода \textit{полярной химии}, где будут взаимодействовать разные полярности на основе степеней своего сродства и дополнительности.

    \subsubsection*{4. Интегральная биология и медицина}
    Также составе ИИН на сегодня пока нет Лабораторий интегральной биологии и интегративной медицины, но автор много занимался этими темами и также может высказать ряд важных соображений по этим направлениям.

    В биологии и медицине существует примерно та же ситуация с отношением конвенционального и альтернативных направлений, что и в физике. Есть много направлений альтернативной биологии (волновой геном, теория биологического поля) и медицины (холистические медицины), которые подчас резко критикуются и отвергаются официальной наукой без достаточно глубокой проверки их с точки зрения расширенной научной методологии, не исключающей холистические подходы. В биомедицине сегодня господствует достаточно жесткий редукционизм, сводящий живое к неживому, биологию – к физико-химии, медицину – к фармакотерапии и хирургии.

    Поэтому главное направление интеграции биомедицинских исследований в ИИН – это сближение холизма и редукционизма в рамках интегрального направления, которое можно называть \textit{холоредукционизм}. Он предполагает, что редукционизм прав в описании нижних – атомно-молекулярных – уровней организации биосистемы, в то время как правда холизма принадлежит более высоким уровням организации – клеточному и выше. Нужна вертикальная межуровневая координация редукционистских и холистических концептов.

    В качестве стартовой модели такой координации в философии неовсеединства предлагается так называемая \textit{физико-информационная модель} биосистемы (ФИМ), которая предполагает, что биосистема определена на 1) высоких информационных уровнях, где выстраиваются интегральные образы реальности и схемы деятельности субъекта, и 2) нижележащих физических уровнях, где эти схемы реализуются в элементарных активностях биохимических процессов. Преимущественная активность биосистемы строится в этом случае сверху вниз – как поток \textit{нисходящей причинности}, когда интегральные схемы активности мгновенно раскладываются на множество все более малых подактивностей. В свою очередь интегральная активность определяется на основе специального вида \textit{органических потенциалов}, которые скоординированы с ценностными структурами биосистемы. Нисходящая координация более интегральных и более дифференциальных активностей реализуется в этом случае как специальный вид \textit{биологической когерентности}.

    Более сложные модели живого представлены в «Мирологии» [Моисеев, 2022], где информационный план рассматривается одновременно как внутренний мир живого существа, который представляет собой \textit{обратномироподобную систему}, реализуемую на специального видах материи – \textit{материи жизни}. На этой основе даются наброски \textit{мироподобной биологии}.

    Подобные же конструкции необходимо использовать и в определениях интегративной медицины, координируя на основе ФИМ и ряда других структур философии неовсеединства идеи западной и восточной медицинских систем. От западной медицины мы должны брать научный метод и ее редукционистские модели. Восточная медицина является богатым источником холистических представлений, но ей не достает научной методологии. Единство холизма, редукционизма и расширенного научного метода является еще одной формой реализации структурного многоединства (МЕС), где редукционизм представляет образы многого, холизм – схемы единого, и научная методология привносит структурные определения.

    Расширенная трактовка научного метода означает, что наука понимается именно как МЕС – координация эмпирически-многого, рационально-единого, и все это должно быть адекватно теоретически-структурно выражено. Причем, материал многого может быть получен в любых реальностях, на основе любых сенсорных модальностей, а схемы единого – на принципах \textit{интегрального рационализма}, который объединяет средства классической и неклассической рациональности, и структурность должна быть достаточно гибкой, чтобы адекватно все это выразить.

    \subsubsection*{5. Интегральная психология}
    В современной психологии, как это верно отмечает Л.С.Выготский в своей работе «Мышление и речь» [Выготский, 2023: 342-343], психика разделена на множество психических функций (восприятие, память, мышление и т.д.) и практически исчезла как самостоятельная целостность. Существует множество психологических направлений и школ, каждое из которых рассматривает свои психические функции, но психика в целом во многом выпадает из внимания современных психологов. Поэтому тема интегральной психологии, как и в других научных дисциплинах, – это в первую очередь тема оценки и интеграции множества частных направлений, уже имеющихся в этой науке.

    Здесь мы встречаем два основных класса концепций – это учения преимущественно бихевиористского толка и гуманитарно-ориентированные направления. Первые представляют психологический редукционизм, сводящий психику к поведению и его материальным носителям, вторые выражают психологический холизм, подчеркивающий момент автономности психического бытия и его несводимость к чему-либо иному. Как и в биологии, здесь возникает тема синтеза в лице \textit{психологического холоредукционизма}.

    Хотя интегральных подходов в современной психологии немного, но некоторые все же присутствуют, и на них нужно опираться при проведении интеграции в психологическом знании. В первую очередь можно выделить подход Карла Юнга и выросшую затем из него трансперсональную психологию, оперирующую понятиями базового (БСС) и измененных (ИСС) состояний сознания. Центром интеграции всех состояния сознания, согласно Юнгу, выступает самость. Развитие личности – интеграция всех психических проявлений самости, что явно коррелирует с конструкциями полярного анализа, где более строго дается определение развития.

    Тема развития была центральной в культурно-исторической школе Л.С.Выготского, П.Я.Гальперина, А.Н.Леонтьева, которая сегодня, к сожалению, незаслуженно находится на периферии внимания отечественных психологов. Необходим возврат идей этой школы в современные психологические исследования и координация концептов развития, наработанных в этом и ряде других направлений (генетическая психология Пиаже и др.) с конструкциями полярного анализа [Моисеев, Коломиец, 2025].

    В центре психологического интегрального логоса, как представляется, необходимо ולה ставить модель субъектных онтологий, обогащая ее все большим числом дополнительных структурных средств – экранными онтологиями, обратными топологиями, мироподобными моделями и т.д. Уже сегодня можно говорить о создании более строгих разделов теории чувств, субъектных зарядов, построенных на основе этой модели.

    \subsubsection*{6. Интегральная этика}
    Идеи интегральной этики в последнее время получили и свое теоретическое выражение, и определенное прикладное воплощение в лице нового учебника и курса по биомедицинской этике [Моисеев, Моисеева, 2021: Т. 1–2].

    Главная идея интегральной этики – положить в основу этических представлений теорию развития, т.е. рассматривать добро как деяние разумных субъектов с максимизацией меры развития в конкретном контексте совершения деяния. В полярном анализе, как уже отмечалось, мы имеем стартовые конструкции теории развития, которые можно использовать в этике. Центральный момент приложения первой ко второй в этом случае – это построение полярных портретов тех ситуаций, с которых начинаются или которыми заканчиваются те или иные деяния субъектов. Построив полярные портреты в лице текущих полярных векторов в полярных векторных пространствах, мы затем хотя бы полуколичественно можем провести оценку их меры развития и ее изменения в процессе деяния, выбрав то из них, которое дает максимальный прирост меры развития в данном контексте. Такое максимизирующее деяние и должно выступить гипотезой объективного добра в данной ситуации. Но конечно, как и во всяком научном познании, это должна быть только гипотеза, которая может подвергаться дальнейшим проверкам и критике.

    Также важно отметить, что для развития интегральной этики все более важным будет развитие настоящей глубокой теории развития, которая будет позволять все более точно и глубоко определять меры развития и динамику этих мер в тех или иных конкретных онтологиях. Впрочем, развитие этой теории имеет важнейшее значение для всей интегральной науки в целом.

    На основе интегральной этики как прикладной теории развития органично возникают возможности интеграции различных более частных этических направлений, в первую очередь утилитаризма и деонтологического подхода. Дается новая формулировка категорического императива как высшего нравственного закона, а именно – \textit{закона добра} как стратегии максимизации развития со стороны свободных разумных субъектов.

    Наряду с законом добра важную роль играет также \textit{закон следствий}, согласно которому субъект, изменяя бытие на некоторую меру развития, меняет свою меру развития на пропорциональную этому величину. Отсюда логично вытекает идея смысла жизни как стратегии развития – хочешь развиваться, развивай бытие вокруг себя.

    Важно подчеркнуть, что представления интегральной этики, как они вкратце описаны выше, предполагают \textit{субъектную объективность} нравственности, поскольку связывают добро и развитие, а закон развития является универсальным объективным законом всего становящегося бытия. И здесь важной темой оказывается координация не только этики и конструкций полярного анализа, но этики и структур субъектных онтологий. В частности, уже проведена большая работа по представлению этических определений в рамках теории субъектных зарядов на материале проекта киноуроков.

    \subsubsection*{7. Интегральная социология и экономика}
    Опираясь на модели субъектных онтологий, необходимо пересмотреть и интегрировать материал общественных наук, в первую очередь – социологии и экономики.

    Здесь также можно начинать с обзора и интеграции уже имеющихся частных подходов и направлений, например, микро- и макросоциальных теорий, экономических теорий более либерального и консервативного толка. Отчасти такая работа в отношении социологических концепций проделана в монографии «Человек и общество: образы синтеза» [8,9].

    В конечном итоге центральной идеей общественных наук должна быть модель коллективного субъекта, организованного на индивидуальных разумных субъектах. Между коллективным и индивидуальными субъектами в идеале должен существовать справедливый обмен ценностями, когда индивидуальный субъект оказывается подсубъектом коллективного субъекта (социума), выполняя некоторую часть общественной деятельности, и за это он получает право на пользование соответствующей меры создаваемого коллективным субъектом общественного блага. И здесь мы видим центральный процесс социальной динамики, включающий в себя в первую очередь движение ценностей, т.е. процессы 1) создания ценностей, 2) их распределения и 3) их ассимиляции (потребления). Те или иные формы социально-экономического устройства в конечном итоге восходят к принципам и видам организации социально-ценностной динамики. Главный вопрос, который здесь возникает, – позволяет ли существующая социальная динамика развиваться обществу и индивиду, и какая из всех возможных динамик наиболее оптимальна для развития?

    На этой основе можно предложить новый взгляд на причинные факторы социальной динамики, выделяя три основных вида субъектов: 
    \begin{itemize}
        \item \textit{Р-субъекты} («прогрессоры» в терминологии Стругацких), объективно развивающие бытие, 
        \item \textit{N-субъекты}, объективно его деградирующие, и
        \item промежуточные \textit{М-субъекты}, деятельность которых особенно не влияет на меру развития.
    \end{itemize}

    Множество соответствующих типов субъектов в социуме образует свои страты: \textit{Р-страт}, \textit{N-страт} и \textit{М-страт} соответственно. Именно эти страты являются главными причинными факторами социальной динамики, определяя, будет ли она развиваться, деградировать или топтаться на месте.

    Главная задача любого типа общества в этом случае – создать такую социальную систему, которая поддерживала бы Р-субъектов, боролась с N-субъектов и способствовала бы трансформации всех субъектов в тип Р-субъектов. Такой вид общества пока условно можно называть \textit{Р-социумом} (\textit{обществом развития}). Имеющийся в настоящее время тип социума, к сожалению, далек от этого идеала и представляет собой скорее то, что можно назвать \textit{N-матрицей}: официально проповедуются идеалы добра и развития, а реально во многом господствует N-страт и его элита.

    Затрагивая вкратце идеи экономики, следует отметить, что она ставит перед собой задачу понимания и выработки наиболее оптимальных принципов социальной динамики. На сегодня ценностная регуляция социума реализуется на основе института денег.

    Деньги – это достаточно специфическая мера ценности, которая, во-первых, редуцирует ценность до стоимости как материально выраженной обменной ценности, и, во-вторых, отождествляет меру ценности с признанной материальной знаковой формой выражения этой меры – вначале материальными объектами (некоторым продуктом, золотом и т.д.), а затем со все более символическими знаковыми формами (бумажные и цифровые деньги). Возникают специфические механизмы оперирования со знаковыми формами денег, которые выступают как своеобразные виды игры и далеки от нравственных определений. Наиболее ловкими в этих играх оказываются обычно не Р-субъекты. Также в рамках N-матрицы не существует достаточного нравственного механизма контроля за условиями получения знаковых материальных форм денег («деньги не пахнут»), в связи с чем большим количеством денег как правило обладают N-субъекты, что значительно деформирует существующую социально-экономическую систему.

    Поэтому важнейшая задача интегральной экономики – разработка нравственного механизма ценностной динамики, которая будет стимулировать не деградацию, а развитие общества, человека и природы.

    \subsubsection*{8. Интегральная история}
    История есть процесс эволюции общества во времени, и такого рода эволюция далеко не всегда является развитием, но в общем случае может сочетать в себе периоды как развития, так деградации и застоя. В то же время идеалом социальной динамики конечно же является развитие общества, культуры и цивилизации.

    Задачи интегральной истории в связи с этим можно представить двояко: во-первых, восстановление реальной картины истории, не зависящей от тех или иных политических и идеологических предпочтений, и, во-вторых, обработка этого материала с точки зрения построения полярных портретов реальных исторических событий и оценки мер развития и их динамики.

    В связи с этим следует учитывать, что современное представление истории сильно искажено относительно реальных событий прошлого, и в системе многообразного исторического знания мы также сталкиваемся с господствующими конвенциональными концепциями и множеством альтернативных подходов. Как в области физики, биомедицинских наук и других сфер научного знания, где встречается подобная ситуация, в интегральной истории необходимо непредвзятое исследование всех концепций и подходов и попытка интеграции их не по принципу конвенциональности, но по степени обоснованности и ответа на разного рода исторические контрпримеры.

    В качестве главных причин исторических процессов, как это было представлено выше, следует рассматривать Р-, N- и М-страты, их элиты и отдельных лидеров, определяя реальный исторический процесс как арену борьбы соответствующих типов субъектов – коллективных и индивидуальных.

    Интегральная история должна адекватно отвечать на вызовы множества борющихся направлений исторической науки и внутри конвенциональных подходов, имея в виду в первую очередь направления формационного и цивилизационного подходов в философии и теории истории. Здесь нужно некое интегральное решение, которое способно их координировать, отводя каждому направлению свое место в рамках целостной концепции. Например, в качестве возможного кандидата на такого рода синтез мог бы выступить \textit{многоуровневый подход} в понимании исторического процесса, где на более высоких уровнях организации реализуются структуры исторического процесса, близкие к формационной динамике, а на более локальных – близкие к цивилизационным определениям исторического процесса.

    Исходя из центральной роли Р- и N-стратов в причинных определениях исторического процесса, представляется важным введение понятия Р- и N-социумов и рассмотрение реального социума как некоторой подвижной пропорции этих пределов. В этом случае центральной линией истории будет борьба этих двух социумов, их элит и стратов как внутри отдельных социокультурных целостностей (наций, народов, государств, цивилизаций), так и в рамках всего человечества в целом. В этом случае становится очень понятной сложная динамика реального исторического процесса, которую пока нельзя подвести под какой-то один идеальный тип.

    Наконец, интегральный историк должен был бы интересоваться не только прошлым, но и будущим человеческой цивилизации, исходя из интегральных моделей истории. В частности, должен будет так или иначе решаться вопрос о смысле и назначении истории – как в конечном итоге процессе победы Р-страта и вывода человеческой цивилизации на пути рождения и формирования общества развития (Р-социума). В этом случае темы интегральной социологии и истории тесно переплетаются.

    \subsubsection*{9. Интегральная лингвистика}
    Язык – сложная семиотическая система, и в области наук о языке также существует множество направлений, исследование и возможности координации которых, – важная задача интегральной лингвистики.

    Как известно, синтаксис, семантика и прагматика – три основных аспекта бытия языковых систем. Синтаксис – это законы построения знаковых форм языковых выражений, и здесь, возможно, важную роль играют конструкции ПМО, когда центральной языковой структурой выступает отношение источника предикации (модуса) и его предикации (моды), и виды предицирования могут быть самые различные, более или менее сложно сочетаясь между собой.

    В области семантики мы имеем центральное отношение знака и его содержания, и здесь важную роль играет исследование и структуризация пространства образов и смыслов, а также их отношения со своими знаковыми выражениями.

    Наконец, вопросы прагматики предполагают введение субъектов, которые используют язык как средство для своей субъектной деятельности (коммуникации, рефлексии и т.д.).

    Отсюда видна структура более интегральных субъектных онтологий языка – это коллективный субъект со множеством своих подсубъектов, далее это образно-смысловые многообразия, находящиеся во внутренних мирах коллективных и индивидуальных субъектов, и это наконец проективно-модальные и иные знаковые формы, посредством которых происходит выражение смысло-образной семантики. Обычные языковые структуры предполагают отсутствие телепатии, т.е. невозможность прямой связи внутренних миров разных субъектов, в связи с чем приходится использовать материальные знаковые формы, которые интерсубъектны и могут на этой основе передаваться от одного внутреннего мира к другому. В рамках интегральной лингвистики нужно предусмотреть и возможности исследования прямого взаимодействия внутренних миров субъектов, которое в этом случае также может использовать некоторые интерсубъектные структуры иного типа.

    Подобная возможность подводит нас к более глубоким разделам интегральной лингвистики, затрагивающим тему фундаментального кодирования бытия – тему своего рода \textit{онтологического кода} (\textit{онто-кода}). Последний представляет собой наиболее глубокую система кодирования бытия на уровне инвариант высшего порядка, которые способны связывать между собой разные типы онтологий – онтологий разных внутренних и внешних миров. Отсюда вытекает, что онто-код одновременно играет роль психофизического кода, в частности, связывая между собой внутренний и внешние миры. В рамках философии неовсеединства уже были сделаны попытки выражения структур онто-кода средствами пифагорейского числа, конструкциями полярной динамики и проективно-модальными структурами. Подобные исследования необходимо развивать и далее.

    \subsubsection*{10. Интегральное право}
    Право, как и этика, – это наука о законах социальной жизни свободных субъектов, в связи с чем такие законы не должны отрицать свободу, но предполагают ее, выступая как законы долженствования, а не необходимости. Так же как в интегральной этике, в основе интегрального права в конечном итоге должен лежать закон развития.

    Здесь вообще стоит отметить, что разного рода нормирующие системы до сих пор строились как наборы множества норм, каждая из которых условна и имеет свой интервал применимости, вне которого она начинает встречать разного рода контрпримеры. Применение норм к конкретной ситуации часто приводит к конфликту разных норм, когда на основе ни одной из норм нет возможности полностью определить должное состояние в данной ситуации и приходится либо редуцировать решение по поводу данной ситуации к какой-то норме, либо искать некоторые теоретически непредсказуемые решения. Такого рода ситуации распространены в праве и других нормативных видах деятельности повсеместно и отражают дискретную структура нормирования континуальной человеческой жизни. Пытаясь приблизить непрерывное дискретным, приходится дискретировать непрерывное, что с неизбежностью приводит к неадекватности. В связи с этим нужна некоторая \textit{континуальная нормативность}, которая бы выходила за границы дискретных норм, в то же время включая их в себя как свои частные случаи в рамках их интервалов адекватности.

    В качестве такой континуальной стратегии нормативности в рамках философии неовсеединства предлагается использование закона развития как высшего нормативного регулятора социальной жизнедеятельности. В отличие от дискретных норм, закон развития обладает возможностью своего приложения в любых ситуациях с сохранением своего высшего инвариантного принципа. Конкретное применение закона развития к данной ситуации включает в себя не только качественные моменты частных норм, но и количественную оценку возможных изменений мер развития в каждом возможном деянии в данном контексте, как это было описано в разделе по интегральной этике. В том числе закон развития содержит в себе момент выхода за границы всех норм (момент анормативности как гипернормативности), если ни одна из них не оказывается вполне адекватной в данной ситуации.

    В связи с этим, в области интегрального права, как и в области интегральной этики, важной является задача переосмысления и перестройки всей системы норм и правил с точки зрения закона развития, создание своего рода \textit{права развития} или \textit{развивающего права}.

    \subsubsection*{11. Интегральное искусство}
    Искусство есть теория и практика развития человека через его чувственность. Отсюда возникают как преимущества искусства перед наукой (более прямой путь воздействия на человека), так и его слабости (частое противопоставление практики искусства его теории и недостаточная развитость последней). В рамках интегрального искусства мы также, как и во всех остальных областях интегральной культуры, должны понимать, что в конечном итоге главная задача бытия человека и общества – это развитие, и этой задаче в том числе должно служить и искусство, будучи \textit{искусством развития} человека, а не его деградации, как это происходит часто в последнее время.

    Главные категории искусства – красота и гармония. Попытка выразить эти категории более структурно приводит нас к конструкциям полярного анализа, когда гармония понимается как завершенное состояние некоторой системы полярностей, т.е. состояние их синтеза. На первый план в определениях такой завершенности и полноты выходит циклический параметр развития – угол бытия.

    В то же время система полярностей тем более эстетична, чем более она не только завершена, но и сложна. В итоге \textit{красота есть своего рода произведение гармонии (завершенности) на сложность}. В статических видах искусства (живопись, скульптура, архитектура) система полярностей дана в статике, в своем законченном состоянии, в то время как в динамических видах искусства (литература, музыка) система полярностей постепенно становится, двигаясь к своему финальному – наиболее сложному и завершенному – состоянию.

    Отсюда следует, что центральными теоретическими конструкциями теории искусства являются полярный анализ и пифагорейское число, которое в свою очередь предполагает структуры R-анализа. Конечно, очень важны в определениях искусства структуры ПМО и субъектных онтологий (теория чувств). Первоочередные задачи интегрального искусства – построение полярных портретов тех или иных произведений искусства и показ этих произведений как либо достаточно развитого финального состояния с высокой мерой развития в своей системе полярностей (для статических видов искусства), либо реконструкция полярных ритмов как ритмов развития для динамических видов искусства. Особенно сложна, но и важна эта задача в теории музыки как одном из самых сложных видов динамического искусства.

    Решение таких задач приведет нас к сближению интегрального искусства и интегральной математики, поскольку полярные структуры искусства окажутся в этом случае структурами поликвантического многообразия в новой математике.

    Двигаясь от Лаборатории интегральной математики к Лаборатории интегрального искусства, мы приближаемся к завершению одного большого витка спирали развития.

    \subsubsection*{12. Интегральная духовность}
    Каково отношение интегральной науки к религии? 

    Выскажу свою точку зрения.

    Как представляется, в религии есть два момента: 1) работа с духовной реальностью (духовными онтологиями), в том числе через активацию и использование разного рода измененных состояний сознания (молитва, медитация), 2) склонность умалять рациональный компонент человека и выводить на первый план слепое доверие к разного рода авторитетам, что в конечном итоге ведет к отказу от использования научной методологии, понимаемой в широком смысле – как МЕС (построение и использование структурных многоединств).

    Принимая первую компоненту, интегральная наука не может принять вторую. С этой точки зрения, религия есть духовная пред-наука, которая пытается взаимодействовать с духовной сферой бытия на основе подавления разума.

    В качестве оправдания такого подавления обычно указывают на узость человеческого ума, блокировку с его стороны высших человеческих способностей. В этом есть момент правды, но только момент. Восходя от ума к разуму как высшей рациональной способности личности, можно выразиться более точно: подавляет духовность низший ум, а не разум, в то время как разум не только не мешает, но выступает необходимым условием подлинной духовности. Разум в данном случае мы понимаем как способность структурировать любые многоединства бытия, не абсолютизируя ни один из вариантов такой активности.

    Поддерживает позицию противопоставления веры и разума и образ современной материалистической науки, которая также ограничивает науку только сферой физической материи. В этом сходятся современная наука и религия – они одинаково отрицают научный подход к духовной сфере бытия. Первая считает, что здесь перестает работать научный метод, который может быть привязан только к физической материи; вторая вообще полагает глубокое научное проникновение в область духа преступлением («ересью»).

    Интегральная наука стоит на иной точке зрения, предполагая возможность научной методологии в широком смысле, которая может адекватно познавать не только область материи, но и духа, и здесь не только нельзя выключать разум, но следует активировать его в еще большей степени. Правда, и разум здесь должен быть особый – не тот, который привык работать только с физической материей. Поэтому необходимо создание особой духовной науки – науки о духе и духовных онтологиях как особом виде субъектных онтологий.

    Важным разделом духовного направления интегральной науки является учение об Абсолютном как высшем виде бытия, своего рода \textit{онтологической бесконечности}. Как математика в свое время сделала большой рывок вперед, начав оперировать с понятием бесконечности – вначале потенциальной, затем актуальной, так и наука нового типа совершит прорыв, используя новый тип логоса о высшем типе бытия как онтологической бесконечности.

    С нашей точки зрения, более адекватным выражением теории Абсолютного является создание научной теории антиномий, поскольку абсолютное бытие проецируется в любой относительный конечный разум как антиномическое состояние. На данный момент, более глубоким и адекватным выражением логики Абсолютного выступает теория L-противоречий – высший раздел теории антиномий. Пока наиболее полно эта логика представлена в «Логике открытого синтеза» [Моисеев, 2012]. Также важным в создании духовной науки являются структуры R-анализа, где может быть выражено учение о финфинтном – новом типе инвариант, объединяющем структуры конечного и бесконечного. Частным случаем такого рода инвариантности является отношение Абсолютного и относительного.

    Главное, чтобы человек понимал, что духовная реальность – так же один из типов бытия, где есть свои законы и своя структура. Эта реальность населена множеством существ, которые не обязательно все обладают светлой направленностью, и нужно быть вдвойне осторожным в отношении с этим типом бытия и существами, его населяющими. Как и везде, необходимо познание духовных законов и умелое им следование в сложном духовном бытии. Простые и радикальные формулы здесь могут только осложнить ситуацию.

    \subsubsection*{Заключение}
    Завершая этот краткий обзор, мне хотелось бы заметить, что конечно все предложенные размышления о возможных направлениях и принципах Института интегральной науки далеко не покрывают всей полноты его оснований и задач и во многом являются результатом конкретной творческой эволюции автора. Но все же хочется надеяться, что предложенные выше варианты содержат в себе свой момент истинности, и от них вполне можно отталкиваться в дальнейших исследованиях ИИН.

    Путь интегральной науки только начинается. Перед нами стоит сложнейшая задача – не просто суммировать уже существующие знания, но выработать принципиально новые подходы к организации знания, позволяющие преодолеть разрывы между дисциплинами, восстановить целостность понимания мира и человека. Это требует как теоретической работы по разработке новых методологических инструментов (ПМО, полярного анализа, R-анализа, мирологии), так и практических усилий по созданию сообщества исследователей, способных мыслить и работать в интегральной парадигме.

    Важно понимать, что интегральная наука – это не просто очередная научная мода или специализация. Это попытка ответить на глубокий эпистемологический кризис современности, на фрагментацию знания, которая ведет к фрагментации самого человеческого бытия. Интегральная наука стремится не к унификации, а к гармоничному единству в многообразии, не к редукции сложного к простому, а к пониманию сложности как таковой.

    Работа, начатая в рамках ИИН, – это лишь первые шаги на долгом пути. Но важно, что эти шаги делаются в верном направлении – направлении восстановления целостности знания и бытия.
    
    \subsubsection*{Литература}
    \printbibliography[heading=none]
\fi