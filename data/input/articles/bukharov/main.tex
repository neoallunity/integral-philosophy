\ifprintabstract
    \begin{english}
    % Bukharov Yu. D. - English abstract
    \subsection{\texorpdfstring{\textbf{Bukharov Yu. D. TRANSCENDENTAL-ONTOLOGICAL UNITY OF THE INNER WORLDS}}{Bukharov Yu. D. TRANSCENDENTAL-ONTOLOGICAL UNITY OF THE INNER WORLDS}}
    \label{subsec:bukharov-en

    This article aims to outline some possible directions in solving the problem of the unity of human inner worlds and their commensurability with the external world.

    The present study draws upon the research of N.A. Podzolkova, as well as the categorical apparatus of classical metaphysics and the theories of projective-modal ontology, worldology, and R-analysis developed by V.I. Moiseyev from the perspective of the philosophy of allunity and neoallunity. In particular, the approach of "strong predicatism" is employed, which assumes the existence of universals in general (and transcendentals as ontologically extremely universal, in particular) as "strong predicates"—that is, predicates with high ontological invariance, which enhances their being, but at the same time does not turn them into some new entities. In addition, strong predicates themselves are not directly predicated by individual entities, but are only articulated through their particular manifestations. In other words, strong predicatism presupposes the criterion of reality as invariance: the more invariant something is, the more real it is.

    At the same time, the question is posed: what are the extremely elementary entities in relation to which transcendental predication is possible in one way or another? From this point of view, a logical-ontological explication of Leibniz's monadology and the theory of transcendentals of European mediaeval thinkers is carried out, on the basis of which the conclusion is drawn that along with the subjective concepts-transcendentals of metaphysics and intersubjective universals-transcendentals of culture, their denotations also objectively subsist—actual ontological transcendentals as extremely universal ontological aspects of ontically existing things in themselves. A hypothesis is formulated that transcendentals form an ontological hierarchy of types, which includes transcendental attributes—integrative transcendentals (All-Unity, Meaning), transcendental properties—fully reversible transcendentals (Existent, One, Good, Truth, Beauty, Love, Being, Thing, Something, ...), transcendental states—partially reversible transcendentals (Essence, Number, Harmony, Logos, Entelechy, Freedom, Hypostasis, Wisdom, ...), transcendental relations—disjunctive transcendentals (matter and form, content and structure, cause and effect, singular and general, possibility and reality, necessity and chance, ...), transcendental connections, which are also elements of disjunctive transcendentals (matter, form, content, structure, cause, effect, singular, general, possibility, reality, necessity, chance, ...). The ellipsis in the lists of fully reversible, partially reversible, and disjunctive transcendentals indicates that the set of known ones is not complete, and that it itself is ontologically akin to the innumerability of attributes postulated by Spinoza.

    The main conclusion is that the unity of human inner worlds and their commensurability with the external world has both ontic (object-immanent) and ontological (object-transcendental) foundations, where transcendentals play a special role—both as extremely universal ontological invariants (extremely strong predicates) of everything that exists, and as universals of society and culture created by the combined human practice. Of particular importance in this case is the temporal problematic, which has a long metaphysical tradition, but requires new understanding.

    \paragraph{Keywords:} {\itshape Spiritual, ideal, transcendentals, monads, inner worlds, worldology, R-analysis, virtual determination, ontological coordination}
    \end{english}
\else
\else
    \begin{russian}
    % Бухаров Ю.Д. - русская статья
    \subsection{\texorpdfstring{\textbf{Бухаров Ю. Д. ТРАНСЦЕНДЕНТАЛЬНО-ОНТОЛОГИЧЕСКОЕ ЕДИНСТВО ВНУТРЕННИХ МИРОВ}}{Бухаров Ю. Д. ТРАНСЦЕНДЕНТАЛЬНО-ОНТОЛОГИЧЕСКОЕ ЕДИНСТВО ВНУТРЕННИХ МИРОВ}}
    \label{subsec:bukharov-ru

    \textit{Цель статьи состоит в том, чтобы обозначить некоторые возможные направления в решении проблемы единства человеческих внутренних миров и их соразмеримости с внешним миром. При решении этой проблемы используются исследования Н. А. Подзолковой, а также категориальный аппарат 
    классической метафизики и разрабатываемые В. И. Моисеевым в ракурсе философии всеединства и неовсеединства теории проективно-модальной онтологии, мирологии и R - анализа. }
    \textit{В частности, используется подход «сильного предикатизма», который предполагает существование универсалий вообще (и трансценденталий как онтологически предельно универсальных, в частности) в качестве «сильных предикатов» – т. е. предикатов, обладающих высокой онтологической
     инвариантностью, которая усиливает их бытие, но вместе с тем не превращает их в некие новые сущие. Кроме того, сильные предикаты сами непосредственно не предицируются отдельным сущим, но лишь экспрессируются через частные свои проявления. Иными словами, сильнопредикатизм предпола
    гает критерий реальности как инвариантности: чем более нечто инвариантно, тем более оно реально. }
    \textit{При этом ставится вопрос: каковы те предельно элементарные сущие, относительно которых так или иначе возможна трансцендентальная предикация? С этой точки зрения осуществляется логико-онтологическая экспликация монадологии Лейбница и теории трансценденталий европейских сред
    невековых мыслителей, на основании чего делается вывод, что наряду с субъектными понятиями-трансценденталиями метафизики и интерсубъектными универсалиями-трансценденталиями культуры объектно наличествуют и их денотаты – действительные онтологические трансценденталии как предельно-
    универсальные онтологические аспекты онтически сущих вещей самих по себе. }
    \textit{Формулируется гипотеза о том, что трансценденталии образуют онтологическую иерархию типов, включающую в себя трансцендентальные атрибуты – интегративные трансценденталии (Всеединство, Смысл), трансцендентальные свойства – полностью обратимые трансценденталии (Сущее, Единое
    , Благо, Истина, Красота, Любовь, Бытие, Вещь, Нечто, ...), трансцендентальные состояния – частично-обратимые трансценденталии (Сущность, Число, Гармония, Логос, Энтелехия, Свобода, Ипостась, Премудрость, ...), трансцендентальные отношения – дизъюнктивные трансценденталии (материя
     и форма, содержание и структура, причина и следствие, единичное и общее, возможность и действительность, необходимость и случайность, ...), трансцендентальные связи, они же элементы дизъюнктивных трансценденталий (материя, форма, содержание, структура, причина, следствие, единичн
    ое, общее, возможность, действительность, необходимость, случайность, ...). Многоточие в перечнях полностью обратимых, частично-обратимых и дизъюнктивных трансценденталий указывает на то, что множество известных из них не завершено, а само оно онтологически сродни постулированной 
    Спинозой бесчисленности атрибутов. }
    \textit{Главный же вывод состоит в том, что единство человеческих внутренних миров и их соразмерилось с внешним миром имеет как онтические (объектно-имманентные), так и онтологические (объектно-трансцендентальные) основания, где особую роль играют трансценденталии – и как предельн
    о универсальные онтологические инварианты (предельно сильные предикаты) всего сущего, и как творимые совокупной человеческой практикой универсалии социума и культуры. Особое значение при этом имеет темпоральная проблематика, имеющая долгую метафизическую традицию, но нуждающаяся в
     новом осмыслении.}

    \paragraph{Ключевые слова:} {\itshape Духовное, идеальное, трансценденталии, монады, внутренние миры, мирология, R-анализ, виртуальная детерминация, онтологическая координация}

    \textit{Индивиды как таковые суть безусловно отдельные замкнутые в себе миры, совершенно между собой не связанные.}
    {\raggedleft И.Г. Фихте «Факты сознания» [Фихте, 1993: 685] \par}

    В противоположность обычному воззрению «я есмь» отнюдь не есть первичная, адекватная и всеобъемлющая форма «внутреннего бытия», непосредственного самобытия, а может быть признано лишь частным и производным моментом более глубокого и первичного откровения реальности в форме бытия «
    мы». 
    {\raggedleft С.Л. Франк «Непостижимое»  [Франк, 1990: 379] \par}

    \textbf{1. Постановка проблемы}
    Проблема сопряжения, взаимопроникновения и взаимодействия внутренних миров различных людей и соразмерения внутренних человеческих миров с внешним миром имеет давнюю историю – настолько давнюю, что ныне в научном ракурсе проблема эта представляется в основном решенной. 
    Действительно, последние десятилетия (а в целом последние два столетия) ознаменованы в том числе существенными успехами социологии, общей и социальной психологии, нейрофизиологии, культурологии, теории коммуникации и т. п. Давно уже стали едва ли не аксиоматическими положения о то
    м, что человек – существо социальное, что производство человеческой жизни с необходимостью включает в себя производство материальных общественных отношений и идеальных форм общения, что все люди принадлежат к одному и тому же биологическому виду Homo sapiens sapiens, а в глубинах 
    человеческой психики наличествуют общечеловеческие архетипы, составляющие наиболее фундаментальную основу для возможного взаимопонимания представителей самых разных социальных групп и этносов. 
    Но все это – на уровне онтическом, относящемся к порядку сущего, а не на уровне онтологическом – не на уровне того, что относится к порядку бытия в его отличии от сущего. В то время как задача восстановления соразмерности внешнего и внутреннего измерений человеческого бытия (см. [
    Подзолкова, 2025]) актуальна прежде всего в своем онтологическом аспекте, без которого не может быть должным образом осмыслен и аспект онтический. Ибо наиболее фундаментальные вопросы здесь: «\textit{возможно ли общение между живыми существами, то есть существами, обладающими внутренними мирами, на уровне самих этих внутренних миров? Возможно ли пересечение внутренних миров? Не является ли совокупность внутренних миров всего живого тоже своеобразным многоединым космосом – континуумом внутреннего пространства и внутреннего времени?}
    » [Подзолкова, 2020: 50].
     В этой связи есть смысл рассмотреть поставленную задачу с точки зрения мирологии в ракурсе трансцендентализма – имея в виду не тот, что берет свой исток от Канта, а в плане теории трансценденталий классической метафизики.
    \textbf{2. От кантовского трансцендентализма к онтологии трансценденталий}
    Трудно не согласиться, что «\textit{в отечественном философском пространстве учение о трансценденталиях не получило достаточного внимания, а между тем именно оно определяло интеллектуальный климат средневековой культуры, которую Ян Арцен удачно назвал «трансцендентальным мышлением
    ». К сожалению, многие историки философии зачастую лишь понаслышке знакомы с этим учением, как правило путая его с проблемой универсалий. Причина этого заключается в том, что трансцендентальная метафизика была предана забвению уже мыслителями Нового времени, которые не смогли оцен
    ить ее философский потенциал}» [Гагинский, 2021: 68]. 
    Впрочем, в не меньшей мере это относится вообще к современной философии, где онтологическим трансценденталиям уделяется внимание разве что в неотомизме и неосхоластике (см. [Корет, 1998]). Хотя попытки вернуть эту тематику в философский дискурс иногда предпринимаются и представител
    ями иных философских направлений. 
    Особо интересны такие из них, которые при этом используют формализованные логико-онтологические средства: как, например, польский аналитический философ и логик Ян Воленский, который также констатировал: «\textit{Теория трансценденталий составляет центральную часть неосхоластическо
    й онтологии. С очень незначительными исключениями она обычно развивается весьма старомодным, часто туманным образом. В этом, вероятно, и состоит основная причина того, что в современной философии за пределами неосхоластического лагеря теорией трансценденталий, как правило, пренебр
    егают. Поскольку ее происхождение связано с самыми фундаментальными проблемами бытия, думаю, что она заслуживает гораздо более позитивного отношения. Я хочу обосновать возможность ее включения в более современные формальные рамки. Я считаю, что теория трансценденталий в ее совреме
    нной версии может пролить свет на формальную онтологию}» [Воленский, 2016: 205]. 
    Однако применение в данном случае инструментария модальной логики не привело у Воленского (как и у его предшественника из Львовско-Варшавской школы Тадеуша Чежовского) к достаточно адекватной экспликации логики и онтологии трансценденталий. Ибо для таковой экспликации требуются ка
    к более содержательные и более универсальные, так и более гибкие формализованные средства, ныне представленные разрабатываемыми В.И. Моисеевым Проективно-Модальной Онтологией (ПМО), мирологией и R-анализом [см. Моисеев, 2022; Моисеев, 2025].
    Проблема трансценденталий возникла в западноевропейской метафизике при переводе (начиная с Боэция) аристотелевской категориальной системы с греческого на латинский. При этом столкнулись с такой трактовкой Аристотелем сущего, во-первых, при котором оно экстенсионально (объемно) ока
    зывается тождественным единому: «\textit{Сущее и единое – одно и то же, и природа у них одна, поскольку они сопутствуют друг другу так, как начало и причина, но не в том смысле, что они выражаемы через одно и то же определение (впрочем, дело не меняется, если мы поймем их и так; 
    напротив, это было бы даже удобнее)... Так что сколько есть видов единого, столько же и видов сущего, и одна и та же по роду наука исследует их суть}» [Аристотель, 1976: 120-121]. А во-вторых, сущее оказывается таким понятием, которое интенсионально (содержательно) весьма специфич
    но. 
    Так, если категории, будучи высказанными о чем-то, всегда добавляют некоторое содержание к тому сущему, о чем они высказаны, то сущее как таковое аналогичного содержания не добавляет, ибо сущее всеобъемлюще: охватывает все, что может существовать каким бы то ни было образом. Поэто
    му сказать, что нечто есть сущее – все равно, что сказать сущее есть сущее. Соответственно, сущее распределяется по десяти категориям (высшим родам – а затем по частным родам, видам, вплоть до индивидов), но само оно не есть род, который можно разделить на виды. 
    Соответственно, то же относится к единому, то же в «Метафизике» касается и блага, «\textit{ибо благо есть цель всякого возникновения и движения}» [Аристотель, 1976: 70], поэтому, как отмечает Аристотель в «Никомаховой этике», оно тоже приложимо ко всем категориям. Соответственно, 
    они выше всех родов, выше всех категорий, omnia genera transcendit. Из этого латинского «transcendit» и возникло понятие трансцендентальной сигнификации (или просто трансценденталий). 
    Сложность онтологического трансцендентализма приводила к тому, что в разные исторические периоды и у разных мыслителей количество выделяемых трансценденталий и их перечень весьма разнятся. Не вдаваясь в подробности, отметим, что в конечном счете Фома Аквинский выделял шесть трансц
    енденталий: ens (сущее), res (вещь), aliquid (нечто, иное), unum (единство), verum (истина) и bonum (благо), среди которых основными считались единство, истина и благо. Правда, многие мыслители (Гильом Овернский, Фома Галл из Верчелли, Александр Гальский, Жан де Ля Рошель, Бонавен
    тура) относили к трансценденталиям также красоту (прекрасное, pulchrum), что в западноевропейской латиноязычной метафизике продолжалось вплоть до Альберта Великого, после которого красота элиминировалась из этого перечня (но осталась в восточноевропейской, грекоязычной, метафизике
    , непосредственно отталкивавшейся от патристической традиции, где прекрасное и красота, καλὸν καὶ κάλλος, по-прежнему рассматривались как универсальные характеристики бытия). Как бы то ни б
    ыло, в логико-онтологическом плане все таковые трансценденталии являются интенсионально (содержательно) различными, но вместе с тем коэкстенсиональными («равнообъемными»), и в этом отношении полностью взаимообратимыми. То есть одинаково универсальными (репрезентирующими весь униве
    рсум сущего). 
    Следующий шаг на этом пути был сделан Дунсом Скотом, который наряду с «простыми» трансценденталиями постулировал в «Оксфордском сочинении» трансценденталии дизъюнктивные, которые коэкстенсиональны и взаимообратимы только взятые в паре, ибо «\textit{сущее имеет не только простые об
    ратимые свойства (passiones), такие как «единое», «истинное», «благое», но и некие свойства, где противоположное различается между собой, как, например, «необходимое или возможное», «акт или потенция», и прочее в том же роде. Но как обратимые свойства суть трансценденталии, поскол
    ьку свойственны сущему, насколько оно не определено в отношении какого-либо рода, так и раздельные свойства суть трансценденталии. И оба члена этой дизъюнкции трансцендентальны, поскольку ни один из них не определяет свое определяемое к определенному роду; однако одному сущему фор
    мально подходит только один особый (specialis) член этой дизъюнкции, как, например, «необходимое» в делении «необходимое или возможное», или  «бесконечное» в делении «конечное или бесконечное», и так же в отношении другого}» [Дунс Скот, 2002: 291]. Там же он пошел еще дальше, факт
    ически введя, наряду с полной обратимостью трансценденталий, их частичную обратимость. Так, даже «\\fi
