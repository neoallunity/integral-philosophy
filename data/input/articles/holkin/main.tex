\ifprintabstract
    \begin{english}
    % Holkin O. M. - English abstract
    \subsection{\texorpdfstring{\textbf{Holkin O. M. THE CASE OF SOCRATES OR “THE EXPLORED LIFE IS WORTH LIVING”}}{Holkin O. M. THE CASE OF SOCRATES OR “THE EXPLORED LIFE IS WORTH LIVING”}}
    \label{subsec:holkin-en

    This article proposes a method for qualifying the state of legal personality of an individual through the application of non-classical philosophical instruments, including the structural-functional model of subjectivity, the structural-functional model of legal personality, projective-modal ontology, the axiom of valence transfer from whole to parts, polar analysis, reflexive ontologies, and related frameworks. In this first part, the question is examined through the lens of the structural-functional model of subjectivity and legal personality, using the case of Socrates as exemplary material. In a second part, planned for subsequent publication, the Socratic case will be interpreted directly in terms of neoallunity philosophy.

    The case of Socrates was selected deliberately rather than arbitrarily. Despite unlawful criminal prosecution, an unjust verdict, and available means of escape—including tacit acquiescence by prison authorities—the sentence was executed with Socrates' own active participation. Under these circumstances, the question arises: how is one to account for such "exceptional" conduct during the most tragic period of the philosopher's life, particularly given that his actions contradicted his own "status quo" philosophy ("reason = virtue = happiness")? This discrepancy suggests not merely an excess in the subjective sphere, but rather a descent on the subject-ontological scale itself.

    The nature of Socrates' "exceptional" behaviour during the case period (399 BCE, between sentencing and the consumption of hemlock) remains unresolved due to the scarcity of specialised philosophical-legal scholarship addressing this question systematically.

    The present article attempts to formulate one possible answer to this question through the structural-functional model hypothesis, as well as through certain structural frameworks derived from neoallunity philosophy, whilst adhering to the programmatic principles of integral science generally, and the guidelines of the Laboratory of Integral Philosophy of Law specifically.

    \paragraph{Keywords:} {\itshape Socratic case; subjectivity; structural-functional model of subjectivity and legal personality; modal tri-positional structure of subjectivity and legal personality; fundamental modes (principles): bionomy, socionomy, onto(noo)nomy; "periodic law" of subjectivity and legal personality; law of modal elevation}
    \end{english}
\else
    \begin{russian}
    % Холкин О.М. - русская статья
    \subsection{\texorpdfstring{\textbf{Холкин О.М. КАЗУС СОКРАТА ИЛИ «ИССЛЕДОВАННАЯ ЖИЗНЬ СТОИТ ТОГО, ЧТОБЫ БЫТЬ ПРОЖИТОЙ»}}{Холкин О.М. КАЗУС СОКРАТА ИЛИ «ИССЛЕДОВАННАЯ ЖИЗНЬ СТОИТ ТОГО, ЧТОБЫ БЫТЬ ПРОЖИТОЙ»}}
    \label{subsec:holkin-ru

    \textit{В статье предлагается один из способов квалифицирования состояния правосубъектности того или иного лица с применением философских неклассических инструментов таких как структурно-функциональная модель субъектности (СФМС) и СФМ правосубъектности (СФМПС), проективно-модальная онтология (ПМО), аксиома переноса валентности с целого на части, полярный анализ (ПА), рефлексивные онтологии и др. В данной части статьи поставленный вопрос будет рассмотрен в терминах СФМС(СФМПС)-концепта на примере казуса (юридический случай) Сократа. Во второй части, планируемой для публикации в дальнейшем, казус Сократа будет проинтерпретирован непосредственно в терминах философии неовсеединства.  }
    \textit{Казус Сократа для целей статьи был выбран не случайно. Дело в том, что не смотря на незаконное уголовное преследование, неправосудный приговор и имеющиеся условия для побега, включая потворство тюремных властей, приговор был приведен в исполнение при самом деятельном способствовании тому самим Сократом. При таких обстоятельствах нам было интересно разобраться в природе такого «особенного» поведения философа в самый трагический период его жизни с учетом того, что занятая Сократом позиция противоречила его  «статус-кво»-философии («разум = добродетель = счастье»), что свидетельствовало не просто об эксцессе в субъектной сфере, но и о наличии признаков  снижения планки на субъектно-онтологической шкале. } 
    \textit{Вопрос о природе «особенного» поведения Сократа в казусный период (399 г. до н.э., после даты вынесения приговора и до момента принятия цикуты) все еще остается без обобщенно-инвариантного по своему характеру ответа за недостаточностью специальных философско-правовых исследований. } 
    \textit{В данном статье автор делает попытку сформулировать один из возможных вариантов ответа на этот вопрос с использованием СФМС(СФМПС)-гипотезы, а также некоторых структур философии неовсеединстваа, сообразуясь при этом с программными положениями Института интегральной науки (ИИН) в целом и установками Лаборатории интегральной философии и теории права (ЛИФиТП), в частности.} 

    \paragraph{Ключевые слова:} {\itshape Казус Сократа, субъектность, структурно-функциональная модель субъектности (правосубъектности), модальностная триположность субъектности (правосубъектности), базовые режимы (начала): биономия, социономия, онто(ноо)номия; «периодический закон» субъектности (правосубъектности), закон модальностного возвышения}

    \textbf{Введение}
    Исследования целого ряда авторов по вопросу о субъектности, как правило, увязывались с отдельными аспектами этого феномена [Ватин, 1984; Декомб, 2011;Забелин, 1970; Комаров, 2014 и др.]. При этом подавляющая часть этих исследований не была доведена до уровня «инструментальных» разработок, имеющих сугубо философско-прикладное значение. На этом фоне выделяются разработки Моисеева В.И. в рамках развиваемой им философии неовсеединства, которые свидетельствуют о наметившейся твердой линии на преодоление «частичных» (узкоспециализированных) по своему характеру подходов и на приведение знания, в том числе по вопросу о субъектной онтологии, в соответствие с актуальной востребованностью целостных моделей с использованием неклассических подходов в сочетании со структурной строгостью и с учетом возможности их (моделей) практической применимости [Моисеев, 2012; Моисеев, URL].
    В качестве «рабочего» инструмента квалифицирования состояния субъектности в данной части статьи автором будет применена предлагаемая в порядке гипотезы структурно-функциональная модель субъектности (СФМС), а также вытекающая из нее для целей применения в правовом обороте структурно-функциональная модель правосубъектности (СФМПС). Во второй части статьи планируем проинтерпретировать казус Сократа с применением структур философии неовсеединства (ПМО, полярный анализ, аксиома переноса валентности, начала рефлексивных онтологий и др.).
    Поскольку в данной части статьи стоит о применении СФМС(СФМПС)- гипотезы к конкретному субъектно-правовому событию, то логично начать с экспликацию этой модели. 
    \textbf{К вопросу о субъектности (правосубъектности).}
    В правовой науке под правосубъектностью понимается способность физического или юридического лица иметь и осуществлять непосредственно или через своих представителей юридические права и обязанности, то есть выступать субъектом правоотношений. При этом правосубъектность рассматривается как  такой этих лиц функционал, который возникает и признается за ними с наступлением установленных законом обстоятельств (юридических фактов (рождение, достижение совершеннолетия, эмансипация и т.д.). Этим  объясняется и определенный элементный состав так понимаемой правосубъектности: правоспособность, дееспособность, деликтоспособность. 
    В данной же статье субъектность (правосубъектность) рассматривается в философско-онтологическом аспекте, имеет иной элементный состав, а также отграничивается равно как от понятия субъективности (правосубъектности), так и от понятия объективности (правообъективности) в правовых представлениях (П), оценках (О) и волеизъявлениях (В). Правообъективность, взятая при условии легистской парадигмы, увязывается с определением соответствий между ПОВ и наличными законностью и правопорядком, конвенциональными по своей природе. Правосубъектность в этой парадигме увязывается с несоответствиями ПОВ условным юридическим нормам и установленному правопорядку, при которых какое-либо несоответствие поведения установленным нормам квалифицируется как нарушение законности и правопорядка. Между тем и субъективное и объективное в ПОВ по определению так или иначе субъектны (относятся к субъекту). Если так, то существует точка пересечения, в которой субъективное объективно, а объективное субъективно с соответствующими «полями» распределения в ту или иную сторону от этой точки. В этой точке пересечения ПОВ по определению будут увязываться (должны увязываться) с более высоким уровнем урегулированности гражданско-правового оборота. Из природы этой точки следует, что она постоянно реляционирует и обусловлена действием закона развития как высшего нормативного регулятора социальной жизнедеятельности.   При «живой» правонормативной практике с опорой на закон развития участники гражданского оборота по определению будут вынуждены (должны) в перманентном режиме заниматься  актуализацией условных правил (закона) и быть готовым к тому, что в случаях расхождения права (в юс-натуралистском смысле) и закона вступать в противоречие с позитивно-правовыми условностями, но при всем этом  иметь  положительную деятельностную «валентность» [Левин, 2025]. Подобный субъектно-онтологический подход к нормативным системам и к описанию субъектного коррелята и его динамики был предложен в рамках философии неовсеединства [Моисеев, 2018: 69] и в дальнейшем развит автором [Холкин, 2016; Холкин, 2017].
    СФМС(СФМПС)-концепт не противоречит существующим в специальной литературе представлениям по поводу субъектности как структурного «обеспечительного» начала «внешней» деятельности тех или иных лиц [Ватин, 1984; Забелин, 1970; Комаров; Моисеев] и используется далее с учетом вышеизложенного в качестве целевого «рабочего» инструмента в рамках заданного интервала исследования. 
    \textbf{«Матричная» структурно-функциональная модель правосубъектности (СФМПС)}
    Правосубъектность, являясь частным случаем субъектности, по определению имеет изоморфную конструкцию. Структурно-функциональная модель субъектности с целым рядом примеров ее практического применения была подробно описана в монографии «Функциональная модель субъективности: гипотеза и практические применения» [Холкин, 2016]. При таких обстоятельствах в данном месте ограничимся лишь схематическим описанием СФМС, преобразованной сразу же в аналогичную «матрицу» СФМПС [Холкин, 2016, 2-4] (см. рис. 1).
     
          Бк                 Ск                  Нк
          Бп                 Сп                                                     Нп
          Бв                 Св                  Нв

    {\centering \textbf{Рис.1: «Матрица» правосубъектности (СФМПС)} \par}


    Когнитивное (классификации);
    Эмотивное (оценки, переживания); 
    Волевое (правовой выборжелания).
    Биономия Социономия Онто(ноо)номия
    где:
    Бк, Бп, Бв – соответственно, биономные по своей природе и доминантной экзистенциальной установке   правовые представления (образы), ценностные правоориентации и волеизъявления;    
    Ск, Сп, Св – соответственно, социономные по своей доминантной экзистенциальной установке правовое когнитивное, эмотивное и волевое;   
    Нк, Нп, Нв – соответственно, онто(ноо)номные по своей доминантной экзистенциальной установке правовые представления, ценностные ориентации и воления. При этом термины биономия, социономия и ноономия (онто(ноо)номия) введены по аналогии с терминами и (или) их смыслами, используемыми в книге «Человек и человечество» [Забелин, 1970]. 
    Из рисунка 1 следует, что СФМПС включает в себя два «среза» субъектности (условно «вертикальный» и «горизонтальный»). Вертикальный «срез» складывается из когнитивного, эмотивного и волевого начал, каждое из которых фундировано в базовой модальностной триположности своих выражений в зависимости от наличных субъектных доминантных состояний, укорененных в биономии, социономии или в онто(ноо)номии, выражающих горизонтальный «срез» феномена субъектности, а с учетом взаимодополнительности – структурно-функциональную полнотность (структурный «канон» (плерон) и направленность модальностного генеза).
    Данная «матричная» структурность, где все компоненты «вертикального» среза правосубъектности в свете «горизонтального» предстают как тройственные, приводит к тому, что каждый из компонентов «вертикального» среза может быть представлен в терминах модальностной троякости.  Это означает, что правовое когнитивное в этом смысле предстает и как трехэлементная слойная «связность», и как «план динамики» (генеза») со смещением доминанты в модальностном поле: 1) «наивный» образ общественного правобытия (правовоззрение на уровне «субъективно-наивной картины правопорядка, основанный на сугубо субъективной достоверности в масштабе «у самого себя»); 2) рассудочные представления, конвенциональные («габитуальные» по своей природе, что суть фикции, т.е. считаемое правильным); 3) «разумный» по природе и основанный на критериях логики, практики, научности и фундаментальности образ правовой онтологии (достоверность, осознанная лицом свободным в понятиях и концептах, имеющих обобщенно-инвариантный, критически-рациональный и  конкретно-исторический характер с контекстуальной увязанностью).  
    У Гегеля в «Философии права» по этому поводу имеется следующий пассаж: «… истина о праве, нравственности, государстве столь же стара, сколь открыто дана в публичных законах, публичной морали, религии и общеизвестна. В чем же еще нуждается эта истина, поскольку мыслящий дух не удовлетворяется обладанием ею таким наиболее доступным для него образом, если не в том, чтобы ее постигли и чтобы самому по себе разумному содержанию была придана разумная форма, дабы оно явилось оправданным для свободного мышления, которое не может остановиться на данном, независимо от того, основано ли оно на внешнем положительном авторитете государства, на общем согласии людей, на авторитете внутреннего чувства и сердца и непосредственно определяющем свидетельстве духа, исходит из себя и именно поэтому требует знания себя в глубочайшем единении с истиной?» (выделено мной. – О.Х) [Гегель, 1990: 45]. 
           Правоволение как «вертикальный» компонент правосубъектности в рассматриваемом смысловом интервале  также предстает как «тройственный» по структуре и в динамике: (1) «воля-произвол» («каприз»),  (2) воля, приведенная в соответствие с наличными условными конвенциями  («общественный договор», позитивное право, нормо-фикции и догматы),  (3) «свободная воля», приведенная в соответствие с освоенными законами всеобщей право-волевой онтологии, а также – с обобщенно-инвариантными представлениями об актуальном правовом должном в конкретно-историческом контексте.  По этому поводу Гегель писал, что достоверность составляет природу «я», ибо не может существовать «я» без знания о себе, но эта достоверность в начале есть абстрактная, субъективная достоверность, так сказать, «в-себе»-достоверность. Точно также и свобода составляет природу воли, но в начальном положении это субъективная свободная воля или «произвол» и только объективная достоверность, истина, соответствует подлинной свободе воли [Гегель, 1997: 318-328]. 
    Очевидно, что подобным образом может быть представлен и эмотивный компонент правосубъектности. 
     «Горизонтальный» срез правосубъектности, рассмотренный сам по себе и как это следует из «матрицы», также предстает как тройственный по своему содержанию. Это означает, что каждая из трех модальностей правосубъектности непреодолимо тематизирована (должна быть тематизирована) когнитивным, эмотивным и волевым началами в силу сложения/умножения «срезовых» матриц. К примеру, социономическая («габитуальная») по своей природе правосубъектность (социономический режим «экзистирования») по определению должна быть представлена во внутреннем мире данного лица принятой в базовом социуме системой правопонимания (официальная «регулятивная» доктрина), правопереживанием, основанным на положительном авторитете государства и так называемого позитивного права, правоволением, основанным на законопослушании.
    По аналогии может быть построена соответствующая систематика в отношении других модальностей правосубъектности. 
    Вышеизложенный подход  позволяет произвести матричное исчисление компонентных констелляций правосубъектности путем перебора всех возможных вариантов соотношения  «матричных» ячеек между собой, а именно:
    1.        Бк-Бп-Бв, Бк-Сп-Бв, Бк-Нп-Бв, Бк-Бп-Св, Бк-Бп-Нв, Бк-Сп-Св, Бк-Сп-Нв, Бк-Нп-Нв, Бк-Нп-Нв;
    2.        Ск-Бп-Бв, Ск-Бп-Св, Ск-Бп-Нв, Ск-Сп-Бв,  Ск-Сп-Св, Ск-Сп-Нв, Ск-Нп-Бв, Ск-Нп-Св,  Ск-Нп-Нв;
    3.        Нк-Бп-Бв, Нк-Бп-Св, Нк-Бп-Нв, Нк-Сп-Бв, Нк-Сп-Св, Нк-Сп-Нв, Нк-Нп-Бв, Нк-Нп-Св, Нк-Нп-Нв.
    В результате данного исчисления оказываются найденными 27 вариантов констелляционных структур правосубъектности. Из них – три модальностно «цельные» констелляции в смысле нахождения «вертикальных» элементов в одном и том же «горизонтальном» положении (выделены полужирным шрифтом), остальные – «смешанные».
    Если полученная модель систематики констелляций правосубъектности отражает действительное положение дел в данной сфере, то мы получаем вполне рациональные основания для утверждения, что каждый индивид в части своей правосубъектности на тот или иной момент своей жизнедеятельности является носителем доминирующей структурной констелляции.  Разумеется, что «живая правосубъектность» есть постоянное движение и изменение, то в этом смысле для выражения данного нюанса по аналогии могла бы подойти фигура речи: «мерцающая правосубъектность». Между тем, несмотря на факт постоянного «движения и изменения» она, как было показано выше, представляет собой также и «качественные» состояния, определяемые процессами доминирования тех или иных экзистенциальных установок, связанных с масштабированием «жизненного мира» и ориентацией на соответствующие системы «нормализации» положения дел.
    Следует обратить внимание на то, что при рассмотрении самого по себе «констелляционного» исчисления «составов» правосубъектности обнаруживается некая связность, имеющая регулярный характер, и которая могла бы быть обобщена как своего рода «периодический закон» правосубъектности, а именно: свойства индивида (в смысле его «правосубъектного» статуса) стоят в периодической зависимости от модальностного состояния (режима) его правосубъектности. Табличным выражением данного «закона» является представленная выше своего рода «периодическая» система (констелляционная распределенность) компонентов правосубъектности. В части же реального генеза состояний субъектности в процессе развития из смысла СФМС(СФМПС) следует выделить также другую связность, касающуюся области генеза и также имеющую характер регулярности,  которую не без оснований можно было обозначить как закон модальностного возвышения состояний субъектности: от низшей модальности к высшей с сохранением предыдущих в «снятом» виде. 
    Кроме того, модель позволяет сформулировать еще один фундаментальный момент, присущий субъектности – это транспоральность (субъектная реляционность), т.е. подвижное содержание внутреннего мира (образы, оценки, воления) с подвижных модальностных позиций (состояний).  
    Вышеизложенный СФМПС-концепт представлен в порядке гипотезы, подлежащей дальнейшей проверке (верификация, фальсификация, конкретизация).  Однако и в данном виде этот концепт может пониматься, как говорят юристы, в качестве «справочного» инструмента, т.е. используемого в качестве «регулятивного» за отсутствием иной удовлетворяющей требованию достаточности объяснительной модели при постановке вопроса в обозначенном автором интервале. При таких обстоятельствах с учетом сказанных ограничений предложенный концепт и будет далее использован в качестве «рабочего» инструмента с соответствующим понятийным аппаратом для разрешения казуса Сократа и квалифицирования состояний субъектности на основании анализа его высказываний (доводов) и фактического поведения в указанный период.
    \textbf{Казус Сократа и субъектный коррелят}
    Историческое по значимости биографическое событие, которое произошло с Сократом в 399 году до н.э. будем именовать как казус Сократа. 
    Суд над Сократом – один из самых казусных в прямом и переносном смысле судебных процессов в истории мирового судопроизводства. Фактически Сократ был осужден за то, что задавал каверзные вопросы обществу, на которые было трудно или невозможно ответить. Тем не менее эти обстоятельства были оценены как разрушительные и опасные для общества, как неуважение к афинским богам и привлечение молодежи к опасной и греховной философии, и легли в основание вынесения смертного приговора.
    Между тем анализ судебного заседания, содержания приговора показывает, что он был вынесен, как это  звучало бы в современном процессе, в отсутствие состава преступления (отсутствие события, вины и т.п.),  что подтверждается в частности выводами и решением «второго судебного процесса» над Сократом, в результате которого Сократ был оправдан в полном объеме\textsuperscript{\footnote{В 2012 году в здании афинского фонда Онассиса состоялся повторный «суд» над Сократом, в котором принимала участие международная группа юристов и почти тысяча зрителей в качестве присяжных. Сократа оправдали, но как это не иронично, вновь не единодушно, хотя и со значительным перевесом [https://www.gazeta.ru/culture/news/2012/05/26/n_2359765.shtml].}}.
    В этом трагическом событии философов и юристов до сих пор интересует удивительный факт, не получивший до сих пор концептуального философско-правового объяснения: почему Сократ в постсудебный период снизил планку стоящих за его поведением принципов, о чем свидетельствует противоречие между философско-правовой "апологией» Сократа от самого Сократа, выраженной в его справедливой речи о своей невиновности на суде, с одной стороны, и сократической «апологией» своего смирения и «деятельным» оправданием приведения приговора в исполнение в постсудебный период, с другой. Как известно, Сократ отказался от реальной возможности избежать смертной казни на условиях, предложенных его соратниками и друзьями (побег из тюрьмы при явленном ослаблении тюремной администрацией режима охраны) со ссылкой на основания, не противоречащие  «статус-кво»-философии Сократа. 
    Объяснение этого «нелогичного» поведения по нашей гипотезе лежит в возникновении «особенного» состояния субъектности Сократа в указанный период. Рассмотрим это обстоятельство подробнее с использованием сведений, содержащихся в диалоге Платона «Критон» [Платон, 1994, Т.1: 97-111]. 
    Из смысла и содержания диалога следует, что на рассмотрение его участников были поставлены следующие вопросы: 1) надлежит ли подчиняться условному закону и (или) судебному акту, если они противоречат идее Права (Справедливости); 2) о подобающем выборе между альтернативами: оказать гражданское неповиновение (совершить побег из тюрьмы) или принять смерть согласно решению суда.
    Из текста следует, что Сократ считает правильным следовать условному закону и принятому судом приговору.
    Критон же утверждает, что побег Сократа, как способ спасения жизни, является наиболее правильным и справедливым действием при сложившейся ситуации со ссылкой на доводы, которые можно свести к следующим:
    а) если иначе, то он (Критон) лишится друга;
    б) многие сочтут, что Критон не позаботился спасти друга (Сократа), будучи в состоянии сделать это, что аморально;
    в) не надо бояться Сократу того, что побег может привести к гонениям его друзей и родственников;
    г) зачем предавать себя, когда можно спастись;
    д) умереть, значит обречь близких на произвол судьбы, между тем как Сократ мог бы их и прокормить, и воспитать;
    е) если иначе, то люди станут думать, что все это случилось с Сократом по какой-то трусости со стороны друзей, что позорно.
    Из доводов Критона следует, что часть из них (а, в, г) соотнесена с экзистенциальными установками, укорененными в биономном в указанном выше смысле начале субъектности, с доминантной направленностью на сохранение жизни самой по себе, а другая (б, д, е) – с неписанными, но широко распространенными в полисе оценочными мнениями, допущениями или и вовсе предрассудками. 
    Сократ, как следует из диалога, возражает, руководствуясь соображениями иного рода:
    а) зачем заботиться о мнениях большинства, когда люди со смыслом, которых только и стоит принимать в расчет, будут думать, что это (казнь) случилась так, как она случилась; 
    б) в этом вопросе следует руководствоваться разумным убеждением, которое и является наилучшим, а не устрашениями всесильного большинства, оковами, казнями и лишениями имущества, которые они насылают на нас;
    в) надо обращать внимание, уважать и ценить  не все мнения, а только  мнения людей разумных (в вопросах, касающихся гимнастики, важно не любое мнение, а мнение именно врача или учителя гимнастики; точно также обстоит дело относительно справедливого и несправедливого, безобразного и прекрасного, доброго и злого: нужно следовать не мнению большинства, а только мнению одного, если только есть такой, кто понимает, и кого должно стыдиться и бояться больше, чем всех остальных, вместе взятых);
    г) нужно заботиться о том, что скажет о нас не большинство, а тот, кто понимает, что справедливо и что несправедливо, – он один, да и еще сама истина;
    д) всего больше нужно ценить не жизнь как таковую, а жизнь достойную;
    е) вопрос не в побеге как таковом, а в том, справедливо ли мы поступим, если прибегнем к подкупу охраны или иным образом избегнем уже назначенной судом казни, и полагаем ли мы, что ни в каком случае не следует добровольно нарушать справедливость или что в одном случае следует, а в другом нет?
    ж) ни в каком случае не следует нарушать справедливость, что означает, что вопреки мнению большинства нельзя воздавать несправедливостью за несправедливость, ни воздавать злом за претерпеваемое зло;
    з) впереди справедливости нельзя ничего ставить – ни детей, ни жизни, ни еще чего-нибудь;
    и) если бы пришли сюда на беседу сами Законы и само Государство, то они спросили бы: разве у нас был какой-нибудь иной договор, кроме того, чтобы твердо стоять за судебные решения, которые вынесет город? Или в своей мудрости  Сократ не замечает, что Отечество и Законы драгоценнее и матери, и отца, и всех остальных предков, что они священнее и неприкосновеннее и в большем почете и у богов и у людей – у тех, у которых есть ум, и если отечество гневается, то ему нужно уступать и угождать больше, чем отцу, а если что велит, то это и  делать. Это – справедливо, ибо все блага приходят от пребывания в отечестве после того, как афинянин занесен в гражданский список.
    Из этих доводов Сократа, на первый взгляд, следует, что избранное им поведение соотнесено с экзистенциальными установками, укорененными в онто(ноо)номическом по природе модальностном состоянии субъектности.    
    Между тем более тщательный анализ приводит к выводам иного рода.
    И действительно, Сократ, как следует из довода «а», пренебрегает мнением «толпы» и считает, что нужно руководствоваться разумением «людей со смыслом», полагающих, что казнь «случилась так, как она случилась» («все действительное разумно»). Между тем согласно «статус-кво» Сократа-Платона действительность (видимая реальность) есть «тень» (становление), а действительная «действительность» есть вечный «мир идей» (эйдосов), включая «Благое», «Законы», «Государство». При таком положении дел верховенство признается за вечными, неизменными и совершенными формами (прообразами) – идеями, а применительно к людям – за «разумным» началом души, а не за «вожделеющим» и «гневливым» [Платон, 1994, Т.1: 212-219]. Если так, то по меньшей мере не все действительное должно быть признано разумным, а только то, что является таковым в специальном значении этого понятия. С учетом сказанного  «а»-довод Сократа и вытекающее из него волеизъявление должны быть квалифицированы как противоречащие «статус-кво», поскольку за ними кроется допущение о верховенстве условного правового порядка, тогда как согласно «статус-кво» и «Закону» условные и преходящие по своей социономной природе дискреционные акты «государства-тени» должны признаваться достойными исполнения только в случае их соответствия «Закону» законов. Судебный акт в отношении Сократа по общепринятому факту не соответствовал «Закону» законов. 
    Таким образом, с учетом того, что судебный акт в отношении Сократа не соответствовал «Закону» законов, то волеизъявление Сократа, основанное на «а»-доводе, нельзя признать «разумным» в смысле   философии Сократа-Платона, поскольку по своей природе оно может (должно) быть соотнесено с социономным началом субъектности и квалифицировано как выражение соответствующего субъектного коррелята (состояние, соответствующее социономической по природе модальности).
    Следующий «б»-довод Сократа  сам по себе может быть расценен как вытекающий из онто(ноо)номического начала субъектной онтологии, поскольку указывает на необходимость его (вопроса) разрешение с отсылкой к разумным убеждениям как наилучшему основанию по сравнению с апелляцией к устрашениям всесильного большинства об оковах, казнях и лишении имущества. Однако не является ли данный довод защитным, прикрывающим субъектный коррелят совсем иной природы? В целях проверки этого сомнения сопоставим смысл и содержание «б»-довода с сократически-платонической онтологией «разумности», с одной стороны, а с другой – сопоставим его с фактическим поведением не предмет выявления соответствия/несоответствия. Для этого дополнительно обратимся к  «в», «г», «д»-доводам, в рамках которых Сократ ставит вопрос об основаниях выбора, из содержания которых следует, что все они   по своей заостренности  и смысловой установке опять-таки соотнесены с высшей модальностью когнитивно-эмотивного «срезов» субъектности, т.е. вытекающими из «разумения» («надо обращать внимание, уважать и ценить  не все мнения, а только  мнения людей разумных», «нужно следовать не мнению большинства, а только мнению одного, если только есть такой, кто понимает»,  нужно заботиться о том, что скажет о нас не большинство, а тот, кто понимает, что справедливо и что несправедливо, – он один да еще сама истина»,  «всего больше нужно ценить не жизнь как таковую, а жизнь достойную»). 
    Анализ этих обстоятельств по аналогии с «а»-доводом  показывает, что Сократ на практике (действие/бездействие) не следует теоремам, вытекающим из аксиом своей же «статус-кво»-философии. При этом непоследовательность обнаруживается как по вопросам права, так и по вопросам факта.
    Сократ также ставит вопрос («е»-довод): «справедливо ли мы поступим, если прибегнем к подкупу охраны или иным образом избегнем уже назначенной судом казни, и полагаем ли мы, что ни в каком случае не следует добровольно нарушать справедливость или что в одном случае следует, а в другом нет?».     
    Однако о какой справедливости (типе справедливости) говорит Сократ в качестве участника диалога Платона?  Этот вопрос является существенным в СФМС(СФМПС)-релевантном смысле, согласно которому абстрактной справедливости нет, а есть всегда конкретная и при данных конкретных обстоятельствах.  
    Так согласно «ж» и «з»-доводам, Сократ определенно утверждает, что ни в каком случае не следует нарушать справедливость, что вопреки мнению большинства нельзя воздавать несправедливостью за несправедливость, ни воздавать злом за претерпеваемое зло, что впереди справедливости нельзя ничего ставить – ни детей, ни жизни, ни еще чего-нибудь. Это означает, что Сократ, не смотря на оценку приговора как несправедливого, тем не менее активное неповиновение приведению его в исполнение относит к действиям несправедливым. Какое допущение о конкретном типе справедливости лежит в основе этих доводов и волеизъявления Сократа?
    Типологию справедливости можно «вытащить» из смысла и содержания учения Платона о трех началах души [Платон, 1994, Т.3:, 212-219, 372-376], из которых он, в частности, выводит три типа удовольствий. Согласно этой логике трем началам человеческой души соответствуют (должны соответствовать) три типа справедливости: в терминах Сократа-Платона это – «корыстолюбивая» (эгоцентричность), «честолюбивая» (социоцентричность в доктринах, оценках, волеизъявлениях и в поведении) и «разумная» справедливость (основанная на познающем начале, онто(ноо)центричность). В свете этого учения доводы Сократа сами по себе апеллируют к «разумному» типу справедливости. Однако, будучи рассмотренными по совокупности с фактическим поведением (непротивление и деятельное обоснование непротивления) Сократа, за ними фактически стоит социономный по своей природе коррелят субъектности, поскольку сделанный выбор основан на оправдании примененной по отношению к нему санкции условной природы. Это означает, что в конкретной ситуации Сократ, не смотря на аргумент «разумности, по факту применяет социономический стандарт справедливости, считая, что если санкция совершается от лица общества, то в любом случае – справедливая она или нет – должна быть исполнена.
    Таким образом, оценка Сократом альтернативного поведения (деятельное неповиновение) как несправедливого также  основана на условной справедливости социономной природы, а не на «разумной» справедливости, что опять-таки свидетельствует о его непоследовательности, за которой стоит соответствующий субъектный модальностный коррелят. 
    Такая, а не другая квалификация поведения Сократа, обусловленного соответствующим модальностным состоянием субъектности (социономия), подтверждается также следующим. В книге первой «Государства» один из героев диалога утверждает, что справедливое это пригодное сильному (властям, органам государства, его институтам, суду в том числе), но Сократ-Платон возражают, считая что это так только при условии, если пригодное (справедливое)  есть то, что пригодно государству как «Целому» при условии, что сословия ведут себя в соответствии с их природой, когда «каждое из них имеет свое и исполняет тоже свое». [Платон, 1994, Т.3: 204-206]. Сократ – по природе мудрец, философ. Если так, то из этого следует, что если «государство-тень» повело себя в противоречии с принципами «Целого», то согласно доктрине Сократа-Платона о справедливости, Сократу необходимо было поступить в соответствии с «велениями» вечного и неизменного «Государства-эйдоса»,  т.е. «разумно» [Платон, 1994, Т: 83-84], что означает: справедливым выбором для Сократа, по-видимому, было оказать деятельное неповиновение неправосудному приговору, наносящим ущерб и «целостности» полиса (Сократ занимался именно своим по его природе делом) и «целостности» самого Сократа. Как бы то ни было, Сократ не применил на практике начала своей философии для надлежащего разрешения трагического казуса, касающегося его лично, там и тогда, где и когда он больше всего в этом нуждался.
    Почему при столкновении с наличным государством дал «сбой» сократическо-платонический концепт о самом «Справедливом» – этот вопрос заслуживает отдельного рассмотрения. Здесь же попробуем в предварительном порядке разрешить чуть ли на самый интересный в данном контексте вопрос, а именно: проявившаяся в казусе не-аутентичность субъектности Сократа имела ситуативный характер, или же она была вполне ординарным субъектным состоянием Сократа? 
    Пролить свет по данному поводу, на наш взгляд, может последнее распоряжение Сократа о воздаянии Асклепию (мифический бог медицины), которое в известном смысле, является ключевым. Почему Сократ обратился к друзьям с таким распоряжением, а не другим? Из смысла этого распоряжения следует, что за ним стоит допущение о некоей болезни, от которой Сократ вдруг избавился и поэтому распорядился сделать воздаяние Асклепию. Но в чем заключалась собственно «болезнь», в том ли, что он был несправедлив, не оказав неповиновение несправедливому закону и акту государства (в момент распоряжения яд уже был принят), или в другом:  «преобразившись»  усомнился в самом «статус-кво»? 
    В этом смысле интересными представляются соображения Ницше, который оценил обобщенную жизненную позицию и поведение Сократа как «тип упадка» [Ницше, 1990: 563]. Ницше подчеркнул, что сократическое уравнение «разум = добродетели = счастье» есть частный случай идиосинкразии [Ницше, 1990: 564], а вера Сократа в «разумность во что бы то ни стало» является заблуждением и патологией [Ницше, 1990: 567]. Ницше считает, что просьба Сократа про Асклепия за момент от кончины подтверждает его (Ницше) утверждение о «болезни» Сократа [Ницше, 1990: 567]. Если так, то эпизод с Асклепием может в известной мере свидетельствовать о том, что философия и философствование Сократа могли являться «защитной» когнитивной диспозицией по отношению к фактически данной и социономически обусловленной «сердечной» природе субъектности Сократа. Если так, то становится понятным и то, почему философия «самой Справедливости» оказалась (должна была оказаться) не «работающей» в самый ответственный момент, когда требовалось  на практике осознанно осуществить свободный выход за границы неадекватных позитивных (условных) норм и актов в направлении развития Целостности Государства и поддержания Справедливости в смысле «каждому по его природе». Однако произошла удивительная подмена «природ».   
    Таким образом, соглашаясь с Сократом о том, что   жизнь, детей, друзей и т.п. нельзя ставить впереди Справедливости, вместе с тем с учетом факта совершения в отношении Сократа самой Несправедливости, с одной стороны, а также применения в целях определения «разумного» волеизъявления (поведения) сократического «статус-кво» и  СФМС(СФМПС)-концепта, с другой стороны, совсем не безосновательным становится вопреки вышеизложенному доводу Сократа вывод, что при определенных обстоятельствах  справедливым следует признать выбор, когда перед фикцией справедливости впереди надо поставить «и жизнь, и детей, и друзей…», т.е. максиму Целого и Развития.
    В «и»-доводе Сократ вопрошает: «…если бы пришли сюда на беседу сами Законы и само Государство, то они спросили бы: разве у нас был какой-нибудь иной договор кроме того, чтобы твердо стоять за судебные решения, которые вынесет город? Или в своей мудрости  Сократ не замечает, что Отечество и Законы драгоценнее и матери, и отца, и всех остальных предков, что они священнее и неприкосновеннее и в большем почете и у богов и у людей – у тех, у которых есть ум, и если отечество гневается, то ему нужно уступать и угождать больше, чем отцу, а если что велит, то это и  делать. Это – справедливо, ибо все блага приходят от пребывания в отечестве после того, как афинянин занесен в гражданский список».
    Из этого пассажа следует, что Сократ отождествил наличное государство и наличные законы с «самим Государством» и «самими Законами». Сама история указала на эту ошибку в постказусный период: по свидетельствам Диогена Лаэртского вскоре после казни Сократа афиняне раскаялись, закрыли палестры и гимнасии, Милета (один из обвинителей) осудили на смерть, остальных (Милет, Ликон) – на изгнание, в честь Сократа воздвигли бронзовую статую [Диоген, 1988: 145].
    Вышеизложенный анализ и его предварительные результаты  по вопросу о фактическом состоянии субъектности Сократа в казусный период не противоречат также результатам от применения к данному случаю аксиомы валентности (Моисеев В.И.). Согласно этой аксиоме применительно к данному случаю, если правовая по характеру активность (действие/бездействие), воспринимаемая субъектом как положительная «степень самого себя у самого себя» соответствует не правовому в юснатуралисткой парадигме закону (в том числе приговору суда), то согласно механизму валентного «смещения» такое поведение является «-»-действием с позиции объективного и перманентно становящегося Права и Справедливости. С другой стороны, если действие/бездействие противоречит принятым, но не соответствующим канону «Права» (Справедливости) условностям, то согласно механизму валентного «смещения» такое поведение должно получить значение «+»-действия. Этим, к примеру, объясняется открытие С. Кьеркегором в работе «Страх и трепет» парадокса, когда лицо, будучи «Единичным» по природе, поднявшись в духе своем до «Абсолютного», должно расцениваться как вышестоящее по отношению к «Всеобщему» (социумным условностям) [Кьеркегор, 2014]. 
    С учетом сказанного и применения аксиомы валентности «особенное» поведение Сократа в исполнительной стадии судебного производства следует в известной мере (в порядке гипотезы) квалифицировать как «-»-действие/бездействие.
    К такому, к сожалению, не утешительному результату, не противоречащего логике и фактам, привело нас применение СФМС(СФМПС)-концепта к казусу Сократа.
    \textbf{Заключение}
     «Опыт умирания» Сократа («философствовать, значит учиться умирать») – Великий Опыт-Урок Великого Человека.
    В вышеприведенной работе «Сумерки идолов (Проблема Сократа) [Ницше, 1990: 563-567] Ницше прямо не сказал, что неисследованная в [«заблуждении и патологии»] жизнь не стоила того, чтобы быть прожитой, но из смысла его сочинения следует, что об этом он сказал (по умолчанию). Между тем имеются достаточные основания с этим допущением не соглашаться, и порукой тому является эпизод с Асклепием: с наступлением факта «выздоровления» жизнь Сократа получила полнотное онтологическое завершение за ее исследованностью. Таким образом, сентенция Сократа – «неисследованная жизнь не стоит того, чтобы быть прожитой» – к Сократу не применима.  
    Представленная в статье интерпретация казуса Сократа является в терминах проективно-модальной онтологии (ПМО) лишь одной из мод соответствующего исследовательского-интерпретационного модуса, полученной при условии применения структурно-функциональной модели субъектности (правосубъектности). Далее с учетом заданного автором интервала рассмотрения темы, предполагается более строго построить логику и динамику состояний субъектности на том же примере с применением структур и конструкций философии неовсеединства.
    \textit{ Продолжение следует.} 

    {\centering \textbf{Литература} \par}
    Архангельский С.И. Учебный процесс в высшей школе, его закономерные основы и методы [Текст] / С.И. Архангельский //. – М.: Высшая школа, 1980. – 369 с.
    Ватин И.В. Человеческая субъективность [Текст] / И.В. Ватин. – Издательство Ростовского университета, 1984. – 200 с.
    В Греции пересмотрели дело Сократа: философ оправдан спустя 2500 лет. https://www.gazeta.ru/culture/news/2012/05/26/n_2359765.shtml (актуально на 15.08.2025 г.)
    Гегель Г. В. Ф. Философия права. Пер. с нем.: Ред. и сост. Д. А. Керимов и В. С. Нерсесянц; Авт. вступ. ст. и примеч. В. С. Нерсесянц.– М.: Мысль, 1990.– 524 [2] с., 1 л. портр.– (Филос. наследие). С.44-524.
    Гегель. Энциклопедия философских наук [Текст] / Гегель // Т.3 Философия духа. Отв. ред. Е.П. Ситниковский. Ред коллегия: Б.М. Кедров и др. М., «Мысль», 1997. –  471 с. (АН СССР, Ин-т философии. Философское наследие)
    Гелен А. О систематике антропологии [Текст] / А. Гелен // Проблема человека в западной философии: Переводы/Сост. и послесл. П. С. Гуревича; Общ. ред. Ю.Н. Попова. – М.: Прогресс, 1988. – С. 152-201.
    Декомб, В. Дополнение к субъекту: Исследование феномена действия от собственного лица / Пер. с фр. М. Голованивской. – М.: Новое литературное обозрение, 2011. – 576 с.
    Дильтей В. «Наброски к критике исторического разума» [Текст] / В. Дильтей // Вопросы философии: журнал. – 1988. -N 4. – С. 135-152.
    Диоген Лаэртский. О жизни, учениях и изречениях знаменитых философов / Диоген Лаэртский ; [перевод М. Гаспарова]. – Москва : Издательство АСТ, 2020. – 800 с. – (эксклюзивная классика). ISBN 978-5-17-119357-7
    Забелин И.М. Человек и человечество. – М.: Советский  писатель,  1970. –  264  с.
    Кирхин А. Суд над Сократом как один из поворотных моментов истории человечества [Электронный ресурс] / А. Кирхин. – Режим доступа: https://rapsinews.ru/historical_memory_publication/20240109/309527704.html (дата обращения: 15.08.2025 г.). 
    Комаров С.В. Проблема субъективности в трансцендентально-феноменологической традиции западной философии [Электронный ресурс] / С.В. Комаров. – Режим доступа: http://www.dissercat.com/content/problema-subektivnosti-v-transtsendentalno-fenomenologicheskoi-traditsii-zapadnoi-filosofii
    Кьеркегор С. Страх и трепет / Пер. с дат. Н.В. Исаевой, С.А. Исаева. – 2-е изд. – М.: Академический Проект, 2014. – 154 с. – (Философские технологии), 2025.
    Левин К. Теория поля в социальных науках / Курт Левин; Пер. с англ. Е.А. Сурпина. – 3-е изд.  – М.: Академический проект, 2025. – 356 с. –(Психологические технологии: Социальная психология).  
    Моисеев В.И. Исчисление форм как проект математической философии. [Текст] / В. Моисеев // Credo New: Журнальный клуб Интелрос. – 2014. – N 4.
    Моисеев В.И. Человек и общество: образы синтеза. Книга первая [Текст] / В. Моисеев // Мск.: Издательский дом «Навигатор», 2012. – 759 с.; ил.
    Моисеев В.И. Человек и Общество: образы синтеза. Книга вторая [Текст] / В. Моисеев // Мск.: Издательский дом «Навигатор», 2012. – 711 с.; ил.
    Моисеев В.И. Тайна Вильгельма Дильтея [Электронный ресурс] / В.И. Моисеев. – Режим доступа:  http://www.vyacheslav-moiseev.narod.ru/Papers/Diltey.htm
    Моисеев В.И. Манифест философии неовсеединства [Электронный ресурс] / В.И. Моисеев. – Режим доступа: https://eurasian-club.com/manifest-filosofii-neovseedinstva-3674/ html (дата обращения: 30.08.2025 г.).
    Ницше Ф. Сочинения в 2 т. Т.2 [Текст ] / Ницше Ф. // Пер. с нем.; Сост., ред. и авт. Примеч. К.А. Свасьян. – Мысль, 1990. – 829, [1] 1 л. портр.
    Платон. Апология Сократа [текст] / Платон // Собрание сочинений в 4 т. Т. 1 [Текст] /Платон. – М.: Мысль, 1994. – с. 70 – 96.
    Платон. Критон [текст] / Платон // Собрание сочинений в 4 т. Т. 1 [Текст] /Платон. – М.: Мысль, 1994. – с. 97 – 111.
    Платон. Государство [текст] / Платон // Собрание сочинений в 4 т. Т. 3 [Текст] /Платон. – М.: Мысль, 1994. – с. 79 – 420.
    Суд над Сократом. Сборник исторических свидетельств / Составитель    А. В. Кургатников. – СПб.: Алетейя, 2000. – 272 с. – (Античная   библиотека). ISBN 5-89329-015-1.  
    Поппер К.Р. Предположения и опровержения: Рост научного знания: Пер. с англ. / К.Р. Поппер. – М.: ООО «Издательство АСТ»: ЗАО YYG «Ермак», 2004. – 638, [2] с.
    Холкин О.М. Функциональная модель субъективности: гипотеза и практические применения / О. М. Холкин. – Киров: Изд-во ООО «ВЕСИ», 2016. – 151 с.
    Холкин О. М. К вопросу о построении структурно-функциональной модели правосознания (СФМП) [Электронный ресурс] / О.М. Холкин. – Режим доступа: Неволинские чтения. Вопросы совершенствования высшего юридического образования на современном этапе [Электронный ресурс]: сборник материалов Международной научно-практической конференции, посвященной 210-летию со дня рождения К.А. Неволина, 85-летию Университета имени О.Е. Кутафина (МГЮА) и 45-летию Волго-Вятского института (филиала) Университета имени О.Е. Кутафина (МГЮА). Киров, 18 ноября 2016 г. / Волго-Вятский институт (филиал) ФГБОУ ВО «Московский государственный юридический университет имени О.Е. Кутафина (МГЮА)». – Электрон. текстовые дан. (9 МБ). – Киров: Аверс, 2017. – с. 62-68.

    \paragraph{Ключевые слова:} {\itshape Казус Сократа, субъектность, структурно-функциональная модель субъектности (правосубъектности), модальностная триположность субъектности (правосубъектности), базовые режимы (начала): биономия, социономия, онто(ноо)номия; «периодический закон» субъектности (правосубъектности), закон модальностного возвышения}
    \end{russian}
\fi